%\vspace{-.1in}
\section{\Naive\ parameter inference via Metropolis-Hastings}
The Rao-Teh algorithm~\cite{RaoTeh13} simplifies
computation by conditioning on the
random grid $W$.
%by combining extant transition times $T$ with a set of 
%thinned candidate transition times $U$, sampled from a rate-$(\Omega-A_{S(t)})$ 
%Poisson process. 
This suggests conditioning on $W$ to update
the parameters as well, following the scheme from
Algorithm~\ref{alg:disc_time_mh}.
%The resulting scheme updates $\theta$ conditioned on the random 
%grid, but with the trajectory integrated out 
In particular, given $W$, discard all state information, and propose a 
new parameter $\vartheta$ from $q(\vartheta|\theta)$. 
%now conditioning on the set of times $W$.
The MH-acceptance probability is $\min\left(1,
\frac{P(X|W,\vartheta)P(W|\vartheta)p(\vartheta)q(\theta|\vartheta)}
     {P(X|W,\theta)P(W|\theta)p(\theta)q(\vartheta|\theta)}\right)$; 
     to calculate it,
make a forward pass over $W$, and calculate 
$P(X|W,\theta)$ and $P(X|W,\vartheta)$. % as in algorithm~\ref{alg:disc_time_mh}.
After accepting or rejecting $\vartheta$, the new parameter $\ntheta$ is used in
a backward pass that samples a new trajectory. Then discard all 
self-transitions, resample $W$ and repeat. Algorithm~\ref{alg:MH_naive}, and 
figure~\ref{fig:naive_mh} in the appendix sketch this out.

The resulting algorithm updates $\theta$ with the MJP trajectory 
integrated out, giving more rapid mixing.
However $\theta$ is still updated {\em conditioned on
$W$}, and 
%determines not just the MJP trajectory $S(t)$.
%, and with $S(t)$ 
%marginalized out, the observations $X$. 
the distribution of $W$ depends on $\theta$: 
%These are the $p(W|\theta)$ terms
%in the acceptance probability; under uniformization, 
$W$ is a homogeneous
Poisson process with rate $\Omega(\theta)$. %$ = 2 \max A(\theta)$. 
%who probability can
%calculated easily during the forward pass.
The fact that the MH-acceptance probability involves a $P(X|\theta)$ term
is inevitable, however we found that the $P(W|\theta)$
terms significantly affects acceptance probabilities. 
Any proposal that halves $\Omega(\theta)$ will halve the
mean and variance of the distribution of the number of events in $W$, 
resulting in a low acceptance probability.
This will affect mixing.
The next section describes an algorithm to get around this.
%\vspace{-.1in}
%\vspace{-.32in}
\begin{algorithm}[H]
   \caption{\Naive\  MH for parameter inference for MJPs }
   \label{alg:MH_naive}
  \begin{tabular}{l l}
   \textbf{Input:  } & \text{Observations $X$}, 
                       \text{the MJP path $S(t) = (S, T)$, the  parameters $\theta$ }and $\pi_0$.\\ 
                     & \text{A  Metropolis-Hasting proposal $q(\cdot | \theta)$}.\\
   \textbf{Output:  }& \text{A new MJP trajectory $S'(t) = (S', T')$, 
                            new MJP parameters $\theta'$}.\\
   \hline
   \end{tabular}
   \begin{algorithmic}[1]
     \State Set $\Omega \assign \Omega(\theta) > \max_s{A_s(\theta)}$ for
     some function $\Omega(\cdot)$ (e.g.\ $\Omega(\theta) = 
      2\max_s A_s(\theta))$.
      \State Sample thinned jumps $U\subset[0, t_{end}]$ from a 
      Poisson process with piecewise-constant rate 
      $R(t) = (\Omega - A_{S(t)})$. 
    Set $W = T \cup U$ and discard MJP state information.
      \State Propose $\vartheta \sim q(\cdot| \theta)$.
          The acceptance probability is given by 
%         \vspace{-.05in}
          \begin{align*}
          \alpha &=  1 \wedge \frac{P(\vartheta|W, X)}{P(\theta|W, X)} \frac{q(\theta|\vartheta)}{q(\vartheta|\theta)}
          =  1 \wedge \frac{P(X| W,\vartheta) P(W | \vartheta)p(\vartheta)}
            {P(X|W, \theta)P(W | \theta)p(\theta)} \frac{q(\theta|\vartheta)}{q(\vartheta|\theta)}.
          \end{align*}
%         \vspace{-.1in}
    \State For both $\theta$ and $\vartheta$, make a forward pass through the 
    elements of $W$, sequentially updating the distribution over states at 
    $w \in W$ given observations up to $w$. 
    For any $\theta$, the transition matrix 
    $B(\theta)$ equals $I + \frac{A(\theta)}{\Omega(\theta)}$ while the initial distribution
      over states is $\pi_0$. The likelihood of state $s$ at step $i$ is 
      $ L_i(s) = P(X_{[w_i, w_{i + 1})} | S(t) = s , t \in [w_i, w_{i + 1})) 
      $, where $X_{[w_i,w_{i+1})}$ is all observation lying in 
      $[w_i,w_{i+1})$. %= \prod_{o: t_o \in [w_i, w_{i + 1})}p(X_{t_o} | s)$.
    At the end, we have 
    $P(X|W,\theta)$ and $P(X|W,\vartheta)$. Use these, and the fact that 
    $P(W|\theta)$ is Poisson-distributed to accept or reject the
    proposed $\vartheta$. Write the new parameter as $\theta'$.
    %as \boqian{$(W,\theta,\vartheta)$}.
    \State For the new parameter $\theta'$, make a backward pass through 
    the elements of
    $W$, sequentially assigning a state to each element of $W$. This
    completes the FFBS algorithm.
    \State Let $T'$ be the set of times in $W$ when the Markov chain changes state. Define $S'$ as the corresponding set of state values. Return $(S', T', \theta')$.
\end{algorithmic}
\end{algorithm}
