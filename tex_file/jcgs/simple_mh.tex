\vspace{-.2in}
\section{\Naive\ parameter inference via Metropolis-Hastings}
%A natural approach to reduce the coupling in the Gibbs sampler is to reduce ?Bayesian fraction of missing information?. 
For discrete-time HMMs, path-parameter coupling can be circumvented by marginalizing out the Markov trajectory, and directly sampling from the marginal posterior $P(\theta|X)$.
In its simplest form, this involves a Metropolis-Hastings (MH) scheme that proposes a new parameter $\vartheta$ from a proposal distribution $q(\vartheta|\theta)$, accepting or rejecting according to the usual MH probability.
The marginal probabilities over $X$ given parameters are computed using the forward-filtering backward-sampling (FFBS) algorithm~\citep{fruhwirth1994data,carter1996markov, RaoTeh13}.
The Rao-Teh algorithm, which recasts posterior simulation for continuous-time models as discrete-time simulation on a random grid, then provides a simple way to incorporate such an MH-scheme into continuous-time settings: directly update $\theta$, conditioning on the random grid $W$, but marginalizing out $(v_0, V)$.
%, following the scheme from Algorithm~\ref{alg:disc_time_mh}.
%The resulting scheme updates $\theta$ conditioned on the random 
%grid, but with the trajectory integrated out 

Specifically, given $\theta$ and the Poisson grid $W$, rather than simulating new path values (the backward pass in algorithm~\ref{alg:Unif_gibbs}), and then conditionally updating $\theta$ (the second step in algorithm~\ref{alg:MJP_gibbs}), we {\em first} propose a parameter $\vartheta$ from $q(\vartheta|\theta)$. This is accepted with probability 
$$ \texttt{acc} = \min\left(1, 
\frac{P(X|W,\vartheta) P(W|\vartheta)P(\vartheta)q(\theta|\vartheta)} {P(X|W,\theta) P(W|\theta)P(\theta)q(\vartheta|\theta)}\right),$$ 
%Only after accepting or rejecting do we simulate state values via a backward pass.
thereby targeting the distribution $P(\theta|W,X)$.
In the equation above, $P(X|W,\theta)$ is the probability of the observations $X$ given $W$ with $(v_0,V)$ marginalized out. 
Uniformization says this is the marginal probability of $X$ under a discrete-time HMM on $W$, with transition matrix $B(\theta)$. This can be computed using the forward pass of FFBS algorithm (steps 4 and 6 of algorithm~\ref{alg:MH_naive} below). 
The term $P(W|\theta)$ is the probability of $W$ under a rate-$\Omega(\theta)$ Poisson process. 
These, and the corresponding terms for $\vartheta$ allow the acceptance probability to be computed.
%make a forward pass over $W$, and calculate and $P(X|W,\vartheta)$. % as in algorithm~\ref{alg:disc_time_mh}.
Only {\em after} accepting or rejecting $\vartheta$ do we simulate new states $(v'_0,V')$, using the new parameter $\theta'$ in a backward pass over $W$. 
%Then discard all self-transitions, resample $W$ and repeat. 
The new trajectory and parameter are used to simulate a new grid $W'$, and the process is repeated.
Algorithm~\ref{alg:MH_naive} includes all details of this algorithm (see also figure~\ref{fig:naive_mh} in the supplementary material).

%\vspace{-.1in}
%\vspace{-.32in}
\begin{algorithm}[H]
   \caption{\Naive\  MH for parameter inference for MJPs }
   \label{alg:MH_naive}
  \begin{tabular}{l l}
   \textbf{Input:  } & \text{Observations $X$}, 
                       \text{the MJP path $(s_0, S, T)$, the  MJP parameters $\theta$ }and $\pi_0$.\\ 
      %               & \text{A  Metropolis-Hasting proposal $q(\cdot | \theta)$}.\\
   \textbf{Output:  }& \text{A new MJP trajectory $(s'_0, S', T')$, 
                            new MJP parameter $\theta'$}.\\
   \hline
   \end{tabular}
   \begin{algorithmic}[1]
     \State Set $\Omega(\theta) > \max_s{A_s(\theta)}$ for
     some function $\Omega(\cdot)$, e.g.\ $\Omega(\theta) = 
      2\max_s A_s(\theta)$.
      \State \textbf{Simulate the thinned times $U$} from a rate-$(\Omega(\theta)-A_{S(\cdot)}(\theta))$ Poisson process: 

      $\qquad \qquad \qquad \qquad U \sim \text{PoissProc}(\Omega(\theta) - A_{S(t)}(\theta)), \quad t\in[0,t_{end})$.

      \State 
      Set $W = T \cup U$ and discard $(s_0,S)$. Define $\tilde{W} = 0 \cup W \cup t_{end}$.
    \State 
    \textbf{Forward pass:}
    Set $B(\theta) = I + \frac{1}{\Omega(\theta)}A(\theta)$ and
    $\fwd^\theta_0(\cdot) = \pi_0$. Recall $\ell_i(\cdot)$ from equation~\eqref{eq:lik_factor}.
 %   Sequentially update $\fwd^\theta_i(\cdot)$ at time $w_i \in W$ as: 
\vspace{-.25in}
$$\textbf{for } i=1\rightarrow |{W}|\textbf{ do:} \quad \fwd^{\theta}_i(s') = \sum_{s \in \cS} \fwd^\theta_{i-1}(s)\cdot \ell_{i}(s) \cdot B_{ss'}(\theta), \quad \forall s' \in \cS.\qquad\qquad\quad $$
    \State \textbf{Propose $\vartheta \sim q(\cdot| \theta)$.}
    For elements of ${W}$, calculate $\fwd^\vartheta_i(\cdot)$ similar to above.
      \State \textbf{Accept/Reject:} 
      For $\theta$ (and similarly for $\vartheta$), set  
      $P(W|\theta) = \Omega(\theta)^{|W|}\exp(-\Omega(\theta)t_{end})$, 
      $P(X|W,\theta) = \sum_{s \in \cS} \fwd_{|{W}|}^\theta(s)\cdot\ell_{|{W}|+1}(s)$.
      With probability $\texttt{acc}$, set $\theta' = \vartheta$, else $\theta'=\theta$; %Here 
      %The acceptance probability for $\vartheta$ is given by 
%         \vspace{-.05in}
          \begin{align}
            \label{eq:ncp_acc}
            \texttt{acc} &=  1 \wedge \frac{P(\vartheta|W, X)}{P(\theta|W, X)} \frac{q(\theta|\vartheta)}{q(\vartheta|\theta)}
          =  1 \wedge \frac{P(X| W,\vartheta) P(W | \vartheta)P(\vartheta)}
            {P(X|W, \theta)P(W | \theta)P(\theta)} \frac{q(\theta|\vartheta)}{q(\vartheta|\theta)}.
          \end{align}
%         \vspace{-.1in}
          %$P(X|W,\vartheta) = \sum_{s \in \cS} \fwd_{|W|}^\vartheta(s) $. Use these, and the fact that $P(W|\theta)$ is Poisson-distributed to accept or reject the proposed $\vartheta$. Write the new parameter as $\theta'$.
    %as \boqian{$(W,\theta,\vartheta)$}.
    \State %For the new parameter $\theta'$, simulate states $V$. 
    \textbf{Backward pass:}
    Simulate $v_{|W|} \sim \bck^{\theta'}_{|W|}(\cdot)$, where $\bck^{\theta'}_{|W|}(s) \propto \fwd^{\theta'}_{|W|}(s)\cdot\ell_{|W|+1}(s) \quad \forall s \in \cS.$ 
    %at time $w_i$ given $v_{i+1}$  at time $w_{i+1}$:
%\vspace{-.1in}
\vspace{-.25in}
    $$ \textbf{for } i=(|W|-1)\rightarrow 0\textbf{ do:} \quad v_i \sim \bck^{\theta'}_i(\cdot),\ \ \text{where } 
    \bck^{\theta'}_i(s) \propto \fwd^{\theta'}_i(s)\cdot B_{sv_{i+1}}(\theta') \cdot \ell_{i+1}(s)  \ \forall s \in \cS.$$
   % Here, $\bck^{\theta'}_{|W|}(\cdot) = \fwd^{\theta'}_{|W|}(\cdot)$.  This completes the FFBS algorithm.
    \State Set $s'_0=v_0$. Let $T'$ be the set of times in $W$ when $V$ changes state. Define $S'$ as the corresponding set of state values. Return $(s'_0, S', T', \theta')$.
\end{algorithmic}
\end{algorithm}
\vspace{-.1in}
The resulting MCMC algorithm updates $\theta$ with the MJP trajectory 
integrated out, and by instantiating less `missing' information, can be expected to mix better. 
This can be quantified by the so-called Bayesian fraction of missing information~\citep{liu1994fraction, papaspiliopoulos2007general}. 
%The Gibbs sampler of algorithm~\ref{alg:MJP_gibbs} can be viewed as operating on a centered parametrization~\citep{papaspiliopoulos2007general} or sufficient augmentation~\citep{yu2011center} of a hierarchical model involving $\theta$, the Poisson events $W$, and the state values $(v_0,V)$. The MH algorithm reverses the order in which the path and parameter are updated, and is closely related to noncentered parametrizations or ancillary augmentations. 
%For a detailed review of the suitability of these two approaches, as well as ways to combine them together, we refer to~\citet{papaspiliopoulos2007general, yu2011center}.
%
We note that even with the state values $(v_0,V)$ marginalized out, $\theta$ is updated {\em conditioned on $W$}. 
The distribution of $W$ depends on $\theta$: $W$ follows a rate-$\Omega(\theta)$ Poisson process. This dependence manifests in the $P(W|\theta)$ and $P(W|\vartheta)$ terms in equation~\eqref{eq:ncp_acc}. 
The fact that the MH-acceptance involves the probability of the observations  $X$ is inevitable, however the $P(W|\theta)$ term is an artifact of the computational algorithm of Rao-Teh. 
In our experiments, we show that this term significantly hurts acceptance probabilities and mixing. 
For a given $\theta$, $|W|$ is Poisson distributed with mean and variance $\Omega(\theta)$. 
If the proposed $\vartheta$ is such that $\Omega(\vartheta)$ is half $\Omega(\theta)$, then the ratio $P(W|\vartheta)/P(W|\theta)$ will be small, and $\vartheta$ is unlikely to be accepted. 
%, resulting in a low acceptance probability.  
%This will slow down mixing.
The next section describes our main algorithm that gets around this.
