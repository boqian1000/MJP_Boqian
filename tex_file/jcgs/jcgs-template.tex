\documentclass[12pt]{article}
\usepackage{amsmath}
\usepackage{graphicx,psfrag,epsf}
\usepackage{enumerate}
\usepackage{natbib}
\usepackage{amsfonts}
\usepackage{amsmath}
\usepackage{algorithm}
\usepackage{algorithmic}
\usepackage{amssymb,mathabx}

\newcommand{\blind}{0}

\addtolength{\oddsidemargin}{-.75in}%
\addtolength{\evensidemargin}{-.75in}%
\addtolength{\textwidth}{1.5in}%
\addtolength{\textheight}{1.3in}%
\addtolength{\topmargin}{-.8in}%


\begin{document}


%\bibliographystyle{natbib}

\def\spacingset#1{\renewcommand{\baselinestretch}%
{#1}\small\normalsize} \spacingset{1}


%%%%%%%%%%%%%%%%%%%%%%%%%%%%%%%%%%%%%%%%%%%%%%%%%%%%%%%%%%%%%%%%%%%%%%%%%%%%%%

\if0\blind
{
  \title{\bf Efficient MCMC Sampling Finite-State Markov Jump Processes and Bayesian Inference}
  \author{Vinayak Rao\thanks{
    The authors gratefully acknowledge}\hspace{.2cm}\\
    Department of Statistics, Purdue University\\
    and \\
    Boqian Zhang \\
    Department of Statistics, Purdue University}
  \maketitle
} \fi

\if1\blind
{
  \bigskip
  \bigskip
  \bigskip
  \begin{center}
    {\LARGE\bf Title}
\end{center}
  \medskip
} \fi

\bigskip
\begin{abstract}
Abstrct.
\end{abstract}

\noindent%
{\it Keywords:}  Markov jump process, MCMC, Metropolis Hasting sampler, Bayesian inference

\spacingset{1.45}
\section{Introduction}
\label{sec:intro}
In this paper, we tackled the problem of sampling MJP parameters from the posteriors, efficiently, using Metropolis Hasting algorithm.
\section{Metropolis Hasting for Bayesian Inference using FFBS  within the Gibbs Sampling On MJPs}~

\begin{algorithm}[tb]
   \caption{MH In Gibbs sampling for MJPs }
   \label{alg:MH In Gibbs}
\begin{algorithmic}
   \STATE {\bfseries Input:} A set of partial and noisy observations $y_{[t_0, t_{N+1})}$, Initial distribution over states $\pi_0$,  Metropolis Hasting proposal $q(. | \theta)$.\\
   The previous MJP path $S(t) = (S, T)$, the previous MJP parameters $\theta$.\\
  
      \STATE {\bfseries Output:} A new MJP trajectorie $\tilde{S} (t) = (\tilde{S}, \tilde{T})$, A series of MJP parameters $\tilde{\theta}$.

%   \STATE Initialize, $i = 0$
%   \\ (a) Set $\theta(0)$ arbitrarily and set current trajectory $[S,T](0)$ arbitrarily.\\
%	(b) Uniformize $[S,T](0)$, to get virtual jumps $U$.
%   \REPEAT
%   \FOR{$i=1$ {\bfseries to} $N$}
	\STATE 0: Let $\Omega = h(\theta)$, with $\Omega > max_s{|A_s|}$ using some deterministic function $h$.
	\STATE 1: Sample virtual jumps $U\subset[t_{start}, t_{end}]$ from a Non homogeneous Poisson process with piecewise-constant rate$$R(t) = (\Omega + A_{S(t)}).$$\\Define $W = T \cup U$.
	\STATE 2: Propose $\theta^* \sim q(.| \theta)$.\\
		Accept $\theta^*$ as $\tilde{\theta}$ with probability $\alpha$.
		\begin{align*}
		\alpha &=  1 \wedge \frac{P(W,\theta^*| y)}{P(W, \theta| y)} \frac{q(\theta|\theta^*)}{q(\theta^*|\theta)}\\
		&=  1 \wedge \frac{P(y| W,\theta^*) P(W | \theta^*)p(\theta^*)}{P(y|W, \theta)P(W | \theta)p(\theta)} \frac{q(\theta|\theta^*)}{q(\theta^*|\theta)}.
		\end{align*}
	\STATE 3: Sample a path $\tilde{V}$, from a discret-time Markov chain with $|W| + 1$ steps, using FFBS algorithm. The transition matrix of the Markov chain is $B = (I + \frac{A}{\Omega})$ while the initial distribution over states is $\pi_0$. The likelihood of state $s$ at step $i$ is 
	$$ L_i(s) = P(Y_{[w_i, w_{i + 1})} | S(t) = s \; for\; t \in [w_i, w_{i + 1})) = \prod_{j: t_j \in [w_i, w_{i + 1})}p(y_{t_j} | S(t_j) = s).$$\\
%(i.e. $V(i) \sim P(V |  \theta(i), W(i - 1), y).$) Then delete all the virtual jumps to get $S(i), T(i) .$\\
	\STATE 4: Let $\tilde{T}$ be the set of times in $W$ when the Markov chain changes state. Define $\tilde{S}$ as the corresponding set of state values. Return $(\tilde{S}, \tilde{T}, \tilde{\theta})$.\\

%   \ENDFOR
%   \UNTIL{$ i = M$ }
\end{algorithmic}
\end{algorithm}

\label{sec:meth}

\section{Metropolis Hasting for Bayesian Inference using FFBS  within the Gibbs Sampling On MJPs}~
\begin{algorithm}[tb]
   \caption{MH In Gibbs sampling for MJPs }
   \label{alg:MH In Gibbs}
\begin{algorithmic}
   \STATE {\bfseries Input:} A set of partial and noisy observations $y_{[t_0, t_{N+1})}$, Initial distribution over states $\pi_0$,  Metropolis Hasting proposal $q(. | \theta)$.\\
   The previous MJP path $S(t) = (S, T)$, the previous MJP parameters $(\theta)$.\\
  
      \STATE {\bfseries Output:} A new MJP trajectorie $\tilde{S} (t) = (\tilde{S}, \tilde{T})$, A series of MJP parameters $\tilde{\theta}$.

%   \STATE Initialize, $i = 0$
%   \\ (a) Set $\theta(0)$ arbitrarily and set current trajectory $[S,T](0)$ arbitrarily.\\
%	(b) Uniformize $[S,T](0)$, to get virtual jumps $U$.
%   \REPEAT
%   \FOR{$i=1$ {\bfseries to} $N$}
	\STATE 0:  Sample $\theta^* \sim q(.| \theta)$. And let $\Omega = h(\theta) + h(\theta^*)$, with $h(\theta) > max_s{|A_s(\theta)|}$, $h(\theta^*) > max_s{|A_s(\theta^*)|}$ using some deterministic function $h$.
	\STATE 1: Sample virtual jumps $U\subset[t_{start}, t_{end}]$ from a Non homogeneous Poisson process with piecewise-constant rate$$R(t) = (\Omega + A_{S(t)}(\theta)).$$\\Define $W = T \cup U$.
	\STATE 2: Propose $(\theta^*, \theta)$ and accept $\theta^*$ as $\tilde{\theta}$ with probability $\alpha$.
		\begin{align*}
		\alpha &=  1 \wedge \frac{P(W,(\theta^*, \theta)| y)}{P(W, (\theta, \theta^*)| y)}\\
		&=  1 \wedge \frac{P(y| W,\theta^*, \theta) P(W | (\theta^*, \theta))p((\theta^*, \theta))}{P(y| W,(\theta, \theta^*)) P(W | (\theta, \theta^*))p((\theta, \theta^*))}\\
				&=  1 \wedge \frac{P(y| W,\theta^*, \theta)p((\theta^*, \theta))}{P(y| W,(\theta, \theta^*))p((\theta, \theta^*))}.
		\end{align*}
	\STATE 3: Sample a path $\tilde{V}$, from a discret-time Markov chain with $|W| + 1$ steps, using FFBS algorithm. The transition matrix of the Markov chain is $B = (I + \frac{A(\tilde{\theta})}{\Omega})$ while the initial distribution over states is $\pi_0$. The likelihood of state $s$ at step $i$ is 
	$$ L_i(s) = P(Y_{[w_i, w_{i + 1})} | S(t) = s \; for\; t \in [w_i, w_{i + 1})) = \prod_{j: t_j \in [w_i, w_{i + 1})}p(y_{t_j} | S(t_j) = s).$$\\
%(i.e. $V(i) \sim P(V |  \theta(i), W(i - 1), y).$) Then delete all the virtual jumps to get $S(i), T(i) .$\\
	\STATE 4: Let $\tilde{T}$ be the set of times in $W$ when the Markov chain changes state. Define $\tilde{S}$ as the corresponding set of state values. Return $(\tilde{S}, \tilde{T}, \tilde{\theta})$.\\

%   \ENDFOR
%   \UNTIL{$ i = M$ }
\end{algorithmic}
\end{algorithm}

\section{Verifications of Algorithm 1}
\label{sec:verify}
Proof of Algorithm 1:\\
\noindent Assume: $S = [S_0,S_1, ...,S_N] \;, T = [T_0, T_1,...,T_N, T_{N+1}(T_{end})]$, and y as observations.\\

%\begin{proof}

In JMLR-2013 Fast MCMC Sampling for MJP and Extensions, the FFBS frame contains $\alpha_t$ as follows.\\
Since after uniformization, the virtual jumps are added.  Then the state process of the trajectory with virtual jumps is just a discrete time markov jump process. The key point is that we need to have $W$ be conditioned, to get the marginal probability $P(y_{[T_0, T_{N + 1})} | \theta, W)$ from FFBS algorithm. \\
\begin{align*}
 \alpha^\theta_t(s) = P(S_t = s\;, y_{[T_0, T_t)}, U, T).\\
P(y_{[T_0, T_{N + 1})} | \theta, W) = \sum_{s = 0}^{N-1} \alpha^\theta_N(s) \cdot P(y_{[T_N, T_{N+1})} | S_N = s).\\
P(\theta, W| y) \propto P(\theta, W, y) = P(y| W, \theta) P(W | \theta) P(\theta).
\end{align*}

$P(y|W, \theta)$ is the marginal probability we get after Forward Filtering Algorithm and the $P(W | \theta)$ is the probability density for the $poisson(\Omega)$, because of the uniformization procedure.
Let denote the kernel for (a), (b) and (c) as  $\kappa_1(\theta^*| \theta, W, T, S, y)$ , $\kappa_2(S^*, T^*|S, T, W, \theta^*, y)$ and $\kappa_3(W^*| S^*, T^*, \theta^*, y)$.\\
For Step (a) $\kappa_1(\theta^*| \theta, W, T, S)$:\\
 \begin{align*} 
P((W, T, S, \theta) \rightarrow (W, T, S, \theta^*)) P(\theta, W | y) &=  P(\theta^*, W | y)q(\theta | \theta^*)
 \wedge P(\theta, W|y) q(\theta^*| \theta) \\&= P((W, T, S,\theta^*) \rightarrow (W, T, S,\theta)) P(\theta^*, W | y).\\
\end{align*}
$\therefore  \int \kappa_1(\theta^*| \theta) P(\theta, W|y) d\theta = P(\theta^*, W |y). $\\
So the stationary distribution of $\kappa_1$ is $P(\theta, W | y)$.\\
Step (b) $\kappa_2(S^*, T^*|S, T, W, \theta^*, y)$: \\ 
Step(b) is the same as Fast MJPs Gibbs sampling scheme.   \\
\begin{align*}
((S, T, \theta, W) \rightarrow (S^*, T^*,\theta, W)|  y) = P(V^* | W, \theta, y) = P(V^* | W, \theta, y) / P(W, \theta, y) \end{align*}
\begin{align*}
P((S, T) \rightarrow (S^*, T^*)| W, \theta, y) P(S, T| W, \theta, y) &= P(V^* | W, \theta, y)P(V | W, \theta, y) \\&= P((S^*, T^*) \rightarrow (S, T)| W, \theta, y) P(S^*, T^*| W, \theta, y)
\end{align*}
\\So the stationary distribution of $\kappa_2(S^*, T^*| S,T,  W, y)$ is $P(S, T | W, \theta, y).$
Now, let's consider $\kappa_2 \circ \kappa_1(S^*, T^*, \theta^* | S, T, \theta, y, W)$.\\
\begin{align*}
((S, T, \theta, W) \rightarrow (S^*, T^*, \theta^*, W)|  y) = P((W, T, S, \theta) \rightarrow (W, T, S, \theta^*)) P((S, T, \theta^*.W) \rightarrow (S^*, T^*, \theta^*, W)| y) .
\end{align*}
The stationary distribution of $\kappa_1(S^*, T^*, U^*|S, T, U)$ is $P(S,T,U| \theta, y).$ And the stationary distribution of $\kappa_2(U^*| U)$ is $P(U| S, T, \theta, y).$ \\
\begin{align*}
&P((S, T, \theta, W) \rightarrow (S^*, T^*, \theta^*, W)|  y) P(S,T,\theta | W, y) \\&= P((W, T, S, \theta) \rightarrow (W, T, S, \theta^*))\cdot P(\theta|W,y) \cdot P((S, T, \theta^*.W) \rightarrow (S^*, T^*, \theta^*, W)| y) P(S, T | \theta , W, y) \\&=P((W, T, S, \theta^*) \rightarrow (W, T, S, \theta))\cdot P(\theta^*|W,y) \cdot P((S^*, T^*, \theta^*.W) \rightarrow (S, T, \theta^*, W)| y) P(S^*, T^* | \theta , W, y) \\&=P((S^*, T^*, \theta^*, W) \rightarrow (S, T, \theta, W)|  y) P(S,T,\theta | W, y).
\end{align*}
So the stationary distribution of $\kappa_2 \circ \kappa_1$ is $P(S, T,\theta| W,y).$\\
Obviously, $\kappa_3(W^*| W, S^*, T^*, \theta^*, y)$ has $P(W| S^*, T^*, \theta^*,y)$ as stationary distribution.\\
Therefore, $\int \kappa_3(W^*| W, S^*, T^*, \theta^*, y) P(W,S^*, T^*, \theta^*|y ) dW= P(W^*,S^*, T^*, \theta^*|y )$.\\
Thus, $\int \kappa_3 \cdot (\int \kappa_2 \circ \kappa_1\cdot P(W,S, T, \theta | y) d\theta dS dT)dW = \int \kappa_3 P(W,S^*, T^*, \theta^*|y )dW = P(W^*,S^*, T^*, \theta^*|y )$.\\
So the stationary distribution of $\kappa_3 \circ \kappa_2 \circ \kappa_1$ is $P(W^*,S^*, T^*, \theta^*|y )$.
%\end{proof}

\section{Verifications of Algorithm 2}
\label{sec:verify}
Proof:
Our state is $(W, S, T, \theta, \theta^*)$. 
\begin{align*}
 p(y, W, S, T, \theta, \theta^*) &= p(\theta) q(\theta^* | \theta) P(S,T| \theta, \theta^*) P(W| S, T, \theta, \theta^*)P(y | S, T, \theta, \theta^*)\\
 &=p(\theta) q(\theta^* | \theta) P(S,T| \theta) P(W| S, T, \theta, \theta^*)P(y | S, T).
\end{align*}
The marginal distribution of $(y, S, T, \theta, \theta^*)$ and $(y, S, T, \theta)$ as follows.\\
\begin{align*}
 p(y, S, T, \theta, \theta^*) &= p(\theta) q(\theta^* | \theta) P(S,T| \theta, \theta^*)P(y | S, T, \theta, \theta^*)\\
 &=P(y, S, T, \theta) q(\theta^* | \theta).
\end{align*}
\begin{align*}
 p(y, S, T, \theta) &= p(\theta)P(S,T| \theta)P(y | S, T, \theta).
\end{align*}
So the conditional distribution over $\theta^*$ given $(y, S, T, \theta)$ is $q(\theta^* | \theta)$. And the conditional distribution over W given $(y, S, T, \theta, \theta^*)$  is $P(W | S, T, \theta, \theta^*)$, which is actually the distribution of Non Homogeneous Poisson Process with piecewise constant rate $h(\theta) + h(\theta^*) - A_{S(t)}(\theta)$.\\
Thus the Step 1 + Step 2 is actually equivalent to sampling from the conditional distribution $P(\theta^* , W| S, T, \theta, y)$.\\
The Step 3 + Step 4 satisfy the detailed balance condition. The reason is as follows.
\begin{align*}
&P((W, S, T, (\theta, \theta^*)) \rightarrow (W, S^*, T^*, (\theta^*, \theta))) P(S,T, (\theta, \theta^*) | W, y)\\
&= (1 \wedge \frac{P((\theta^*,\theta) | W, y)}{P((\theta,\theta^*) | W, y)})P(S^*, T^* | W, (\theta^*, \theta), y)P(S, T | W, (\theta, \theta^*), y)P((\theta, \theta^*) | W, y)\\
&= P((W, S^*, T^*, (\theta^*, \theta)) \rightarrow (W, S, T, (\theta, \theta^*))) P(S^*,T^*, (\theta^*, \theta) | W, y)
\end{align*} 
Therefore the stationary distribution of this MCMC sampler is $P(W, S, T, (\theta, \theta^*) | y)$. Thus the stationary distribution of $(S, T, \theta)$ is the corresponding marginal distribution $P(S, T, \theta | y)$.  

\section{Exponential Model which do not have conjugate posterior}~

\noindent Assume: $S = [S_0,S_1, ...,S_N] \;, T = [t_0(t_{start}), t_1,...,t_N, t_{N+1}(t_{end})]$, and y as observations.\\
We consider a specific structure of rate matrix $A$. $A_{ij} = \alpha f_{ij}(\beta), \; i \neq j$. $A_{ii} = -\sum_{j \neq i} A_{ij}$. $0 \leq f_{ij} \leq 1$. Denote $F_i(\beta) = \sum_{j \neq i} f_{ij}(\beta)$.\\
\begin{align*}
P(s_0, S, T | \alpha, \beta) &= \pi_0(s_0)\prod_{i = 1}^N A_{S_{i - 1}S_i} \exp(- \int_{t_{start}}^{t_{end}} |A_{S(t)}| dt)\\
&= \pi_0(s_0) \alpha^N \prod_{i = 1}^N f_{S_{i - 1}S_i} \exp(-\alpha  \sum_{i = 0}^{N} F_{S_i}(\beta)(t_{i + 1} - t_i))\\
\end{align*}
Assume the prior distributions of $\alpha, \beta$ are $p_1(\alpha)$ and $p_2(\beta)$.\\
Then the posterior distribution of parameters $\alpha, \beta$ will be as follows.\\
\begin{align*}
P(\alpha, \beta | s_0, S, T ) \propto \alpha^N \prod_{i = 1}^N f_{S_{i - 1}S_i} \exp(-\alpha  \sum_{i = 0}^{N} F_{S_i}(\beta)(t_{i + 1} - t_i)) p_1(\alpha)p_2(\beta)\\
\end{align*}
If we assume the priors of $\alpha$, $\beta$ are $Gamma(\mu, \lambda)$, $Gamma(\omega, \theta)$, then the posterior will have a simper form as follows. 
\begin{align*}
P(\alpha, \beta | s_0, S, T ) = C \alpha^{\mu + N - 1}\exp(-\alpha (\lambda + \sum_{i = 0}^{N} F_{S_i}(\beta)(t_{i + 1} - t_i))) \prod_{i = 1}^N f_{S_{i - 1}S_i}  \beta ^{\omega - 1} \exp(-\theta \beta)\\
\end{align*}
We notice that given $\beta,\; S,\; T$, $\alpha$ is distributed as a $Gamma$ distribution.\\
$\alpha | \beta, S, T, y  = \alpha | \beta, S, T \sim Gamma(\mu + N, \lambda + \sum_{0}^NF_{S_i}(\beta)(t_{i + 1} - t_i))$.\\
But there is no conjugate distribution to sample $\beta \sim P(\beta| s_0, S, T)$. We will have to use Metropolis Hasting within Gibbs to sample $\beta$.\\
$$ P(\beta | S, T) = C \frac{\prod_{i = 1}^N f_{S_{i -1}S_i}(\beta)\beta^{\omega - 1} \exp(-\theta \beta)}{(\lambda + \sum_{i = 0}^{N} F_{S_i}(\beta)(t_{i + 1} - t_i))^{\mu + N}}$$
\begin{algorithm}[tb]
   \caption{Generic Gibbs sampling for MJPs for Gamma priors}
   \label{alg:Generic Gibbs}
\begin{algorithmic}
   \STATE {\bfseries Input:} observations $y_{[t_0, t_{k+1})}$
   \STATE Initialize, $i = 0$
   \\ (a) Set $\alpha(0), \beta(0)$ arbitrarily and set current trajectory $[S,T](0)$ arbitrarily.\\
	(b) Uniformize $[S,T](0)$, to get virtual jumps $U$.
   \REPEAT
   \FOR{$i=1$ {\bfseries to} $N$}
	\STATE (a) Sample $U(i) \sim P( U | \beta(i - 1), \alpha(i - 1), S(i - 1), T(i - 1), y)$.\\	
	\STATE (b) Use FFBS algorithm to  sample states given all the jump times(both true jumps and virtual jumps).
(i.e. $V(i) \sim P(V |  \beta(i - 1), \alpha(i - 1), W(i ), y).$) Then delete all the virtual jumps to get $S(i), T(i) .$\\
	\STATE (c) Propose $\beta^* \sim q(.| \beta(i -1))$.\\
	  Set $\beta(i) = \beta^*$, with probability $P_{acc} = 1 \wedge \frac{P(\beta^* |S(i), T(i))}{P(\beta(i - 1) |S(i), T(i))} \frac{q(\beta(i - 1)|\beta^*)}{q(\beta^*|\beta(i - 1))}$;\\Otherwise set $\beta(i) = \beta(i-1)$.\\	  
	\STATE (d) Sample $\alpha(i) \sim P(. | \beta(i), S(i), T(i), y)$.\\ It is a $Gamma(\mu + N, \lambda + \sum_{0}^NF_{S_i}(\beta)(t_{i + 1} - t_i))$ distribution actually.\\
   \ENDFOR
   \UNTIL{$ i = N$ }
\end{algorithmic}
\end{algorithm}

\section{Immigration models with capability}~

Now, let's consider a immigration model as follows. We have state space${0, 1, 2, ..., N}$, representing the total population. The transition matrix is defined as follows. 
$$A_i =: A_{i,i} = -(\alpha + i\beta), \; \; i =0,1,...,N$$ $$A_{i, i+1} = \alpha, \; \; i =0,1,...,N-1,$$ $$A_{i, i-1}  = \beta, \; \;  i =1,...,N.$$
We already know the conditional density(given $\alpha,\; \beta$) of a MJP trajectory $(s_0, S, T)$ in time interval $[t_{start}, t_{end}]$, with $S=(s_1, s_2,..., s_k)$, $T=(t_1, t_2,..., t_k)$. 
$$f(s_0,S,T| \alpha, \beta) = \prod_{i=0}^{k-1} A_{s_i, s_{i+1}} \exp(\sum_{i=0}^{k} A_{s_i}(t_{i+1} - t_{i})), $$
where $t_0 = t_{start}$, $t_{k+1} = t_{end}.$\\
Let's denote some notations here.\\
$$U(s_0, S, T):= \sum_{i=0}^{k-1} \mathbb{I}_{\{s_{i+1} - s_i = 1\}}$$
$$D(s_0, S, T):= \sum_{i=0}^{k-1} \mathbb{I}_{\{s_{i+1} - s_i = -1\}}$$
Call them U and D for short.
Let's denote the total time when the trajectory state stays at state i as $\tau_i$, i.e. $\tau_i = \sum_{j=0}^{k} (t_{j+1} -t_j)\mathbb{I}_{\{s_j = i\}}$, then $\sum_{i=0}^k (t_{i+1} - t_i)s_i = \sum_{i=0}^N \tau_ii$\\

$$f(s_0,S,T| \alpha, \beta) = \exp(-\alpha(t_{end} - t_{start}- \tau_N) )\alpha^U \cdot  \exp((-(\sum_{i=0}^k (t_{i+1} - t_i)s_i)\beta) \prod_{i=1}^N i^{\sum_{j=0}^{k-1}\mathbb{I}_{s_{j+1} = i -1 \;,  s_j = i} }   \beta^D$$\\
If we assume the prior of $\alpha$, and $\beta$ are $Gamma(\mu,\lambda)$, $Gamma(\omega, \theta)$, which are independent with each other. \\
$$p(\alpha) = \frac{\lambda^\mu}{\Gamma(\mu)}\alpha^{\mu -1}e^{-\lambda \alpha} $$.
$$p(\beta) = \frac{\theta^\omega}{\Gamma(\omega)}\beta^{\omega -1}e^{-\theta \beta} $$.
Then we can get the posterior distribution $$f(\alpha, \beta | s_0,S,T)$$ as follows.
$$ f(\alpha, \beta | s_0,S,T) \propto \exp(-(\lambda + t_{end} - t_{start}- \tau_N)\alpha) \alpha^{\mu + U -1} \cdot \exp(-(\sum_{i=0}^k (t_{i+1} - t_i)s_i + \theta)\beta) \beta^{\omega+ D -1}.$$
It means that the posterior distributions of $\alpha$, $\beta$ are still independent. \\
$\alpha | s_0,S,T$ is following $Gamma(\mu+ U,\lambda + t_{end} - t_{start}- \tau_N)$\\
$\beta | s_0,S,T$ is following $Gamma(\omega+ D,\theta + \sum_{i=0}^k (t_{i+1} - t_i)s_i)$, which is equivalent to $Gamma(\omega+ D,\theta +\sum_{i=0}^N \tau_ii)$\\

\section{Conclusion}
\label{sec:conc}


\bigskip
\begin{center}
{\large\bf SUPPLEMENTAL MATERIALS}
\end{center}

\begin{description}

\item[Title:] Brief description. (file type)

\item[R-package for  MYNEW routine:] R-package �MYNEW� containing code to perform the diagnostic methods described in the article. The package also contains all datasets used as examples in the article. (GNU zipped tar file)

\item[HIV data set:] Data set used in the illustration of MYNEW method in Section~ 3.2. (.txt file)

\end{description}

\begin{thebibliography}{}

\bibitem[Azzalini(2005)]{azza:05}
Azzalini, A. (2005).
\newblock The skew-normal distribution and related multivariate families.
\newblock \emph{Scandinavian Journal of Statistics} \textbf{32}, 159--188.

\end{thebibliography}{}

\end{document}
