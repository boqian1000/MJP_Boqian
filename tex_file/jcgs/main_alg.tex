
\section{An improved Metropolis-Hasting algorithm}
%\vspace{-.05in}
Our main idea is to symmetrize the probability of $W$ under the old and 
proposed parameters, so that 
$P(W|\theta)$ disappears from the acceptance ratio. This results
in a significantly more efficient, and also a simpler MCMC scheme.
%This forms the main contribution of this paper.
As before, the MCMC iteration begins with the pair $(S(t), \theta)$. 
Instead of simulating the Poisson events $U$, we first generate a new 
parameter $\vartheta$ from $q(\vartheta|\theta)$. Treat this as an 
auxiliary variable, so that the augmented space now is the triplet 
$(S(t), \theta,\vartheta)$. We pretend $S(t) \equiv (S,T)$ was sampled by  
uniformization, where the dominating Poisson rate $\Omega$ equals 
$(\Omega(\theta) + \Omega(\vartheta))$ instead of just $\Omega(\theta)$ 
(recall any choice greater than $\max_s A_s$ is valid).
Now the set of thinned events $U$ is piecewise-constant
Poisson with intensity $\Omega(\theta) + \Omega(\vartheta) - 
A_{S(t)}$. Following algorithm~\ref{alg:Unif_gibbs} or~\cite{RaoTeh13}, 
the {\em a priori} probability of the reconstructed set $W = U \cup T$, 
$P(W|\theta,\vartheta)$, is a homogeneous Poisson 
%the union of $U$ with the actual trajectory transition times $T$, 
process with rate $\Omega(\theta) + \Omega(\vartheta)$. Discard all 
MJP state information, so that the MCMC state space is $(W, \theta, \vartheta)$,
and propose swapping $\theta$ with $\vartheta$. 
Observe from
symmetry that the Poisson skeleton $W$ has the same probability both
before and after this proposal, so that unlike the previous scheme,
the ratio $P(W|\vartheta)/P(W|\theta)$ equals $1$.  This simplifies 
computation, and significantly improves mixing.
The acceptance probability 
%depends only on the probability of the observations
%under both set of parameters, %as we can use the forward-backward algorithm
%to calculate this. Our acceptance probability 
equals
$ 
  \min\left(1, \frac{P(X,\vartheta)q(\theta|\vartheta)}
   {P(X,\theta)q(\vartheta|\theta)}\right) = 
  \min\left(1, \frac{P(X|\vartheta)p(\vartheta)q(\theta|\vartheta)}
   {P(X|\theta)p(\theta)q(\vartheta|\theta)}\right).
   $
   The terms $P(X|\vartheta)$ and  $P(X|\theta)$ can be calculated by 
   running a forward pass of the forward-backward algorithm, and after
   accepting or rejecting the proposal, a new trajectory is sampled by
   completing the backward pass. Finally, the thinned events are
   discarded. See Algorithm~\ref{alg:MH_improved} and 
   figure~\ref{fig:MH_improved} in the appendix. % for details. 
\begin{algorithm}[H]
   \caption{Symmetrized MH for parameter inference for MJPs }
   \label{alg:MH_improved}
  \begin{tabular}{l l}
   \textbf{Input:  } & \text{The observations $X$,}
                      \text{the MJP path $S(t) = (S, T)$, parameters $\theta$} and $\pi_0$.\\ 
                     & \text{A  Metropolis-Hasting proposal $q(\cdot | \theta)$}.\\
   \textbf{Output:  }& \text{A new MJP trajectory $S'(t) = (S', T')$, 
                            new MJP parameters $\theta'$}.\\
   \hline
   \end{tabular}
   \begin{algorithmic}[1]
      \State Sample $\vartheta \sim q(\cdot| \theta)$, and 
      set %$\Omega = \max_i A_i(\theta) + \max_i A_i(\theta^*)$. 
	$\Omega \assign \Omega(\theta) + \Omega(\vartheta)$ for some function 
    $\Omega(\theta) \ge \max_s A_s(\theta)$.
      %In the case of uniformization, we
      %have a single $\Omega$ for all states, with $\Omega = \max_i A_i(\theta) + \max_i A_i(\theta^*)$.
      %, with $h(\theta) > max_s{|A_s(\theta)|}$, $h(\theta^*) > max_s{|A_s(\theta^*)|}$ using some deterministic function $h$.
    \State Sample thinned jumps $U\subset[0, t_{end}]$ from a Poisson process with 
    piecewise-constant rate $R(t) = (\Omega - A_{S(t)}(\theta))$. 
    Set $W = T \cup U$ and discard MJP states.
    \State The current MCMC state-space is $(W,\theta,\vartheta)$. Propose swapping
    $\theta$ and $\vartheta$. %the new state-space is 
     The acceptance probability is given by
        %\begin{align*}
     $
        \alpha 
        =  1 \wedge \frac{P(X| W,\vartheta,\theta)p(\vartheta)q(\theta|\vartheta)}
        {P(X| W,\theta, \vartheta)p(\theta) q(\vartheta|\theta)}.
        $
        %\end{align*}
    \State For both $\theta$ and $\vartheta$, make a forward pass through the 
    elements of $W$, sequentially updating the distribution over states at 
    $w \in W$ given observations up to $w$. At the end, we have calculated
    $P(X|W,\theta, \vartheta)$ and $P(X|W,\vartheta, \theta)$. Use these to accept or reject the
    proposed swapping of $\theta$ and $\vartheta$. Write the new state-space
    as $(W,\theta',\vartheta')$.
    \State For the new transition matrix $B(\theta',\vartheta')$, make a backward pass through 
    the elements of
    $W$, sequentially assigning a state to each element $w_i \in W$ given 
    $w_{i+1}$.
%    Sample a path $\tilde{V}$, from a discret-time Markov chain with $|W| + 1$ steps, using FFBS algorithm. The transition matrix of the Markov chain is $B = (I + \frac{A(\tilde{\theta})}{\Omega})$ while the initial distribution over states is $\pi_0$. The likelihood of state $s$ at step $i$ is 
%    $$ L_i(s) = P(Y_{[w_i, w_{i + 1})} | S(t) = s \; for\; t \in [w_i, w_{i + 1})) = \prod_{j: t_j \in [w_i, w_{i + 1})}p(y_{t_j} | S(t_j) = s).$$\\
%%(i.e. $V(i) \sim P(V |  \theta(i), W(i - 1), y).$) Then delete all the virtual jumps to get $S(i), T(i) .$\\
    \State Let $T'$ be the set of times in $W$ when the Markov chain changes state. Define $S'$ as the corresponding set of state values. Return $(S', T', \theta')$.
\end{algorithmic}
\end{algorithm}

