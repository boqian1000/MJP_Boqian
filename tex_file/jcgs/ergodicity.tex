\vspace{-.1in}
\section{Geometric ergodicity}
\vspace{-.05in}
We derive conditions under which our proposed algorithm
inherits mixing properties of an `ideal' sampler that operates without
computational constraints. The latter proposes a new parameter $\vartheta$
from distribution $q(\vartheta|\theta)$, and accepts with probability
$\alpha_I(\theta,\vartheta; X) = 1 \wedge \frac{P(X , \vartheta)q(\theta| \vartheta)}
{P(X , \theta)q(\vartheta|\theta)}$.  The resulting ideal (but intractable)
Markov chain has transition probability
$P_I(\theta'|\theta) = q(\theta'|\theta)\alpha_I(\theta,\theta';X) + \left[1-\int d\vartheta
q(\vartheta|\theta)\alpha_I(\theta,\vartheta;X)\right]\delta_\theta(\theta')$, the first
term corresponding to acceptance, and the second, rejection.

Our main result is Theorem~\ref{thm:geom_erg}, which shows that if the ideal MCMC
sampler is geometrically ergodic, then so is our tractable auxiliary
variable sampler (Algorithm~\ref{alg:MH_improved}). 
%Assumption~\ref{asmp:cond_num} is the only non-trivial
%one we have to make, requiring that the rate-matrix $A(\theta)$ not be
%arbitrarily ill-conditioned over all
%settings of the parameter $\theta$. This assumption is reasonable from
%both theoretical and practical viewpoints.
%%\boqian{Write $X$ as the observations, which is fixed.}
We first state all our definitions and assumptions, before diving into the 
proofs.
%For clarity, we focus on the case where the uniformization rate 
%is set as 
%$\Omega(\theta, \vartheta) \doteq \Omega(\theta) + \Omega(\vartheta)$,
%where $A$ is the rate matrix of the Markov jump process and $\Omega(\theta) \doteq k \max_s |A_{ss}(\theta)|$,
%for some $k > 1$.
\begin{assumption}
The uniformization-rate is set as $\Omega(\theta, \vartheta) = \Omega(\theta) + 
\Omega(\vartheta)$, where %$A$ is the rate matrix of the Markov jump process and 
$\Omega(\theta) = k_1 \max_s A_{s}(\theta) + k_0$, for some 
$k_1 > 1, k_0 > 0$.
\label{asmp:unif_rate}
\end{assumption}
%If $\inf_\theta \Mx{\theta} > 0$ (as is often the case),
% we can just set $\Omega(\theta) = \Mx{\theta} + \mx{\theta}$ so that
% $m_2 = \frac{1}{1+\mu}$.}
Although it is possible to specify broader conditions under which our 
result holds, for clarity we focus on this case. 
We can drop the $k_0$ if $\inf_\theta \max_s A_s(\theta) > 0$
%This is also how we setup our sampler in our experiments.

\begin{assumption}
 % $\Omega(\theta)$ is not bounded, and 
 There exists a positive constant $\theta_0$ such 
  that for any $\theta_x, \theta_y$ satisfying
  $\| \theta_x \| \ge \| \theta_y \| > \theta_0$, we have $\Omega(\theta_x) \ge \Omega(\theta_y)$.  
%  \begin{align*}
%   \lim_{\|\theta\| \rightarrow +\infty} \Omega(\theta) = + \infty.
%  \end{align*}
  \label{asmp:mono_tail}
\end{assumption}
\noindent %We will see that proving our result when $\Omega(\theta)$ is bounded is 
%easy. %This assumption usually holds under suitable reparametrization,
%and is not critical: we impose 
%When this is not so, 
This assumption avoids book-keeping by making 
$\Omega(\theta)$ increase monotonically with $\theta$.

\begin{definition}
Let $\pi_\theta$ be the stationary distribution of the MJP with rate-matrix 
$A(\theta)$, and define $D_\theta = \text{diag}(\pi_\theta)$. Define 
$\tilde{A}(\theta) = D_\theta^{-1}A(\theta)D_\theta$, and the 
{\em reversibilization} of $A(\theta)$ as $R_A(\theta) = 
(A(\theta)+\tilde{A}(\theta))/2$. 
\label{def:mjp_symm}
\end{definition}
This definition is from~\cite{fill1991}, who show that the matrix 
$R_A(\theta)$ is reversible with real eigenvalues, the smallest being $0$. 
The larger its second smallest eigenvalue, the faster the MJP converges to its 
stationary distribution $\pi_\theta$.
Note that if the original MJP is reversible, then $R_A(\theta) = A(\theta)$.

%{
%\begin{definition}
%Let $P_{st}(X | \theta)$ be the stationary distribution of the observations 
%$X$, given the parameter $\theta$. Write $V_X$ for the states of the markov chain corresponding to the observations.
%\begin{align*}
%P_{st}(X | \theta) \doteq \sum_{V_X} P(X | V_X, \theta) \pi_\theta(V_X).
%\end{align*}
%\label{def:stationary_pst}
%\end{definition}
%}
%\begin{assumption}
%For the stationary distribution $\pi_\theta$, there exists $\pi_\lb > 0$, such that $\inf_{\theta, s} \pi_{\theta}(s) > \pi_\lb$.
%\label{asmp:stationary_dist_lower_bound}
%\end{assumption}
\begin{assumption}
%  The rate-matrix $A(\theta)$ is ergodic for all $\theta$ and satisfies
 % $\inf_\theta \frac{\mx{\theta}}{\Mx{\theta}} = \mu > 0$.
%  $\forall \theta$ satisfying
%  $|\theta| > |\hat{\theta}|$, we have $\Mx{\theta} \asymp
%  \mx{\theta}$.
  Write $\lambda^{R_A}_2(\theta)$ for the second smallest eigenvalue of
    $R_A(\theta)$. There exist $\mu > 0, \theta_1 > 0$
    such that for all $\theta$ satisfying $ \| \theta \|> \theta_1$, 
    we have $ \lambda^{R_A}_2(\theta) \geq \mu \max_s A_s(\theta)$
    (or equivalently from Assumption~\ref{asmp:unif_rate}, 
    $ \lambda^{R_A}_2(\theta) \geq \mu \Omega(\theta)$),
    and $\min_s \pi_\theta(s) > 0$. 
  \label{asmp:cond_num}
\end{assumption} 
\noindent %This is the strongest assumption needed to prove our result;
%essentially requiring the rate matrix to be well-conditioned for all 
%large $\theta$. 
%Assumption~\ref{asmp:mono_tail} allows $\max |A_{ss}(\theta)|$ 
%(and therefore $\Omega(\theta)$) to grow with $\|\theta\|$. Here, we 
The assumption on $\lambda^{R_A}_2$ is the strongest we need, requiring that 
$\lambda^{R_A}_2(\theta)$ (which sets the MJP mixing rate) grows 
at least as fast as $\max A_s(\theta)$. 
%This controls the relative stability of different MJP states, and 
This is satisfied when, for example, all elements of $A(\theta)$ grow 
with $\theta$ at similar rates, controlling the relative stability of 
the least and most stable states.
%rules 
%out any state $s'$ such that 
%$\frac{|A_{s's'}(\theta)}{\max_s A_{ss}(\theta)}| \rightarrow 0$ as $\theta$ increases.
While not trivial, this is a reasonable assumption: the MCMC chain over MJPs 
will mix if we can control the mixing of the MJP itself.
%it holds for the examples we considered. %One open question is whether this assumption is necessary.
%We suspect that assumption~\ref{asmp:ideal_geom}, which requires 
%the ideal sampler to be geometrically ergodic, implies this condition,
%however we do not try to prove it here.
%is not sufficient to ensure that $ $ is a 
%stronger requirement, ensures
%parameters of interest\{?: the most
%unstable state cannot have a leaving rate that is larger than that of
%the most stable state by an arbitrary factor.} This condition controls the
%mixing behavior of the MJP (and the embedded Markov chain) for any
%starting state.
%We note that we only need this
%condition to hold in the tails of $\theta$, i.e.\ we only need
%  $\inf_\theta \frac{\mx{\theta}}{\Mx{\theta}} > 0$ for all $\theta$
%  satisfying $|\theta| > h$ for some $h$. For clarity, we restrict
%  ourselves to $h=0$, extending our proof to the general case is
%  straightforward.
%for any interval $\Delta$, we can always find a $\theta_0$ such that the
%MJP has mixed at the end of the interval. The earlier assumption ensures
%this holds for all $\theta \ge \theta_0$.
To better understand this, recall $B(\theta, \theta') = I+\frac{A(\theta)}{\Omega(\theta, \theta')}$
is the transition matrix of the embedded Markov chain, and note 
it has the same stationary distribution $\pi_\theta$ as $A(\theta)$.
Define the reversibilization $R_B(\theta,\theta')$ from $B(\theta,\theta')$ 
just as we did $R_A(\theta)$ from $A(\theta)$. 
\begin{lemma}
  Let $\|\theta\| > \max(\theta_0, \theta_1)$ and $\theta'$ satisfy 
$\frac{1}{K_0} \le \frac{\Omega(\theta')}{\Omega(\theta)} \le K_0 $ 
with $K_0$ %> 1$, which 
satisfying $(1 + \frac{1}{K_0})k_1 \ge 2$. 
For all such $(\theta,\theta')$, the Markov chain with transition matrix 
$B(\theta,\theta')$ converges geometrically to stationarity at a rate 
uniformly bounded away from $0$.
%The second largest eigenvalue $\lambda^2_B(\theta,\theta')$ of $R_B$ and  
%second smallest eigenvalue $\lambda^2_B(\theta,\theta')$ of $R_B$ satisfy  
%$\lambda^2_B(\theta,\theta') = 1 - \frac{\lambda^2_A(\theta)}{\Omega(\theta, \theta')}$.
  \label{lem:eig_lemma}
\end{lemma}
\begin{proof}
%it is easy to see that both $A(\theta)$ and $B(\theta,\theta')$ have
%the same stationary distribution $\pi_\theta(s)$. For the matrix $B(\theta,\theta')$, 
A little algebra gives $R_B(\theta,\theta') = I + R_A(\theta)/\Omega(\theta,\theta')$. It 
follows that both $R_A$ and $R_B$ share the same eigenvectors, with 
eigenvalues satisfying 
%and that 
%the eigenvalues of $A(\theta)$ and $B(\theta,\theta')$ satisfy
$\lambda_{R_B}(\theta, \theta') = 1 - \frac{\lambda_{R_A}(\theta)}{\Omega(\theta,
\theta')}$. In particular, the second largest eigenvalue 
$\lambda_2^{R_B}(\theta,\theta')$ of $R_B$ and  
second smallest eigenvalue $\lambda^{R_A}_2(\theta,\theta')$ of $R_A$ satisfy  
$\lambda^{R_B}_2(\theta,\theta') = 1 - \frac{\lambda^{R_A}_2(\theta)}{\Omega(\theta, \theta')}$.
Then, from assumptions~\ref{asmp:unif_rate} and~\ref{asmp:cond_num}, and 
the lemma's assumptions, 
$1 - \lambda^{R_B}_2(\theta,\theta') = \frac{\lambda^{R_A}_2(\theta)}{\Omega(\theta, \theta')} 
\ge \frac{\lambda^{R_A}_2(\theta)}{(K_0+1)\Omega(\theta)} 
\ge \frac{\mu}{K_0+1} $. 
Also, 
\begin{align*}
\Omega(\theta, \theta') &= \Omega(\theta) + \Omega(\theta') \ge (1 + \frac{1}{K_0})\Omega(\theta)
 > (1 + \frac{1}{K_0})k_1\max_s A_{s}(\theta) \ge 2\max_s A_{s}(\theta). %\ge -2 A(\theta)_{s,s}.
\end{align*}
So for any state $s$, the diagonal element $B_s(\theta, \theta') = 1 - 
\frac{A_s(\theta)}{\Omega(\theta, \theta')}> \frac{1}{2}$.
From~\cite{fill1991}, this diagonal property and the bound 
on $1-\lambda_2^{R_B}(\theta,\theta')$ give the result.
\qed
\end{proof}

Our overall proof strategy is to show that for {$\| \theta \|$} and $W$ large 
enough, the conditions of Lemma~\ref{lem:eig_lemma} hold with 
high probability. We show that Lemma~\ref{lem:eig_lemma} then allows the 
distribution over latent states for the continuous-time MJP and its 
discrete-time counterpart embedded in $W$ to be brought arbitrarily 
close to $\pi_\theta$ (and thus to each other), allowing our sampler 
to inherit mixing properties of the ideal sampler. For the remaining 
$\theta$ and $W$, we will exploit their boundedness to establish a 
`small-set condition' where the MCMC algorithm forgets its state with 
some probability. These two conditions will be sufficient for 
geometric ergodicity. The next assumption states corresponding conditions 
for the ideal sampler.
% This combined with the
%likelihood $p(X|s, \theta)$ being bounded  gives the result.
%\begin{definition}
%Let $\pi_\theta$ be the stationary distribution of the MJP with rate-matrix 
%$A(\theta)$, and define $D_\theta = \text{diag}(\pi_\theta)$. Define 
%$\tilde{A}(\theta) = D_\theta^{-1}A(\theta)D_\theta$, and the 
%{\em reversibilization} of $A(\theta)$ as $R_A(\theta) = 
%(A(\theta)+\tilde{A}(\theta))/2$. 
%\label{def:mjp_symm}
%\end{definition}

\begin{assumption}
For the ideal sampler with transition probability $p_I(\theta'|\theta)$: \\
i) the set $\SM_M=\{\theta:\Omega(\theta)\le M \}$ is a $1$-small set,
i.e.\ for each $M$, there exists a probability measure $\phi$ and a constant
$\kappa_1 > 0$ s.t.\ % $q(\ptheta | \theta) \alpha_I(\theta, \ptheta)
%$p_I(\theta'|\theta) \geq \kappa_1 \phi(\theta')$ for $\theta \in \SM_M$, \\
$\alpha_I(\theta, \theta'; X) q(\theta' | \theta) \ge \kappa_1 \phi(\theta')$
 for $\theta \in \SM_M$, \\
ii) for $M$ large enough, $\exists \rho < 1$ s.\ t.\
$\int \Omega(\ptheta) p_I(\ptheta|\theta) d\ptheta
\leq (1-\rho) \Omega(\theta)+L_I$, $\forall \theta \not\in \SM_M$.
  \label{asmp:ideal_geom}
\end{assumption}
\noindent %Here we specify the properties of the ideal sampler that we
%want our proposed sampler to inherit. 
These two conditions are standard
small-set and drift conditions necessary for the ideal sampler to satisfy
geometric ergodicity. The first implies that for $\theta$ in
$\SM_M$, the ideal sampler forgets its current
location with probability $\kappa_1$. The second condition ensures that
for $\theta$ outside this set, the ideal sampler drifts towards
$\SM_M$. These two conditions together imply geometric
mixing with rate equal or faster than $\kappa_1$~\cite{meyn2009}.
Observe that we have used $\Omega(\theta)$ as the so-called 
Lyapunov-Foster function to define the drift condition for the ideal 
sampler. %and whose sub-level sets form the small sets. 
This is the most natural choice,
though our proof can be tailored to different choices. Similarly, we
could easily allow $\SM_M$ to be an $n$-small set for any $n\ge 1$ (so
the ideal sampler needs $n$ steps before it can forget its current
value in $\SM_M$); we restrict ourselves to the $1$-small case for
clarity.


%\begin{assumption}
%  $\inf_{\ptheta} \left(\Omega(\theta,\ptheta) - \Mx{\theta}\right)
%  \ge m\Mx{\theta} \forall \theta$.
 % There exist constants $m_1$ and $m_2$ such that
  %$\inf_\theta \Omega(\theta) = m_1 > 0$ and $\sup_{s,\theta}
  %\frac{A_s(\theta)}{\Omega(\theta)} = m_2 < 1$.
  %\label{asmp:low_bnd}
%\end{assumption}
% \noindent
% \{
% Assumption~\ref{asmp:cond_num} allows us to easily ensure this is
% satisfied by setting $\Omega(\theta) = \Mx{\theta} + \mx{\theta}+m_1$.
% Another option is to set $\Omega(\theta) = \kappa\Mx{\theta} + m_1$ some
% $\kappa > 1$.  If $\inf_\theta \Mx{\theta} > 0$ (as is often the case),
% we can just set $\Omega(\theta) = \Mx{\theta} + \mx{\theta}$ so that
% $m_2 = \frac{1}{1+\mu}$.}


\begin{assumption}
$\exists$ $ \ub > \lb > 0$ s.t.
$\prod P(X | s_o, \theta) \in [\lb, \ub]$ for any state $s_o$ and $\theta$.%\vinayak{Is this for all $s_o$?}
  \label{asmp:obs_bnd}
\end{assumption}
\noindent This assumption follows~\cite{miasojedow2017}, and holds if
%the observation process involves no hard constraints over the latent state, and
$\theta$ does not include parameters of the observation process (or if so,
the likelihood is finite and nonzero for all settings of $\theta$). We can relax this assumption,
though this will introduce technicalities unrelated to our focus, which 
is on complications in parameter inference arising from the continuous-time
dynamics, rather than the observation process. 
%Extensions to the more
%general case should be clear from our proof.

\begin{assumption}
Given the proposal density $q(\ptheta | \theta)$, $\exists \eta_0 > 0, \theta_2 > 0$ 
such that for $\theta$ satisfying $\| \theta \|  > \theta_2$, 
$ \int_\Theta \Omega(\ptheta)^2 q(\ptheta | \theta)d\ptheta \leq \eta_0 \Omega(\theta)^2.$
\label{asmp:integ_bound}
\end{assumption}
\noindent This mild requirement can be satisfied by choosing a proposal 
distribution $q$ that does not attempt to explore large $\theta$'s too 
aggressively.
\begin{corollary}
Given the proposal density $q(\ptheta | \theta)$, $\exists \eta_1 > 0, \theta_2 > 0$ such that for $\theta$ 
satisfying $\| \theta \|  > \theta_2$, 
$ \int_\Theta \Omega(\ptheta) q(\ptheta | \theta)d\ptheta \leq \eta_1 \Omega(\theta).$
\label{corol:integ_bound}
\end{corollary}
\begin{proof}
From assumption \ref{asmp:integ_bound},  we have $ \int_\Theta \Omega(\ptheta)^2 q(\ptheta | \theta)d\ptheta \leq \eta_0 \Omega(\theta)^2$ for $\theta$ satisfying $\| \theta \|  > \theta_2$.
For such $\theta$, by the Cauchy-Schwarz inequality, we have
\begin{align*}
\left[ \int_\Theta \Omega(\ptheta) q(\ptheta | \theta) d\ptheta \right]^2 &\le \int_\Theta \Omega(\ptheta)^2 q(\ptheta | \theta) d\ptheta \cdot \int_\Theta q(\ptheta | \theta) d\ptheta \le \eta_0 \Omega(\theta)^2.
\end{align*}
So for $\theta$ satisfying $\| \theta \|  > \theta_2$, we have $\int_\Theta \Omega(\ptheta) q(\ptheta | \theta) d\ptheta \le \sqrt{\eta_0} \Omega(\theta).$
\qed
\end{proof}
%\begin{assumption}
%For any positive $\epsilon > 0$ and $K_0 > 1$, there exists $\theta_2$, such that for $\theta > \theta_2$, $P(\mid \frac{q(\theta | \theta')p(\theta')}{q(\theta | \theta')p(\theta)}\mid \leq K_0 | \theta) \geq 1 - \epsilon$, where $\theta' \sim q(\cdot | \theta)$.
%\end{assumption}

%\begin{assumption}
%For any positive $\epsilon$, there exists $h > 0$ and $\theta_2 > 0$, such that $P(\theta' \in B_{\theta, h} | \theta) > 1 - \epsilon$ for $\theta \in B_{0, \theta_2}^c$ .
%\end{assumption}

%\begin{assumption}
%For any positive $\epsilon > 0$ and $K_1 > 1$, there exists $\theta_3$, such that for $\theta > \theta_3$, $P(\frac{1}{K_1}%\frac{\Omega(\theta')}{\Omega(\theta)} \leq K_1 | \theta) \geq 1 - \epsilon$, where $\theta' \sim q(\cdot | \theta)$.
%\end{assumption}
%\begin{assumption}
%There exists $M_1 > 0$ and $\theta_3 > 0$ such that $\frac{-p'(\theta)}{p(\theta)} \le M_1$ for $\theta \in B_{0, \theta_3}^c$.
%\end{assumption}

We need two further assumptions on the proposal distribution $q(\theta'|\theta)$.
\begin{assumption}
For any $\epsilon>0$,  there exist finite $M_\epsilon$, $\theta_{3,\epsilon}$
such that for $\theta$ satisfying $\| \theta \|  > \theta_{3,\epsilon}$,
the condition $q(\{\theta': \frac{p(\theta')q(\theta|\theta')}{p(\theta)q(\theta'|\theta)} \le M_\epsilon\}|\theta) 
> 1 - \epsilon$ holds.
  \label{asmp:prior}
\end{assumption}
This holds, when e.g.\ $p(\theta)$ is a gamma distribution,
and $q(\theta'|\theta)$ is Gaussian.
%\begin{proof}
%From the assumptoin, there exists $\theta_5 > 0$ such that $p(\theta)$ is decreasing with respect to $\Vert \theta \Vert$ for $\theta \in B_{0, \theta_5}^c$. For $\theta > \max(\theta_2, \theta_5)$, $\theta' \in B_{\theta, h}$ with probability greater than $1 - \epsilon$.
%\begin{align*}
%\log \frac{p(\theta')}{p(\theta)} &\le \log\frac{p(\theta - h)}{p(\theta)}
%= \log p(\theta - h) - \log p(\theta)\\
%& \le -h \frac{p'(\chi)}{p(\chi)} \le M_1h
%\end{align*}
%,where $\chi \in B_{\theta, h}$.
%So $p(\theta')/p(\theta) \le M_1 h$ with probability greater than $1 - \epsilon$.
%\end{proof}

%\begin{assumption}
%There exists $M_2 > 0$ and $\theta_4 > 0$ such that $\frac{\Omega'(\theta)}{\Omega(\theta)} \le M_2$ for $\theta \in B_{0, \theta_4}^c$.
%\end{assumption}

%\begin{assumption}
%There exists $\theta_5 > 0$ such that $p(\theta)$ is decreasing with respect to $\Vert \theta \Vert$ for $\theta \in B_{0, \theta_5}^c$.
%\end{assumption}

\begin{assumption}
For any $\epsilon > 0$ and  $K > 1$, there exists $\theta_{4,\epsilon}^K$ such that 
for $\theta$ satisfying $\| \theta \|  > \theta_{4,\epsilon}^K$, the 
condition
$q(\{\theta':\frac{\Omega(\theta')}{\Omega(\theta)} \in 
  \left[\frac{1}{K}, K\right]\} | \theta) > 1 - \epsilon$ holds.
  \label{asmp:omega}
\end{assumption}
This holds when e.g.\ $q(\theta'|\theta)$ is a centered on $\theta$ and has 
finite variance.

%\begin{proof}
%From the assumptoin, there exists $\theta_5 > 0$ such that $p(\theta)$ is decreasing with respect to $\Vert \theta \Vert$ for $\theta \in B_{0, \theta_5}^c$. For $\theta > \max(\theta_2, \theta_5)$, $\theta' \in B_{\theta, h}$ with probability greater than $1 - \epsilon$.
%\begin{align*}
%\log(\frac{\Omega(\theta')}{\Omega(\theta)}) & \le \log(\frac{\Omega(\theta + h)}{\Omega(\theta)})\\
%& = h \frac{\Omega(\chi')}{\Omega(\chi')} \le h M_2
%\end{align*}
%Similarily, we have
%\begin{align*}
%\log(\frac{\Omega(\theta')}{\Omega(\theta)}) & \ge \log(\frac{\Omega(\theta - h)}{\Omega(\theta)})\\
%& = -h \frac{\Omega(\chi'')}{\Omega(\chi'')} \ge -h M_2
%\end{align*}
%So $\Omega(\theta') / \Omega(\theta) \in [\exp(-hM_2), \exp(hM_2)]$ with probability greater than $1 - \epsilon$.
%\end{proof}

%\begin{assumption}
%For any positive $\epsilon > 0$ and $0 < K_2 < 1$, there exists $\theta_4$, such that for $\theta > \theta_4$, $P(\theta' > \theta K_2 | \theta) %\geq 1 - \epsilon$, where $\theta' \sim q(\cdot | \theta)$.
%\end{assumption}

%\begin{assumption}
%For the above $C \subseteq \Theta$, $\exists$ $\bar{\Omega} > 0$ s.t.
%$\Omega(\theta)  \leq \bar{\Omega}$ for $\forall \theta \in C$.
%\end{assumption}

%\begin{definition}
%Given $K > 1$, $\epsilon > 0$, define $\theta_\epsilon^K$ as $\max\{ \theta_0, \theta_1, \theta_2, \theta_3, \theta_4^\epsilon, \theta_{5, K}^\epsilon\}$.
%\label{def:constant}
%\end{definition}


\begin{theorem}
Under the above assumptions, our auxiliary variable MCMC sampler is
geometrically ergodic.  \label{thm:geom_erg}
\end{theorem}
\begin{proof}
\noindent This theorem follows from two lemmas we will prove.
Lemma~\ref{lem:small_set} shows there exist small sets 
$\{(W,\theta,\vartheta): \lambda_1|W| + \Omega(\theta) < M \}$ for 
$\lambda_1, M > 0$, within which our sampler forgets its current state with 
some positive probability. Lemma~\ref{lem:drift}
shows that for appropriate $(\lambda_1,M)$, our sampler drifts towards this set whenever
outside. Together, these two results imply geometric
ergodicity~\citep[Theorems 15.0.1 and Lemma 15.2.8]{meyn2009}.
If $\sup_\theta \Omega(\theta) < \infty$, we just need 
the small set $\{(W,\theta,\vartheta: |W| < M \}$ for some $M$.
\qed
\end{proof}
\vinayak{Add $(T,U)$}
For our proof, we modify the start and end of an MCMC iteration from 
Algorithm~\ref{alg:MH_improved}. We regard our sampler as operating on 
the space $(\theta,\vartheta,W)$,  with $\theta$ the current MJP 
parameter, $\vartheta$ the auxiliary variable, and $W$ the Poisson grid. 
An MCMC transition step updates this to $(\theta',\vartheta',W')$ by 
1) sampling $(S,T)$ given $(\theta,\vartheta)$ with a backward pass, 
2) discarding $\vartheta$, 3) sampling $\ptheta$ from $q(\ptheta|\theta)$, 
4) sampling a new $W'$ given $(\theta,\ptheta)$, 5) proposing to swap 
$(\theta,\ptheta)$ and then 6) accepting or rejecting with a forward pass. 
On acceptance, $\theta' = \ptheta$ and
$\vartheta' = \theta$, whereas on rejection, $\theta'=\theta$ and 
$\vartheta'=\ptheta$. We write $(\theta'',\vartheta'',W'')$ for the 
MCMC state after two iterations.
% \{?
% We establish the drift condition via the Lyapunov-Foster function
% $\cV(W,\theta,\vartheta) = \left. \lambda_1|W| + \lambda_2 \Omega(\theta) +
% \Omega(\vartheta) +L \right. := \cV_W(W) + \cV_{\theta}(\theta) +
% \cV_{\theta^*}(\theta^*)$, for settings of $\lambda_1,
%   \lambda_2$ and $L$ we will define later.}
% We point out that if $\Mx{\theta}$ is bounded, then we can replace
% the two $\Omega$ terms by their supremum and absorb them into the
% constant $L$, so that the Lyapunov function only involves $W$.
% The small set will be a level set of this function, i.e.\
% $B_\alpha := \{(\theta,\theta^*,W): \cV(\theta,\vartheta,W) \le \alpha\}$.
% We show that this is a 2-small set (i.e.\ the distribution over states
% after 2-steps of our sampler satisfies $P^2(\cdot|\theta,\vartheta,W)
% \ge \epsilon \ptheta(\cdot)$ for all $(\theta,\vartheta,W) \in \SM_\alpha$.
% Unlike the ideal sampler, we need a 2-step transition because of the
% exchange move we make,
% where on acceptance $\vartheta'=\theta$, while on rejection, $\theta'=\theta$.

We first bound self-transition probabilities
of the embedded Markov chain from $0$: 
%Proofs, if not included, can be found in the appendix.
%\begin{proposition}
%The a priori probability the embedded Markov chain makes a self-transition,
%$P(V_{i + 1}=s|V_i=s,\theta,\theta^*)$ is uniformly bounded away
%from $0$ for all $s,\theta,\theta^*$.
%\end{proposition}
%\begin{proof}
%    & \ge 1-\frac{A_{s_0}(\theta)}{\Mx{\theta}} \ge 1 - \mu
% \intertext{For $\Omega(\theta) \ge k\Omega(\theta^*)$}
% P(V_{i + 1}=s|V_i=s,\theta,\theta^*) &=
%   1 - \frac{A_{s_0}(\theta)}{\Omega(\theta) + \Omega(\theta^*)} \ge
%   1 - \frac{\Omega(\theta)}{m + \Omega(\theta^*)}
% \intertext{For $\Omega(\theta) < k\Omega(\theta^*)$}
% P(V_{i + 1}=s|V_i=s,\theta,\theta^*) &\ge
%   1 - \frac{A_{s_0}(\theta)}{\Omega(\theta) + \Omega(\theta)/k}
%   \ge 1 - \frac{1}{1 + 1/k}
%\end{align*}
%\end{proof}
\begin{proposition}
The posterior probability that the embedded Markov chain makes a
self-transition,
$P(V_i = V_{i + 1} | W, X, \theta, \vartheta) \ge \delta_1 > 0$,
for %$i = 0, 1, 2, ..., |W|$ and
any $\theta,\vartheta, W$.
\label{prop:self_tr}
\end{proposition}
The proof (in the appendix) exploits the bounded likelihood from 
assumption~\ref{asmp:obs_bnd}. A simple by-product of the proof is
the following corollary:
\begin{corollary}
%The posterior probability that the embedded Markov chain makes a
%self-transition,
$P(V_{i + 1} = s|V_s=s, W, X, \theta, \vartheta) \ge \delta_1 > 0$,
for %$i = 0, 1, 2, ..., |W|$ and
any $\theta,\vartheta, W,s$.
\label{corol:self_tr}
\end{corollary}

\begin{lemma}
  For all $M,h > 0$, the set $B_{h,M} =
\left\lbrace (W, \theta, \vartheta) : |W| \leq h, \theta \in \SM_M
\right\rbrace$ is a 2-small set under our proposed sampler. 
Thus, for all $(W,\theta,\vartheta)$ {in $B_{h, M}$},
the two-step transition probability satisfies 
$P(W'', \theta'',\vartheta'' | W, \theta, \vartheta) \ge \rho_1 
\phi_1(W{''}, \theta'',\vartheta'') $ for a constant $\rho_1$ and a 
probability measure $\phi_1$ independent of the initial state.
\label{lem:small_set}
\end{lemma}
\begin{proof} Recall the definition of $B_M$, and of an $n$-small set from 
  Assumption~\ref{asmp:ideal_geom}. The $1$-step transition probability of our MCMC algorithm
  consists of two terms, corresponding to the proposed parameter being
  accepted and rejected. Discarding the latter, we have %the bound
\begin{align*}
  P(W',\theta',\vartheta'|W,\theta,\vartheta,X)&\geq
  \delta_\theta(\vartheta') q(\theta'|\theta)
 \sum_{S,T}  P(S,T | W, \theta, \vartheta, X) 
             P(W'| S, T, \theta', \theta)
             \alpha(\theta', \theta, W';X)
\end{align*}
%Suppose $(W, \theta, \vartheta) \in B_{h, M}$.
Here we use the fact that given $(S,T)$,
$P(W'|S,T,\theta',\theta,X)$ is independent of  $X$.
We further bound the summation over $(S,T)$ by considering only the terms
with $S$ a constant. When this constant is  state $s^*$, we write this as 
$(S=[s^*], T= \emptyset)$. This corresponds to $|W|$ self-transitions 
after the starting state $S_0=s^*$, so that
\vspace{-.1in}
\begin{align*}
  P(S=[s^*], T = \emptyset | W, \theta, &\vartheta, X) =
P(S_0=s^*|X,W, \theta, \vartheta)\prod_{i = 0}^{|W| - 1} 
P(V_{i + 1} = s^* | V_i = s^*,X,W,\theta,\vartheta) \\ 
%&\prod P(X_{[w_i, w_{i + 1})} | v_i = s_0, \theta)\\
& \geq P(S_0=s^*|X,W, \theta, \vartheta)\delta_1^{|W|} %\numberthis %\eta_1
\end{align*}
Here $\delta_1$ is the lower bound from Corollary~\ref{corol:self_tr}.
With $S(t)$ fixed at $s^*$, $W'$ is a Poisson process with rate
%$r(\theta', \theta, s_0) = 
$\Omega(\theta') + \Omega(\theta) - A_{s^*}(\theta)$.
%$> \epsilon_1 > 0$.
From the %fact that $\Omega(\theta',\theta) = \Omega(\theta') + \Omega(\theta)$,
%we apply the 
Poisson superposition theorem,
%$\PP(r(\theta, \ptheta, s_0)) = \PP(\Omega(\ptheta)) \cup \PP(r(\theta, \ptheta, s_0) - \Omega(\ptheta))$.
\begin{align*}
  P(W' |  S=&[s^*], T = \emptyset, \theta', \theta)  \geq P(W' \from
\PP(\Omega(\theta')))
P(\emptyset \from \PP(\Omega(\theta)-A_{s^*}(\theta) ))\\
  & \geq P(W' \from \PP(\Omega(\theta'))) P(\emptyset \from \PP(\Omega(\theta) ))\\
%& = P(W' \from \PP(\Omega(\theta'))) \exp(-\Omega(\theta)t_{end})\\
  & \geq P(W' \from \PP(\Omega(\theta'))) \exp(-M t_{end}) 
\quad \text{(since for $\theta \in B_M$, $\Omega(\theta) \le M$)}.
%\numberthis %\quad (\text{since }\theta\in B_M)
\end{align*}
\vspace{-.05in}
Thus we have
\begin{align*}
  \sum_{S,T} P(S,T,W' | W, \theta, \vartheta, X) & 
  \geq \sum_{{s^*}} P(S{=[s^*]}, T = \emptyset | W, \theta, \vartheta, X)
  P(W' | S{=[s^*]}, T=\emptyset,\theta', \theta)\\
               &\geq \delta_1^{|W|} \exp(-Mt_{end})
P(W' \from \PP(\Omega(\theta'))) \numberthis
\label{eq:marg}
\end{align*}
Finally we relate the acceptance rate to that of the ideal sampler:
\begin{align*}
\alpha(\theta', \theta, W'; X) &
%= 1 \wedge \frac{P(X | W', \theta', \theta)
%q(\theta|\theta')p(\theta')}{P(X | W', \theta, \theta')q(\theta'|\theta)p(\theta)}
= 1 \wedge \frac{P(X|W', \theta', \theta) / P(X|\theta')}{P(X|W', \theta,
\theta') / P(X|\theta)} \cdot \frac{P(X | \theta')
q(\theta|\theta')p(\theta')}{P(X | \theta)q(\theta'|\theta)p(\theta)}\\
& \geq 1 \wedge \frac{\lb^2}{\ub^2} \cdot 	\frac{P(X | \theta')
q(\theta|\theta')p(\theta')}{P(X | \theta)q(\theta'|\theta)p(\theta)}
 \geq \alpha_I(\theta, \theta';X)\frac{\lb^2}{\ub^2}.
\numberthis
\label{eq:acc}
\end{align*}
Since by assumption $|W| \le h$, %from equations~\eqref{eq:marg} and~\eqref{eq:acc}, 
and $q(\theta'|\theta)\alpha_I(\theta,\theta';X) \ge \kappa_1 \phi(\theta')$ 
(from assumption \ref{asmp:ideal_geom}),
\begin{align*}
P(W', \theta',\vartheta' | W, \theta, \vartheta)  
%&\ge \frac{\lb^2 }{\ub^2}\delta_1^{h}
%\exp(-M t_{end})\delta_\theta(\vartheta')P(W' \from \PP(\Omega(\theta')) 
% \alpha_I(\theta', \theta;X)q(\theta'|\theta)\\
  & \geq \frac{\lb^2 }{\ub^2}\delta_1^{h}
\exp(-M t_{end})\delta_\theta(\vartheta')\kappa_1P(W' \from \PP(\Omega(\theta'))\phi(\theta') \\
  & \assign \rho_1 \delta_\theta(\vartheta')P(W' \from \PP(\Omega(\theta'))\phi(\theta')
\end{align*}
{Write $F_{Poiss(a)}$ for the CDF of a rate-$a$ Poisson.
The two-step transition satisfies}
\begin{align*}
  P(W'', \theta''&,\vartheta'' | W, \theta, \vartheta)  
  %\int P(W'', \theta'',\vartheta'' | W', \theta', \vartheta')
  %    P(W', \theta',\vartheta' | W, \theta, \vartheta)
  %    dW' d\theta' d\vartheta' \\
       \ge \int_{\SM_{h,M}} P(W'', \theta'',\vartheta'' | W', \theta', \vartheta')
       P(W', \theta',\vartheta' | W, \theta, \vartheta)
       dW' d\theta' d\vartheta' \\
       &\ge \int_{\SM_{h,M}}  \rho_1 \delta_{\theta'}(\vartheta'')P(W'' \from \PP(\Omega(\theta''))\phi(\theta'') \\
         &\qquad \qquad \rho_1 \delta_\theta(\vartheta')P(W' \from \PP(\Omega(\theta'))\phi(\theta')
       dW' d\theta' d\vartheta' \\
       &\ge \rho_1^2 \phi(\theta'')P(W'' \from \PP(\Omega(\theta''))
       \int_{\SM_{h,M}} \!\!\!\! \delta_{\theta'}(\vartheta'')
       F_{Poiss(\Omega(\theta'))}(h)\phi(\theta')
       d\theta'  \\
       & \ge \rho_1^2 P(W'' \from
       \PP(\Omega(\theta''))\phi(\theta'')\phi(\vartheta'')F_{Poiss(\Omega(\vartheta''))}(h)\delta_{\SM_{h,M}}(\vartheta'') \\
       & \ge \rho_1^2 P(W'' \from
       \PP(\Omega(\theta''))\phi(\theta'')
       \phi(\vartheta'')\delta_{\SM_{h,M}}(\vartheta'')\exp(-\Omega(\vartheta''))  \numberthis
       \label{eq:density_lowbound}
\end{align*}
The last line uses $F_{Poiss(a)}(h) \ge F_{Poiss(a)}(0) = \exp(-a)\ \forall 
a$, 
%Then
%\begin{align*}
%\int \phi(\vartheta'')\delta_{\SM_{h,M}}(\vartheta'')\exp(-\Omega(\vartheta'')) d\vartheta'' \le \int_{\SM_{h,M}}\phi(\vartheta'')d\vartheta'' < 1
%\end{align*}
%So, $\phi(\vartheta'')\delta_{\SM_{h,M}}(\vartheta'')\exp(-\Omega(\vartheta''))$ is proportional to a probability density.
and gives our result, 
with $\phi_1(W'',\theta'',\vartheta'') \propto P(W'' \from
  \PP(\Omega(\theta''))\phi(\theta'') \phi(\vartheta'')
  \delta_{\SM_{h,M}}(\vartheta'')\exp(-\Omega(\vartheta''))  $.
\qed
\end{proof}
\noindent We have established the small set condition: for any point 
inside $B_{h,M}$ our sampler forgets its state with nonzero 
probability. We next establish a drift condition, showing that outside 
this small set, the algorithm drifts back towards it 
(Lemma~\ref{lem:drift}).
%Observe that if $\Omega(\theta)$ is bounded uniformly, then
%the drift condition need not include $\theta$. To prove the drift
%condition for the general case when $\Omega(\theta)$ is unbounded,
We first establish a result needed when $\Mx{\theta}$ is unbounded 
as $\theta$ increases.
This states that the acceptance probabilities of our 
sampler and the ideal sampler can be brought arbitrarily close
outside a small set, so long as $\Omega(\theta)$ and
$\Omega(\theta')$ are sufficiently close.
  \begin{lemma}
  %Consider two successive observations, separated by time $t_{diff}$, and
  Suppose %$\theta$ and $\theta'$ satisfy 
  $\frac{1}{K_0} \le \frac{\Omega(\theta)}{\Omega(\theta')} \leq K_0
  $, for $K_0$ satisfying $(1 + \frac{1}{K_0})k_1 \ge 2$  
  ($k_1$ is from Assumption~\ref{asmp:unif_rate}). Write $\mW$ for the
  minimum number of elements of grid $W$ between any successive pairs of observations.
  %$\| \theta'  - \theta \| \leq h$
  For any $\epsilon > 0$, there exist  $w^{K_0}_\epsilon,  \theta_{5, \epsilon}^{K_0} > 0$ such that
  $|P(X| W, \theta, \theta') - P(X | \theta)| < \epsilon$
  for any $(W, \theta)$ with $\mW > w^{K_0}_\epsilon$ and $\| \theta \| > \theta_{5, \epsilon}^{K_0}$.
  \label{lem:eigenvalue_lemma}
  \end{lemma}
  \begin{proof}
% Since $B(\theta, \theta') = I+\frac{A(\theta)}{\Omega(\theta, \theta')}$,
% %there is a one to one mapping between the eigen values of B and the
% it is easy to see that both $A(\theta)$ and $B(\theta,\theta')$ have
% the same stationary distribution, call this $P_{st}(s|\theta)$. Further,
% the eigenvalues of $A(\theta)$ and $B(\theta,\theta')$ satisfy
% $\lambda_B(\theta, \theta') = 1 - \frac{\lambda_A(\theta)}{\Omega(\theta,
% \theta')}$, in particular, the second eigenvalue
% $\lambda^2_B(\theta,\theta')$ of $B$ (which determines its mixing
% properties) equals $1 - \frac{\lambda^2_A(\theta)}{\Omega(\theta, \theta')}$.
%
{From lemma \ref{lem:eig_lemma}, for all $\theta, \theta'$ satisfying 
 the lemma's assumptions, 
 %$\frac{1}{K_0} \le \frac{\Omega(\theta)}{\Omega(\theta')} \leq K_0$ for some $K_0 > 1$ which satisfies $(1 + \frac{1}{K_0})k_1 \ge 2$, 
 the Markov chain with transition matrix $B(\theta, \theta')$ converges 
 geometrically to stationarity distribution $\pi_\theta$  
 at a rate uniformly bounded away from 0.
}
By setting $\mW$ large enough, for all such $(\theta,\theta')$ and for 
any initial state, the Markov chain would have mixed beween each pair of
observations, with distribution over states returning arbitrarily 
close to $\pi_\theta$.


%Our proof strategy is to show that %$\theta$ and $W$ large enough 
%{for $\| \theta \|$ and $\mW$ large enough}, the
%distribution over latent states for the continuous-time MJP and
%its discrete-time counterpart embedded in $W$ can be brought arbitrarily
%close to %$P_{st}$ (and thus 
%to each other. This combined with the
%likelihood $p(X|\theta)$ being bounded  gives the result.
%\do we need to define $P_{st}$?
%We start with the discrete-time system.
%Write $P^{w}(\cdot | W, \theta, \theta' )$ for the $w$-step transition
%probability (equal to $B^w(\theta, \theta')$).
%\boqian{Assumption~\ref{asmp:cond_num} requires
%%that $\lambda^A_2(\theta)$ satisfies
%$\lambda^A_2(\theta) \geq \mu \Omega(\theta)$.
%Together with the assumption $\Omega(\theta') \leq K_0 \Omega(\theta)$,
%this gives
%$\frac{\lambda^A_2(\theta)}{\Omega(\theta) + \Omega(\theta')} \geq
%\mu / (1 + K_0) $, implying that the second eigenvalue of
%$B(\theta, \theta')$ is bounded away from $1$.}\boqian{remove?}
%any $\epsilon' > 0$, there exists a $w_0$ such that for all $W$
%with $|W| > w_0$, we have $\| P^{|W|}(\cdot | W, \theta, \theta')
%- P_{st}(\cdot | \theta)\|{\text{TV}} \leq
%{\epsilon'}$.


%From the assumption \ref{asmp:obs_bnd}, we have %$\exists \ \xi_1 > \eta_1 > 0$, such that,
%$0 < \eta_1 \leq P(X | V=v, \theta) \leq \xi_1$, so that
%the following. Assume the state space has $N$ elements.
Write $W_X$ for the indices of the grid $W$ containing observations, and write $V_X$
for the states of the Markov chain at these times.
Let $P_B(V_X | W, \theta, \theta')$ be the probability distribution over
$V_X$ under the Markov chain with transition matrix $B$ given the grid 
$W$ and $P_{st}(V_X|\theta)$ be the probability of $V_X$ sampled
independently under the stationary distribution. 
Let $P(X | W, \theta, \theta')$ be the marginal probability of the 
observations $X$ under that Markov chain $B(\theta,\theta')$ given $W$. Dropping $W$ and 
$\theta'$ from notation, $P(X|\theta)$
is the probability of the observations under the rate-$A(\theta)$ MJP.

From the first paragraph, for $\mW > w_{0}$ for 
large enough $w_0$,
$P_B(V_X | W, \theta, \theta')$ and  $P_{st}(V_X | W, \theta)$ can be
brought $\epsilon'$ close.
% Then,
% \begin{align*}
%   P(X|W , \theta, \theta') &= \sum_{V_X} P(X | V_X, W, \theta, \theta') P_B(V_X | W, \theta, \theta')\\
% &= \sum_{V_X} P(X | V_{X}, \theta) P_{B}(V_{X} | W, \theta, \theta')\\
% \end{align*}
%  We also define the stationary likelihood as
%  $$P_{st}(X | \theta) = \sum_{V_X} P(X | V_X, \theta) P_{st}(V_X | \theta).$$
Then for any $W$ with $\mW > w_0$, we have
\begin{align*}
  |P(X|W , \theta, \theta') - & P_{st}(X | \theta)| = | \sum_{V_X} P(X|V_X, \theta) [P_B(V_X | W, \theta, \theta') -  P_{st}(V_X | \theta) ]|\\
& \leq \sum_{V_X} P(X | V_X, \theta)|P_B(V_X | W, \theta, \theta') -  P_{st}(V_X | \theta)|\le \epsilon'',
%& \leq \sum_{V_X} P(X | V_X, \theta){\epsilon'} \le \epsilon'' \quad (\text{from Assumption}~\ref{asmp:obs_bnd}).
%& \leq N \| X \| \ub \epsilon' := \epsilon''
\end{align*}
using $P(X|V_X,\theta) \le \lb$ 
(Assumption~\ref{asmp:obs_bnd}), and 
$\sum_{V_X} |P_B(V_X | W, \theta, \theta') -  P_{st}(V_X | \theta)| < \epsilon$.
%So for $W$ with $|W| > W_0$, we have  $|P(X| W, \theta, \theta') - P_{st}(X | \theta)| < \epsilon / 4$. \\
For large $\theta$, we prove a similar result in the continuous 
case. For any $\theta'$, by uniformization, 
\begin{align*}
P(X | \theta) %&= \int_W dW P(X , W | \theta, \theta')\\
= \int dW P(X | W, \theta, \theta') P(W | \theta, \theta') %+ \int_{\mW \leq w_0}dW P(X | W, \theta, \theta') P(W | \theta, \theta').
\quad 
(\text{$P(W|\theta,\theta')$ is a rate-$\Omega(\theta) + 
\Omega(\theta')$ Poisson process}).
\end{align*}
%\sum_{v} P(X | V_{|W|} = v, \theta) P^{(|W|)}(V_{|W|} = v | W, \theta, \theta') P(W|\theta, \theta')
%&= \sum_W \sum_{V_{|W|}} P(X | V_{|W|}, \theta) P^{(|W|)}(V_{|W|} | W, \theta, \theta') P(W|\theta, \theta')\\
%&= \sum_{W s.t. |W| \leq W_0'} \sum_{V_{|W|}} P(X | V_{|W|}, \theta) P^{(|W|)}(V_{|W|} | W, \theta, \theta') P(W|\theta, \theta') + \\
%&\sum_{W s.t. |W| > W_0'} \sum_{V_{|W|}} P(X | V_{|W|}, \theta) P^{(|W|)}(V_{|W|} | W, \theta, \theta') P(W|\theta, \theta')
%\end{align*}
% Consider the difference between the first term $\sum_{W s.t. |W| > W_0} P(X | W, \theta, \theta') P(W | \theta, \theta')$ and $P_{st}(X | \theta)P(|W| > W_0 | \theta, \theta')$. Also from the previous derivations, $|P(X| W, \theta, \theta') - P_{st}(X | \theta)| < \epsilon / 4$ for $W$ with $|W| > W_0$.\\
% \begin{align*}
% &|\sum_{W s.t. |W| > W_0} P(X | W, \theta, \theta') P(W | \theta, \theta')  - P_{st}(X | \theta)P(|W| > W_0 | \theta, \theta')|  \leq  \\
% &\sum_{W s.t. |W| > W_0} |P(X | W, \theta, \theta') - P_{st}(X | \theta)|  P(W | \theta, \theta') \leq P(|W| > W_0 | \theta, \theta') \epsilon / 4
% \end{align*}
% For the $W_0$, there exists $\theta_0 > 0$ such that for any $\theta > \theta_0$,and any $\theta'$, $P(|W| \leq W_0 | \theta, \theta') < \frac{\epsilon}{4\xi_1}.$
% For the second term, we have
% \begin{align*}
% &|\sum_{W s.t. |W| \leq W_0} \sum_{v} P(X | V_{|W|} = v, \theta) P^{(|W|)}(V_{|W|} = v | W, \theta, \theta') P(W|\theta, \theta')|  \\
% &\leq \xi_1 P(|W| \leq W_0 | \theta, \theta') \leq \epsilon / 4
% \end{align*}
% Consider the difference between $P(X|\theta)$ and $P_{st}(X|\theta)$ for $\theta > \theta_0$, and any $\theta'$ with $\Omega(\theta') \le K_0\Omega(\theta)$.
% \begin{align*}
% |P(X|\theta) - P_{st}(X|\theta)| &\leq |\sum_{W s.t. |W| > W_0} P(X | W, \theta, \theta') P(W | \theta, \theta')  - P_{st}(X | \theta)P(|W| > W_0 |\theta, \theta')| \\
% &+ |\sum_{W s.t. |W| \leq W_0} \sum_{v} P(X | V_{|W|} = v, \theta) P^{(|W|)}(V_{|W|} = v | W, \theta, \theta') P(W|\theta, \theta')|\\
% &+ |P_{st}(X | \theta)P(|W| \leq W_0 | \theta, \theta')|\\
% &\leq \epsilon / 4 + \epsilon / 4 + \epsilon / 4 = 3\epsilon/4
% \end{align*}
We split this integral into two, one over the set $\{\mW > w_0\}$, and 
the second over its complement. On the former, for $w_0$ large enough, 
$|P(X |W, \theta, \theta')-P_{st}(X|\theta)|
\le \epsilon''$. %Since, under uniformization, $W$ comes from a 
By choosing $\theta$ large enough, $\{\mW > w_0\}$ occurs
with arbitrarily high probability for any $\theta'$. Since the likelihood is bounded, the
integral over the second set can be made arbitrarily small (say, $\epsilon''$ again). 
Finally, from the triangle
inequality,
%So for $\epsilon > 0$, and $K_0 > 0$, if $|W| > W_0$ and $\theta > \theta_0$ and $\Omega(\theta') \leq K_0 \Omega(\theta)$, we have
\begin{align*}
|P(X | \theta) - P(X | W, \theta, \theta')| &\leq |P(X | \theta) -P_{st}(X | \theta) | + | P_{st}(X | \theta) -  P(X | W, \theta, \theta')|\\
       & \leq (\epsilon'' + \epsilon'') + \epsilon'' \assign \epsilon
\end{align*}
\end{proof}

\begin{proposition}
%Write $P(W', \theta' | W, \theta, \vartheta)$ for the transition kernel of our MCMC sampler given the observations. For any positive $\epsilon$, there exist $\theta_\epsilon > 0$ and corresponding $W_\epsilon > 0$, such that for any $\theta \ge \theta_\epsilon$, and any $\vartheta$, we have
Let $(W, \theta, \vartheta)$ be the current state of the sampler, and 
define $\E_\epsilon = \{(W', \theta'): 
|\alpha_I(\theta,\theta';X) - \alpha(\theta,\theta';W',X)| \le \epsilon\}$,
 %Given this, let $E$ be the event that $| \alpha_I(\theta,\vartheta,X) -
 %\alpha(\theta,\vartheta,W',X)| \le \epsilon $.
%Write $q(W', \theta' | W, \theta, \vartheta)$ for the proposal
%distribution of our MCMC sampler.
Then, for any $\epsilon$, there exists $\theta_\epsilon > 0$ 
 such that for $\theta$ satisfying $\| \theta \| > \theta_\epsilon$ and any $\vartheta$, we have
%and $W_\epsilon > 0$, such that for
%$\theta \ge \theta_\epsilon$, and any $\vartheta$, we have
%  $P(|W'| > W_\epsilon, \theta' > \theta_\epsilon  | W, \theta, \vartheta) \ge 1-\epsilon$.
%  $| \alpha_I(\theta,\vartheta,X) - \alpha(\theta,\vartheta,W',X)|
%  \le \epsilon $ for any $(W', \theta')$ with $|W'| > W_\epsilon$ and $\theta' > \theta_\epsilon$.
  %$(\theta,\theta^*)$ satisfying
$P(E_\epsilon|W,\theta,\vartheta) > 1-\epsilon$.
\label{prop:mix0}
\end{proposition}
The previous lemma bounded the difference in probability of observations 
under the discrete-time and continuous-time processes. This result 
uses this to bound the acceptance probabilities of the ideal sampler, 
and our proposed sampler (where acceptace probabilities are calculated 
conditioned on the grid $W$). See the appendix for the proof.
%   \begin{proposition}
%     Write $P_{\theta}(W',\theta^*)$ for the conditional distribution
%     over $(W',\theta^*)$ given the current state of the MCMC sampler and the
%     observations, and let
%     $\lim_{\theta\rightarrow\infty} \Mx{\theta} = \infty$. Then for any positive
%     $\epsilon$, there exist $B_\theta$ such that for all
%     $\theta \ge B_\theta$, % with $\Omega(\theta) \ge k\Omega(\vartheta)$,
%     we have
%     $P_{\theta}(\{W'\ s.t.\ |\alpha_I(\theta,\vartheta,X) - \alpha(\theta,\vartheta,W',X)|
%     \le \epsilon)\} \ge 1-\epsilon$.%$(\theta,\theta^*)$ satisfying
%   \label{prop:mix}
%   \end{proposition}
%   \begin{proof}
%   Consider the nearest pair of observations, occuring at times $t$ and
%   $t + \Delta$. %let them be separated by a time interval $\Delta$.
%   By setting $\theta$ high enough, assumption~\ref{asmp:cond_num} ensures
%   that the distribution over waiting times of each state can be concentrated
%   arbitrarily close to $0$. Thus, the distribution over states after an
%   interval $\Delta$ can be brought arbitrarily close to the equilibrium
%   distribution of $A(\theta)$  (call this $p_{\theta}$).

%   Next, recall that the embedded Markov chain has a transition matrix
%   given by $B(\theta^*,\theta) = (I + A(\theta^*)/\Omega(\theta,\theta^*))$.
%   For any setting of $\Omega$, this has the same stationary distribution
%   $p_{\theta}$: this can be verified by multiplying $B(\theta,\theta^*)$ with
%   $p_\theta$. For the embedded Markov chain to be brought close to
%   equilibrium over $[t,t+\Delta]$, we need two conditions, 1) the number of
%   Poisson events must be large enough (to allow sufficient transitions), and 2) the
%   transition matrix $B$ must mix well enough (otherwise even a
%   large number of transition opportunities will not forget the initial state).
%   We show that for $\theta$ large enough, this holds with high probability.
%   Assumption~\ref{asmp:mono_tail} will ensure that this continues to hold
%   for all larger $\theta$ as well.

%   Recall first that the proposal distribution $q(\theta^*|\theta)$ is
%   centered at the current value $\theta$. For any $\epsilon$, we can find
%   an interval $[\theta-h,\theta+h]$ such that for all $\theta$,
%   $q(\theta^* \in [\theta-h,\theta+h]|\theta)
%   \ge 1-\epsilon$. By choosing $\theta$ large
%   enough, we can ensure that %$\Omega(\theta) \approx \Omega(\theta^*)$ (i.e.\
%   $\frac{\Omega(\theta)}{ \Omega(\theta^*)} \in [1-\epsilon,1+\epsilon]$)
%   with probability greater than $1-\epsilon$.
%   %The condition $\Omega(\theta) \ge k\Omega(\vartheta)$,
%   This, together with assumption~\ref{asmp:cond_num}, ensures that with
%   probability greater than $1-\epsilon$, the self-transition probability of $B(\theta,\vartheta)$
%   is bounded away from one, and limits how poorly $B$ can mix.
%   Second, since $W$ comes from a Poisson process with intensity
%   $\Omega(\theta')+ \Omega(\theta) - A_{S(t)}(\theta)$, a large $\theta'$
%   ensures a large $|W'|$ with high probability: by setting $B_\theta$ large
%   enough we can ensure $|W'|$ is large enough with arbitrary probability.

%   Thus, for large enough $B_\theta$ we can ensure that for all $\theta >
%   B_\theta$, the distributions $p(x_{t+\Delta}|s_t)$ and
%   $p(x_{t+\Delta}|W,s_t)$ can be brought arbitrarily close with arbitrarily
%   high probability.
%   By a simple chaining argument, this holds for $p(X|\theta)$ and
%   $p(X|\theta,W)$ as well and so too
%     \vinayak{expand}
%   $\alpha_I(\theta',\theta,X)$ and $\alpha(\theta',\theta,W,X)$
%   \end{proof}
%\begin{lemma}
%Write $P(W' | W, \theta, \vartheta, \theta')$ as the transition density with respect to the grids $W'$, there exists $\psi > 1$, such that $P(W' | W, \theta, \vartheta, \theta') \leq \psi ^{|W'|} \tilde{\Omega}(\theta,\theta')^{|W'|}\exp(-\tilde{\Omega}(\theta,\theta')\tau)$, where $\tilde{\Omega}(\theta, \theta') = (\Omega(\theta) + \Omega(\theta')) / \psi$ and $\tau$ is the length of the time interval.
%\end{lemma}
%\begin{proof}
%As we defined,  there exists $k > 1$, such that $\Omega(\theta) = k \max\{ A_s(\theta)\}$. Recall that given the current state $(W, \theta, \vartheta)$, we first sample $(S, T)$, which is a discrete time Markov chain with rate matrix $B(\theta, \vartheta) = I + \frac{A(\theta)}{\Omega(\theta) + \Omega(\vartheta)}$. Then we sample $W'$ which is a Poisson process with rate $\Omega(\theta) + \Omega(\theta') - A_{S_t}(\theta)$. Integrating out $S, T$, we can get the transition density with respect to the grids $W'$. We have
%\begin{align*}
%P(W' | W, \theta, \vartheta, \theta') &= \sum_{S,T} P(S, T | W, \theta, \vartheta) P(W' | S, T, \theta, \theta') .
%\end{align*}
%Given any $S, T$, the Poisson rate has a lower bound $\frac{k - 1}{k} (\Omega(\theta) +\Omega(\theta'))$. So
%\begin{align*}
%P(W' | S, T, \theta, \theta') &= \prod_{i = 0}^{|T|} (\Omega(\theta) + \Omega(\theta') - A_{S_i}(\theta))^{|W_i'|} \exp(-(\Omega(\theta) + \Omega(\theta') - A_{S_i}(\theta))(T_{i + 1} - T_i))\\
%& \leq \prod_{i = 0}^{|T|} (\frac{k - 1}{k}(\Omega(\theta) + \Omega(\theta')))^{|W_i'|} \exp(- \frac{k - 1}{k} (\Omega(\theta) + \Omega(\theta'))(T_{i + 1} - T_i))(\frac{k}{k - 1})^{|W'|} \\
%&\leq (\frac{k - 1}{k}(\Omega(\theta) + \Omega(\theta')))^{|W'|} \exp(- \frac{k - 1}{k} (\Omega(\theta) + \Omega(\theta'))(T_{|T| + 1} - T_0))(\frac{k}{k - 1})^{|W'|}
%\end{align*}
%So we can set $\psi = \frac{k}{k - 1}$ and $\tilde{\Omega}(\theta, \theta') = (\Omega(\theta) + \Omega(\theta')) / \psi$, and then sum up $S, T$.
%\begin{align*}
%P(W' | W, \theta, \vartheta, \theta') &= \sum_{S,T} P(S, T | W, \theta, \vartheta) P(W' | S, T, \theta, \theta') \\
%& \leq  \psi ^{|W'|} \tilde{\Omega}(\theta,\theta')^{|W'|}\exp(-\tilde{\Omega}(\theta,\theta')\tau)
%\end{align*}
%\end{proof}

\begin{lemma}(drift condition) $\exists \delta_2 \in (0, 1), L > 0$
  s.t.
  $\mathbb{E}\left[\lambda_1|W'| + \Omega(\theta')  | W, \theta, \vartheta, X\right]
  \leq (1 - \delta_2)\left(\lambda_1|W| + \Omega(\theta)   \right) + L$ %where $\lambda = \lceil \frac{(t_{end})k_2(\eta_0 + 1)}{(\eta_1^2 \kappa_1 \mathbb{P}_\phi(C)/\xi_1^2 - \eta_0)} \rceil.$
\label{lem:drift}
\end{lemma}
\begin{proof}
Since $W'=T\cup U'$, we consider $\mathbb{E}[|T| |W,\theta,\vartheta,X]$
and $\mathbb{E}[|U'| | W, \theta, \vartheta, X]$ separately.
An upper bound of $\mathbb{E}[|T| | W,\theta,\vartheta, X]$ can be derived
directly from proposition~\ref{prop:self_tr}:
\begin{align*}
\mathbb{E}[|T| |W,\theta,\vartheta,X] &= \mathbb{E}[\sum_{i = 0}^{|W|-1}
  \mathbb{I}_{\{ V_{i + 1} \neq V_i \}}| W, \theta, \vartheta, X]
\leq \sum_{i = 0}^{|W| - 1} (1 - \delta_1) = |W|(1 - \delta_1).
\end{align*}
By corollary \ref{corol:integ_bound}, there exist $\eta_1 , \theta_2$ 
such that for 
%$\theta$  satisfying 
$ \| \theta \| > \theta_2$, 
 $ \int \Omega(\ptheta) q(\ptheta | \theta)d\ptheta \leq \eta_1 \Omega(\theta) 
 $. Then,
\begin{align*}
%So for $\theta$ satisfying $ \| \theta \| > \theta_3$, we have
\mathbb{E}[|U'| |W, \theta, \vartheta, X] &= 
\mathbb{E}_{S,T, \ptheta}\mathbb{E}[|U'| | S, T, W, \theta, \vartheta, \ptheta, X] = \mathbb{E}_{S,T, \ptheta}\mathbb{E}[|U'| | S, T, W, \theta, \ptheta] \\
& \leq \mathbb{E}_{S,T, \ptheta} \left[t_{end}\Omega(\theta, \ptheta)\right] = t_{end}\int \Omega(\theta, \ptheta) q(\ptheta | \theta) d\ptheta\\
& = t_{end} \left[ \left(  \Omega(\theta) +
\int_\Theta \Omega(\ptheta) q(\ptheta | \theta)d\ptheta \right) \right] 
 \leq t_{end} (\eta_1 + 1) \Omega(\theta) %\right]
%\assign a \Omega(\theta) + b.
\end{align*}
%Next, we note that $\vartheta'$ takes on value $\theta$ with
%probability of acceptance, else it takes the value $\ptheta$ proposed  from
%$q(\ptheta|\theta)$. We bound the acceptance
%probability by $1$, so that
%\begin{align}
%\mathbb{E}[\Omega(\vartheta')|\theta,\vartheta,W,X)] &\le \Omega(\theta)
%+ \int d\ptheta (1-\alpha(\ptheta,\theta)) q(\ptheta|\theta) \Omega(\ptheta)\nonumber \\
 % & \le (1+\eta_0) \Omega(\theta)
%\end{align}
To bound $\mathbb{E}\left[\Omega(\theta')  | W, \theta, \vartheta, X\right]$,
consider the transition probability over $(W',\theta')$:
\begin{align*}
  P(dW', d\theta'&| W, \theta, \vartheta)
=d\theta' dW' \left[q(\theta' | \theta)
  \sum_{S,T} P(S, T | W, \theta, \vartheta, X)P(W' | S, T, \theta, \theta')
\alpha(\theta, \theta' ; W', X)\right. \\
&\left.+ \int q(\ptheta | \theta) \sum_{S,T} P(S, T|W,\theta,\vartheta,
    X)P(W' | S, T, \theta, \ptheta) ( 1 - {\alpha(\theta, \ptheta ; W', X)})d\ptheta
    \delta_\theta(\theta')\right].
\end{align*}
With $P(W' | W, \theta, \vartheta, \theta', X) =
\sum_{S,T} P(S, T | W, \theta, \vartheta, X)P(W' | S, T, \theta, \theta')$,
integrate out $W'$:
\begin{align*}
  P(d\theta'| W, \theta, \vartheta) &=d\theta' \int_{W'}dW'
  \left[q(\theta' | \theta)
     P(W' | W, \theta, \vartheta, \theta', X) \alpha(\theta, \theta' ; W', X) + \right.\\
  &\left.  \int q(\ptheta | \theta)  P(W' |  W, \theta, \vartheta, \ptheta,
X) ( 1 - \alpha(\theta, \ptheta ; W', X))d\ptheta
\delta_\theta(\theta')\right] %\\
%&\assign d\theta' I_1(W, \theta', \theta, \vartheta) + d\theta'\delta_\theta(\theta')I_2(W, \theta, \vartheta).
\end{align*}
%Then let  $\int \Omega(\theta') P(d\theta'| W, \theta, \vartheta)
%  = \int d\theta' \Omega(\theta') I_1(W, \theta', \theta, \vartheta) + \Omega(\theta) I_2(W, \theta, \vartheta) 
%$, with
Then let  $\int \Omega(\theta') P(d\theta'| W, \theta, \vartheta)
  = I_1(W, \theta, \vartheta) + \Omega(\theta) I_2(W, \theta, \vartheta) 
$, with
\begin{align*}
  &I_1(W, \theta, \vartheta) = \int d\theta' \Omega(\theta') q(\theta' | \theta)\int dW'P(W' | W, \theta, \vartheta, \theta', X)\alpha(\theta, \theta' ; W', X), \\
&I_2(W, \theta, \vartheta) =\int d\ptheta  dW'q(\ptheta | \theta)P(W' | W, \theta, \vartheta, \ptheta, X)(1 - \alpha(\theta, \ptheta ; W', X)).
\end{align*}
{Consider the second term $I_2$.
  From Proposition~\ref{prop:mix0}, for any positive $\epsilon$, there
  exists $\theta_\epsilon > 0$ such that the set $E_{\epsilon}$
  (where $|\alpha(\theta, \ptheta ; X,W') - \alpha_I(\theta, \ptheta ; X)| \le
  \epsilon$) has probability greater than $1-\epsilon$.
  Write $I_{2,\E_{\epsilon}}$ for the integral restricted to this set, and
  $I_{2,\E_{\epsilon}^c}$ for that over the complement, so that
 $I_{2}= I_{2,\E_{\epsilon}}+I_{2,\E_{\epsilon}^c}$.
 Then for $\theta > \theta_\epsilon$,
% such that for $\theta > \theta_0$, we have $P(\theta' > \theta_0, |W'| >
% W_0 |\theta, \vartheta, W) > 1 - \epsilon$. Defining $\E_{\epsilon} =
% \{ (\theta', W') | \theta' > \theta_0, |W'| > W_0\}$, we can divide the
% area of integration into two complementary parts: $\E_{\epsilon}$, the
% part of $(\ptheta,W')$ where
%  $|\alpha(\theta, \ptheta | X,W') - \alpha_I(\theta, \ptheta | X)| \le \epsilon$,
%and its complement $\E^c_{\epsilon}$. Call these $I_{2,\E_{\epsilon}}$ and
%$I_{2,\E_{\epsilon}^c}$ respectively.
%For $\theta > \theta_0$, the second term has probability less than $\epsilon$, and using the bound
%$1-\alpha \le 1$, we have
}
%use $\alpha(\theta, \theta' | W', X)\le 1$ and $q(\theta'|\theta) < \epsilon$
%to get}
%%and equation~\eqref{eq:acc} to get}
%  P(d\theta'| &W, \theta, \vartheta)
%\leq d\theta' \epsilon
% &\left.\left(1 -  \int q(\ptheta | \theta) \sum_S P(S, T|W, \theta,
% \vartheta, X)P(W'|S, T, \theta, \ptheta)\alpha_I(\theta, \ptheta)
% \frac{\eta_1^2}{\xi_1^2}d\ptheta dW' dT\right) \delta_\theta(\theta')\right]\\
%          & \leq d\ptheta \left[q(\ptheta | \theta) + \left(1 -
%          \frac{\eta_1^2}{\xi_1^2} \int_\Theta q(\ptheta |
%      \theta)\alpha_I(\theta, \ptheta) d\ptheta\right)\delta_\theta(\theta')\right]
%      \numberthis \label{eq:nu1}
\begin{align*}
I_{2,\E_{\epsilon}}(W, \theta, \vartheta) &= \int_{\E_{\epsilon}} d\ptheta  dW'q(\ptheta | \theta)P(W' | W, \theta, \vartheta, \ptheta, X)(1 - \alpha(\theta, \ptheta ; W', X)) \\
&\le \int_{\E_\epsilon}d\ptheta dW' q(\ptheta | \theta)
  P(W' | W, \theta, \vartheta, \ptheta, X)  [ 1 - (\alpha_I(\theta, \ptheta ; X)-\epsilon)] \\
&\le   \int d\ptheta dW'  q(\ptheta | \theta)
  P(W' | W, \theta, \vartheta, \ptheta, X)
  [ 1 - (\alpha_I(\theta, \ptheta ; X)-\epsilon)] \\
  &\le (1+\epsilon)  - \int  q(\ptheta | \theta) \alpha_I(\theta, \ptheta ; X) d\ptheta, \quad \text{and}  \\
  I_{2,\E_{\epsilon}^c}(W, \theta, \vartheta)  &= \int_{\E^c_{\epsilon}} d\ptheta  dW'q(\ptheta | \theta)P(W' | W, \theta, \vartheta, \ptheta, X)(1 - \alpha(\theta, \ptheta ; W', X)) \\
  &\le \int_{\E^c_{\epsilon}} d\ptheta dW'
  q(\ptheta | \theta) P(W' | W, \theta, \vartheta, \ptheta, X) \le \epsilon.
% &= \int_{\E^c_{\epsilon}} d\ptheta dW'
% q(\ptheta | \theta) P(W' |  W, \theta, \ptheta, \vartheta, X)
\end{align*}
%\int_{W_1}dW' \int_S q(\ptheta | \theta) &\int \sum_S P(S, T | W, \theta, \vartheta,
%X)P(W' | S, T, \theta, \ptheta)dT ( 1 - [\alpha_I(\theta, \ptheta |
%X,W')])d\ptheta \\
%I_{2,\E_{\epsilon}} \le \int_{\E_\epsilon}dW' d\ptheta q(\ptheta | \theta) &
%  \sum_{S,T} P(S, T | W, \theta, \vartheta, X)P(W' | S, T, \theta, \ptheta)
%  [ 1 - (\alpha_I(\theta, \ptheta | X)-\epsilon)] \\
%\le \int dW' d\ptheta q(\ptheta | \theta) &
%  P(W' | W, \theta, \vartheta, \ptheta, X)
 % [ 1 - (\alpha_I(\theta, \ptheta | X)-\epsilon)] \\
 %\le (1+\epsilon) & - \int  q(\ptheta | \theta) \alpha_I(\theta, \ptheta | X) d\ptheta, \text{while} \\
%   \intertext{Then}
%      P(d\theta'| &W, \theta, \vartheta)   \le d\theta' \int_{W'}dW'
%      \left[q(\theta' | \theta) \int \sum_S P(S, T
%   | W, \theta, \vartheta, X)P(W' | S, T, \theta, \theta')dT
%   \left[\alpha_I(\theta', \theta | X)+\epsilon_{}\right]\right.\\
%   &\left. + \int q(\ptheta | \theta) \int \sum_S P(S, T | W, \theta, \vartheta,
%   X)P(W' | S, T, \theta, \ptheta)dT ( 1 - [\alpha_I(\theta, \ptheta |
%   X)-\epsilon_\ptheta])d\ptheta \delta_\theta(\theta')\right]  \\
%   & \leq d\theta'\left[q(\theta' | \theta)\left[\alpha_I(\theta', \theta |
%   X)+\epsilon_{\theta'}\right] +
%   \left(1 -  \int q(\ptheta | \theta) [\alpha_I(\ptheta, \theta|X)-\epsilon_\ptheta]
%   d\ptheta \right) \delta_\theta(\theta')\right]\\
%   & \leq d\theta'\left[q(\theta' | \theta) \alpha_I(\theta', \theta | X) +
%   \left(1 -  \int q(\ptheta | \theta) \alpha_I(\theta, \ptheta|X) d\ptheta \right)
%   \delta_\theta(\theta')\right]+\epsilon_{\theta'}q(\theta'|\theta)+
%   \delta_\theta(\theta')\int \epsilon_{\ptheta} q(\ptheta|\theta) d\ptheta \\
%   & = p_I(\theta'|\theta,X) + \epsilon_{\theta'}q(\theta'|\theta)+
%   \delta_\theta(\theta')\int \epsilon_{\ptheta} q(\ptheta|\theta) d\ptheta
%   \numberthis \label{eq:nu2}
%, and bounding
%$\alpha$ by one on the $W$-set with probability $\epsilon$, we have the
%bound}
%the first term involves an integral over $(\theta',W')$ which 
We similarly divide the integral $I_1$ into two parts, $I_{1,E_\epsilon}$ 
(over $\E_\epsilon$) and $I_{1,E_\epsilon^c}$ (over its complement 
$\E^c_\epsilon$).
%From lemma 8, we have
%$P(W' | W, \theta, \vartheta, \theta') \le \psi ^{|W'|} \tilde{\Omega}(\theta,\theta')^{|W'|}\exp(-\tilde{\Omega}(\theta,\theta')\tau)$, where $\tilde{\Omega} = (\Omega(\theta) + \Omega(\theta')) / \psi$ and $0 < \psi < 1$ and $\tau$ is the length of the time interval.\\
%\begin{align*}
%  \tilde{I}_{1,\E^c_\epsilon} &\le  \int_{\E^c_\epsilon}  \Omega(\theta') q(\theta' | \theta)P(W' | W, \theta, \vartheta, \theta')d\theta'dW'\\
%  &=  \int_{\theta' \leq \theta_0}  \Omega(\theta') q(\theta' | \theta)P(W' | W, \theta, \vartheta, \theta')d\theta'dW' +  \int_{\theta' > \theta_0, |W'| < W_0}  \Omega(\theta') q(\theta' | \theta)P(W' | W, \theta, \vartheta, \theta')d\theta'dW'\\
%  & \leq  \Omega(\theta_0) \int_{\theta' > \theta_0}q(\theta' | \theta) d\theta' + \int_{\theta' > \theta_0} \Omega(\theta') q(\theta' | \theta) [ \sum_{w \le W_0}\frac{\tilde{\Omega}(\theta,\theta')^{w}}{w!}\exp(-\tilde{\Omega}(\theta,\theta')\tau)]d\theta'\\
%  & \leq  \Omega(\theta) \epsilon + \int_{\Theta} \Omega(\theta') q(\theta' | \theta)\epsilon d\theta'\\
%  &= \Omega(\theta) \epsilon (1 + \eta_0)
%  \{?}
%  \end{align*}
%  For $\epsilon > 0$, there exists $\theta_1 > 0$, such that for $\theta > \theta_1$ $\int_{\theta' > \theta_0}q(\theta' | \theta) d\theta' \le \epsilon$. There also exists $\theta_2 > 0$, such that for $\theta > \theta_2$ and any $\theta'$,  $\sum_{w \le W_0}\frac{\tilde{\Omega}(\theta,\theta')^{w}}{w!}\exp(-\tilde{\Omega}(\theta,\theta')\tau) < \epsilon$. For $\theta > \max(\theta_0, \theta_1, \theta_2)$, we have
%\begin{align*}
%\tilde{I}_{1,\E^c_\epsilon} &\le \Omega(\theta_0) \int_{\theta' > \theta_0}q(\theta' | \theta) d\theta' + \int_{\theta' > \theta_0} \Omega(\theta') q(\theta' | \theta) [ \sum_{w \le W_0}\frac{\tilde{\Omega}(\theta,\theta')^{w}}{w!}\exp(-\tilde{\Omega}(\theta,\theta')\tau)]d\theta'\\
%& \le  \Omega(\theta_0) \epsilon + \int_{\Theta} \Omega(\theta') q(\theta' | \theta)\epsilon d\theta' \\
%& \le \Omega(\theta_0) \epsilon + \eta_0 \Omega(\theta)\epsilon \\
%& \le (1 + \eta_0) \Omega(\theta)\epsilon
%\end{align*}
%$$\int d\theta' \Omega(\theta') I_1(W, \theta', \theta, \vartheta) =
%$ I_1(W, \theta, \vartheta) = \int_{\E_\epsilon \cup \E_\epsilon^c} 
%\hspace{-.0in} d\theta' \Omega(\theta')I_1(W,\theta',\theta,\vartheta) \assign I_{1,\epsilon} + I_{1,\epsilon}^c. \\
%$
%
%\int_{\E_\epsilon} d\theta' \Omega(\theta') I_1(W, \theta', \theta, \vartheta) +
%\int_{\E_\epsilon^c} d\theta' \Omega(\theta') I_1(W, \theta', \theta, \vartheta).
For $\|\theta\|$ large enough, we can bound the acceptance probability 
by $\alpha_I(\theta,\theta'; X) + \epsilon$ on the set $E_\epsilon$, and by corollary 
\ref{corol:integ_bound}, we get 
\begin{align*}
%\int_{\E_\epsilon} d\theta' \Omega(\theta') I_1(W, \theta', \theta, \vartheta)  &
  I_{1,E_\epsilon} & \leq \int_{\E_\epsilon} \Omega(\theta')q(\theta' | \theta) (\alpha_I(\theta, \theta'; X) + \epsilon) d\theta' 
 \leq \int \Omega(\theta')q(\theta' | \theta) \alpha_I(\theta, \theta'; X) d\theta' + \eta_1 \epsilon \Omega(\theta).
\end{align*}
For $I_{1,E_\epsilon^c}$, 
from assumption \ref{asmp:integ_bound}, %there exists $\eta_0 , \theta_2 > 0$  
%such that for $\theta$ satisfying $ \| \theta \| > \theta_2$, 
we have $\int_\Theta \Omega(\ptheta)^2 q(\ptheta | \theta)d\ptheta \leq \eta_0 \Omega(\theta)^2$
for $\|\theta\| > \theta_2$.
So, by Cauchy-Schwarz inequality and bounding the acceptance probability by one, we have
\begin{align*}
%\left[  \int_{\E_\epsilon^c} I_1(W, \theta', \theta, \vartheta) \Omega(\theta') d\theta'  \right]^2 
  \left( I_{1,E_\epsilon^c}\right)^2 %\le \left[  \int_{\E_\epsilon^c}  \Omega(\theta') q(\theta' | \theta) P(W' | W, \theta, \vartheta, \theta', X)d\theta' \right]^2\\
& \le \int_{\E_\epsilon^c} q(\theta' | \theta) P(W' |W, \theta, \vartheta, \theta', X)d\theta'dW'  \int_{\E_\epsilon^c}  \Omega(\theta')^2 q(\theta' | \theta) P(W' |W, \theta, \vartheta, \theta', X)d\theta'dW'   \\
& \le \epsilon \int  \Omega(\theta')^2 q(\theta' | \theta)d\theta' 
 \le \epsilon \eta_0 \Omega(\theta)^2 ,
\end{align*}
giving
$
I_{1,E_\epsilon^c}  \le \sqrt{\epsilon \eta_0 }\Omega(\theta) .
$
Now for $\theta$ satisfying $ \| \theta \| >\max(\theta_2, 
\theta_\epsilon, M) $ 
(where $M$ is from Assumption~\ref{asmp:ideal_geom} on 
the ideal sampler), we have 
\begin{align*}
  \int \Omega(\theta') P(d\theta'| W, \theta, \vartheta)
%  &= \int_{\E_\epsilon} I_1(W, \theta', \theta, \vartheta) \Omega(\theta') d\theta' + \int_{\E_\epsilon^c} I_1(W, \theta', \theta, \vartheta) \Omega(\theta') d\theta' \\ &+ I_{2,\E_{\epsilon}}(W, \theta, \vartheta) + I_{2,\E_{\epsilon}^c}(W, \theta, \vartheta) \\
  & \leq \int \Omega(\theta') q(\theta' | \theta)\alpha_I(\theta, \theta'| X) d\theta'  + \Omega(\theta)\int  q(\ptheta | \theta) (1 - \alpha_I(\theta, \ptheta | X)) d\ptheta+ \\
  &\sqrt{\eta_0\epsilon}\Omega(\theta)  +  \eta_1 \epsilon \Omega(\theta) + 2\epsilon \Omega(\theta)\\
  & \leq (1 - \rho) \Omega(\theta) + (\sqrt{\eta_0\epsilon} +  \eta_1 \epsilon + 2\epsilon) \Omega(\theta) + L_I, \quad \text{giving}
\end{align*}
%For $\theta$ satisfying $ \| \theta \| >\max(\theta_2, \theta_3, \theta_\epsilon)$, we have %\boqian{need to change afterwards}
\vspace{-.3in}
\begin{align*}
\mathbb{E}[\lambda_1 | W'| &+ \Omega(\theta')| W, \theta, \vartheta, X] \le \lambda_1(1 - \delta_1)|W| + \lambda_1 t_{end} (1 + \eta_1)\Omega(\theta)\\
&+  (1 - \rho) \Omega(\theta) + (\sqrt{\eta_0} \sqrt{\epsilon} +  \eta_1 \epsilon + 2\epsilon) \Omega(\theta) + L_I\\
& = (1 - \delta_1)\lambda_1 |W| + [1 - (\rho - \lambda_1 t_{end} (1 + \eta_1) - (2 + \eta_1)\epsilon - \sqrt{\eta_0 \epsilon})]\Omega(\theta) + L_I\\
& \assign (1 - \delta_1)\lambda_1 |W| + (1 - \delta_2)\Omega(\theta) + L_I
\end{align*}
For $(\lambda_1,\epsilon)$ small enough, $\delta_2 \in (0,1)$, 
%If we 
%fix $\lambda_1$ small enough, this holds outside some small set $C$.
%Inside that compact set, define 
%$L = \sup_{(W, \theta)} \{ \mathbb{E}[\tilde{\lambda}_1 | W'| + 
%      \Omega(\theta')| W, \theta, \vartheta, X] \}$.
  %if we fix $\lambda_1$
  %There exists  $\tilde{\epsilon} > 0$ and $\tilde{\lambda}_1 >0 $ , such that $\rho >  \tilde{\lambda}_1 t_{end}  (1 + \eta_1) + (2 + \eta_1)\tilde{\epsilon} + \sqrt{\eta_0} \sqrt{\tilde{\epsilon}}$, and denote $\rho_2  \assign \rho - \tilde{\lambda}_1 t_{end}  (1 + \eta_1) + (2 + \eta_1)\tilde{\epsilon} + \sqrt{\eta_0} \sqrt{\tilde{\epsilon}}
  % > 0$. Let $\tilde{\rho} = \min(\rho_2, \delta_1)$. So there exists a small set $C$ such that \begin{align*}
  %\mathbb{E}[\tilde{\lambda}_1 | W'| + \Omega(\theta')| W, \theta, \vartheta, X] &\le (1 - \tilde{\rho}) [\tilde{\lambda}_1 | W| + \Omega(\theta)] +
  %\mathbb{I}_{C} \sup_{(W, \theta)} \{ \mathbb{E}[\tilde{\lambda}_1 | W'| + \Omega(\theta')| W, \theta, \vartheta, X] \}.
  %\end{align*}
      and  $\delta = \min(\delta_1,\delta_2)$ gives the drift condition.
      \qed
% From equation~\eqref{eq:nu2} and assumption 2, we have for
% $\Omega(\theta) > k\Omega(\vartheta)$
% \begin{align*}
% \int \Omega(\theta')P(d\theta'| W, \theta, \vartheta)  &\leq
% \int \Omega(\theta') p_I(\theta'|\theta,X) d\theta' +\int d\theta'
% \Omega(\theta') \left[ \epsilon_{\theta'}q(\theta'|\theta)+
% \delta_\theta(\theta')\int \epsilon_{\ptheta} q(\ptheta|\theta) d\ptheta \right] \\
% %\epsilon_\ptheta[q(\ptheta|\theta)+\delta_\theta(\ptheta)] \\
% &\leq (1-\rho_I) \Omega(\theta)  + C_I  + \int d\theta'
% \Omega(\theta') \epsilon_{\theta'}q(\theta'|\theta)+
% \Omega(\theta)\int \epsilon_{\ptheta} q(\ptheta|\theta) d\ptheta
% %&\leq (1-\rho_I) \max_sA_s(\theta)  + C_I  + \epsilon_\theta \max_sA_s(\theta)
% % + \epsilon_\theta \max_s A_s(\theta)  + C_q
% \end{align*}
% We can bound the middle integral as follows
% \begin{align*}
%   \int d\theta'\Omega(\theta')  \epsilon_{\theta'} q(\theta'|\theta) &=
%   \int_{B_\alpha} d\theta'\Omega(\theta')  \epsilon_{\theta'} q(\theta'|\theta) +
%   \int_{B_\alpha^c} d\theta'\Omega(\theta')  \epsilon_{\theta'}
%   q(\theta'|\theta)  \\
%   &\le \alpha +
%   \int_{B_\alpha^c} d\theta'\Omega(\theta')  \epsilon_{\theta'} q(\theta'|\theta) \\
%   &\le \alpha + \epsilon_\theta \eta_0 \Omega(\theta)+1
% \end{align*}

% From these equations, we have for $\Omega(\theta) > k \Omega(\vartheta)$,
% \begin{align*}
%   \mathbb{E}[\lambda |W'| &+ \lambda_2 \Omega(\vartheta') + \Omega(\theta')|
%   W, \theta, \vartheta, X] \leq
%   \lambda \left[|W|(1 - \delta_1) +  \Omega(\theta) + b\right] + \\
%   & \lambda_2 (1+\eta_0)\Omega(\theta) +
%   \left[(1-\rho_I)  +\epsilon_\theta \eta_0 \right] \Omega(\theta) + C_I
% \end{align*}
% For $\Omega(\theta) < k \Omega(\theta^*)$,
% \begin{align*}
%   \mathbb{E}[\lambda |W'| &+ \lambda_2 \Omega(\vartheta') +
%   \Omega(\theta)| W, \theta, \vartheta, X] \leq
%   \lambda \left[|W|(1 - \delta_1) +  \Omega(\theta)| + b\right] + \\
% & \lambda_2(1+\eta_0) \Omega(\theta) +
%     \frac{(\eta_0 + 1)}{k} \Omega(\vartheta) + C_I \\
%     &= (1-\delta_1)\lambda|W| + \frac{(\eta_0 + 1)}{k\lambda_2} \lambda_2\Omega(\vartheta)+
%     (\lambda + \lambda_2(1+\eta_0))a\Omega(\theta)
% \end{align*}
\end{proof}
