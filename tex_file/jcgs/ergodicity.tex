\newpage
\section{Geometric ergodicity}
We derive conditions under which our proposed algorithm
inherits mixing properties of an `ideal' sampler that operates without
computational constraints. The latter proposes a new parameter $\vartheta$
from distribution $q(\vartheta|\theta)$, and accepts with probability
$\alpha_I(\vartheta,\theta) = 1 \wedge \frac{P(X , \vartheta)q(\theta| \vartheta)}
      {P(X , \theta)q(\vartheta|\theta)}$. The resulting ideal
Markov chain has transition probability
$p_I(\theta'|\theta) = q(\theta'|\theta)\alpha(\theta',\theta) + \left[1-\int d\vartheta
q(\vartheta|\theta)\alpha_I(\vartheta,\theta)\right]\delta_\theta(\theta')$, the first
term corresponding to acceptance, and the second, rejection.

Our main result is Theorem~\ref{thm:geom_erg}, which shows that if the ideal MCMC
sampler is geometrically ergodic, then so is our tractable auxiliary
variable sampler. 
%Assumption~\ref{asmp:cond_num} is the only non-trivial
%one we have to make, requiring that the rate-matrix $A(\theta)$ not be
%arbitrarily ill-conditioned over all
%settings of the parameter $\theta$. This assumption is reasonable from
%both theoretical and practical viewpoints.
%%\boqian{Write $X$ as the observations, which is fixed.}
We first state all our definitions and assumptions, before diving into the details
of our proof.
%For clarity, we focus on the case where the uniformization rate 
%is set as 
%$\Omega(\theta, \vartheta) \doteq \Omega(\theta) + \Omega(\vartheta)$,
%where $A$ is the rate matrix of the Markov jump process and $\Omega(\theta) \doteq k \max_s |A_{ss}(\theta)|$,
%for some $k > 1$.
\begin{assumption}
The uniformization-rate is set as $\Omega(\theta, \vartheta) = \Omega(\theta) + 
\Omega(\vartheta)$, where %$A$ is the rate matrix of the Markov jump process and 
$\Omega(\theta) = k_1 \max_s |A_{ss}(\theta)| + k_0$, for some 
$k_1 > 1, k_0 > 0$.
\label{asmp:unif_rate}
\end{assumption}
%If $\inf_\theta \Mx{\theta} > 0$ (as is often the case),
% we can just set $\Omega(\theta) = \Mx{\theta} + \mx{\theta}$ so that
% $m_2 = \frac{1}{1+\mu}$.}
Although it is possible to specify broader conditions under which our 
result holds, for clarity we focus on this case. This is also how we 
setup our sampler in our experiments.

\begin{assumption}
  $\Omega(\theta)$ is not bounded, and there exists a positive constant $\theta_0$ such that $\forall$ $\theta_x, \theta_y$ satisfying
  $\| \theta_x \| \ge \| \theta_y \| > \theta_0$, we have $\Omega(\theta_x) \ge \Omega(\theta_y)$.  
%  \begin{align*}
%   \lim_{\|\theta\| \rightarrow +\infty} \Omega(\theta) = + \infty.
%  \end{align*}
  \label{asmp:mono_tail}
\end{assumption}
\noindent Proving our result when $\Omega(\theta)$ is bounded is 
easy. %This assumption usually holds under suitable reparametrization,
%and is not critical: we impose 
When this is not the case, this assumption avoids book-keeping by making 
$\Omega(\theta)$ increase monotonically with $\theta$.

\begin{definition}
Let $\pi_\theta$ be the stationary distribution of the MJP with rate-matrix 
$A(\theta)$, and define $D_\theta = \text{diag}(\pi_\theta)$. Define 
$\tilde{A}(\theta) = D_\theta^{-1}A(\theta)D_\theta$, and the 
{\em reversibilization} of $A(\theta)$ as $R_A(\theta) = 
(A(\theta)+\tilde{A}(\theta))/2$. 
\label{def:mjp_symm}
\end{definition}
This definition is from~\cite{fill1991}, who show that the matrix 
$R_A(\theta)$ is reversible with real eigenvalues, the smallest being $0$. 
The larger the second smallest eigenvalue, the faster the MJP converges to its 
stationary distribution $\pi_\theta$.
Note that if the original MJP is reversible, then $R_A(\theta) = A(\theta)$.

\begin{assumption}
For the stationary distribution $\pi_\theta$, there exists $\pi_0 > 0$, such that $\inf_{\theta, s} \pi_{\theta}(s) > \pi_0$.
\label{asmp:stationary_dist_lower_bound}
\end{assumption}
\begin{assumption}
%  The rate-matrix $A(\theta)$ is ergodic for all $\theta$ and satisfies
 % $\inf_\theta \frac{\mx{\theta}}{\Mx{\theta}} = \mu > 0$.
%  $\forall \theta$ satisfying
%  $|\theta| > |\hat{\theta}|$, we have $\Mx{\theta} \asymp
%  \mx{\theta}$.
	Writing $\lambda ^A_2(\theta)$ for the second smallest eigenvalue of
    matrix  $R_A(\theta)$, there exist constants $\mu > 0$ and $\theta_1 > 0$
    such that $ \lambda ^A_2(\theta) \geq \mu \Omega(\theta)$
   for all $\theta$ satisfying $ \| \theta \|> \theta_1$.
  \label{asmp:cond_num}
\end{assumption} 
\noindent %This is the strongest assumption needed to prove our result;
%essentially requiring the rate matrix to be well-conditioned for all 
%large $\theta$. 
%Assumption~\ref{asmp:mono_tail} allows $\max |A_{ss}(\theta)|$ 
%(and therefore $\Omega(\theta)$) to grow with $\|\theta\|$. Here, we 
This is the strongest assumption we need, requiring that 
$\lambda^A_2(\theta)$ (which determines the mixing rate of the MJP) grows 
at least as fast as $\Omega(\theta)$. 
This controls the relative stability of different MJP states, and is 
satisfied when, for example, all elements of $A(\theta)$ grow with 
$\theta$ at similar rates.
%rules 
%out any state $s'$ such that 
%$\frac{|A_{s's'}(\theta)}{\max_s A_{ss}(\theta)}| \rightarrow 0$ as $\theta$ increases.
While not trivial, it holds for the examples we 
considered. %One open question is whether this assumption is necessary.
%We suspect that assumption~\ref{asmp:ideal_geom}, which requires 
%the ideal sampler to be geometrically ergodic, implies this condition,
%however we do not try to prove it here.
%is not sufficient to ensure that $ $ is a 
%stronger requirement, ensures
%parameters of interest\boqian{?: the most
%unstable state cannot have a leaving rate that is larger than that of
%the most stable state by an arbitrary factor.} This condition controls the
%mixing behavior of the MJP (and the embedded Markov chain) for any
%starting state.
%We note that we only need this
%condition to hold in the tails of $\theta$, i.e.\ we only need
%  $\inf_\theta \frac{\mx{\theta}}{\Mx{\theta}} > 0$ for all $\theta$
%  satisfying $|\theta| > h$ for some $h$. For clarity, we restrict
%  ourselves to $h=0$, extending our proof to the general case is
%  straightforward.
%for any interval $\Delta$, we can always find a $\theta_0$ such that the
%MJP has mixed at the end of the interval. The earlier assumption ensures
%this holds for all $\theta \ge \theta_0$.

Let $B(\theta, \theta') = I+\frac{A(\theta)}{\Omega(\theta, \theta')}$
be the transition matrix of the embedded Markov chain, and note that 
it has the same stationary distribution $\pi_\theta$ as $A(\theta)$.
Define the reversibilization $R_B(\theta,\theta')$ from $B(\theta,\theta')$ 
just as we did $R_A(\theta)$ from $A(\theta)$. 
\begin{lemma}
  Suppose $\theta > \max (\theta_0, \theta_1)$ and $\theta'$ satisfies 
$\frac{1}{K_0} \le \frac{\Omega(\theta)'}{\Omega(\theta)} \le K_0 $ for some $K_0 > 0$, which satisfies $(1 + \frac{1}{K_0})k_1 > 2$. 
For all such $(\theta,\theta')$, the Markov chain with transition matrix $B(\theta,\theta')$ converges geometrically to its stationary 
distribution at rates uniformly bounded away from $0$.
%The second largest eigenvalue $\lambda^2_B(\theta,\theta')$ of $R_B$ and  
%second smallest eigenvalue $\lambda^2_B(\theta,\theta')$ of $R_B$ satisfy  
%$\lambda^2_B(\theta,\theta') = 1 - \frac{\lambda^2_A(\theta)}{\Omega(\theta, \theta')}$.

  \label{lem:eig_lemma}
\end{lemma}
\begin{proof}
%it is easy to see that both $A(\theta)$ and $B(\theta,\theta')$ have
%the same stationary distribution $\pi_\theta(s)$. For the matrix $B(\theta,\theta')$, 
A little algebra gives $R_B(\theta,\theta') = I + R_A(\theta)/\Omega(\theta,\theta')$. It 
follows that both $R_A$ and $R_B$ share the same eigenvectors, with 
eigenvalues satisfying 
%and that 
%the eigenvalues of $A(\theta)$ and $B(\theta,\theta')$ satisfy
$\lambda_B(\theta, \theta') = 1 - \frac{\lambda_A(\theta)}{\Omega(\theta,
\theta')}$. In particular, the second largest eigenvalue 
$\lambda^2_B(\theta,\theta')$ of $R_B$ and  
second smallest eigenvalue $\lambda^2_A(\theta,\theta')$ of $R_A$ satisfy  
$\lambda^2_B(\theta,\theta') = 1 - \frac{\lambda^2_A(\theta)}{\Omega(\theta, \theta')}$.
Then $1 - \lambda^2_B(\theta,\theta') = \frac{\lambda^2_A(\theta)}{\Omega(\theta, \theta')} 
\ge \frac{\lambda^2_A(\theta)}{(K_0+1)\Omega(\theta)} 
\ge \frac{\mu}{K_0+1} $. The last two inequalities follow from 
assumptions~\ref{asmp:unif_rate} and~\ref{asmp:cond_num}. \\
Also, we have
\begin{align*}
\Omega(\theta, \theta') &= \Omega(\theta) + \Omega(\theta') > (1 + \frac{1}{K_0})\Omega(\theta)\\
& > (1 + \frac{1}{K_0})k_1\max_s |A_{ss}(\theta)| > 2\max_s |A_{ss}(\theta)| %\ge -2 A(\theta)_{s,s}.
\end{align*}
So for any state $s$, the diagonal element $B(\theta, \theta')_{s,s} = 1 + \frac{A(\theta)_{s,s}}{\Omega(\theta, \theta')}> \frac{1}{2}$.
Following~\cite{fill1991}, this bound on the eigenvalue gap of $R_B(\theta,\theta')$ gives the result.
\qed
\end{proof}

Our overall proof strategy is to show that for $\theta$ and $W$ large enough, 
the condition $\Omega(\theta') \le K_0 \Omega(\theta)$ holds with 
high probability. Lemma~\ref{lem:eig_lemma} then allows the distribution 
over latent states for the continuous-time MJP and its discrete-time 
counterpart embedded in $W$ to be brought arbitrarily close to $\pi_\theta$ 
(and thus to each other).
This will allow our sampler 
to inherit mixing properties of the ideal sampler. For the remaining 
$\theta$ and $W$, we will exploit their boundedness to establish a 
`small-set condition' where the MCMC algorithm forgets its state with 
some probability. These two conditions will be sufficient for 
geometric ergodicity, and the next assumption outlines them for the ideal 
sampler.
% This combined with the
%likelihood $p(X|s, \theta)$ being bounded  gives the result.

\begin{assumption}
For the ideal sampler with transition probability $p_I(\theta'|\theta)$: \\
i) the set $\SM_M=\{\theta:\Omega(\theta)\le M \}$ is a $1$-small set for
every
$M$, i.e.\ there exists a probability measure $\phi$ and a constant
$\kappa_1 > 0$ s.t.\ % $q(\nu | \theta) \alpha_I(\theta, \nu)
$p_I(\theta'|\theta) \geq \kappa_1 \phi(\theta')$ for $\theta \in \SM_M$, \\
\boqian{
$\alpha_I(\theta', \theta; X) q(\theta' | \theta) \ge \kappa_1 \phi(\theta')$?
}\\
ii) for $M$ large enough, $\exists \rho < 1$ s.\ t.\
$\int \Omega(\nu) p_I(\nu|\theta) d\nu
\leq (1-\rho) \Omega(\theta)+C$, $\forall \theta \not\in \SM_M$.
  \label{asmp:ideal_geom}
\end{assumption}
\noindent %Here we specify the properties of the ideal sampler that we
%want our proposed sampler to inherit. 
These two conditions are standard
small-set and drift conditions necessary for the ideal sampler to satisfy
geometric ergodicity. The first implies that for $\theta$ in
$\SM_M$, the ideal sampler forgets its current
location with probability $\kappa_1$. The second condition ensures that
for $\theta$ outside this set, the ideal sampler drifts towards
$\SM_M$. These two conditions together imply geometric
mixing with rate equal or faster than $\kappa_1$~\cite{meyn2009}.
Observe that we have used $\Omega(\theta)$ as the so-called 
Lyapunov-Foster function to define the drift condition for the ideal 
sampler. %and whose sub-level sets form the small sets. 
This is the most natural choice,
though our proof can be tailored to different choices. Similarly, we
could easily allow $\SM_M$ to be an $n$-small set for any $n\ge 1$ (so
the ideal sampler needs $n$ steps before it can forget its current
value in $\SM_M$); we restrict ourselves to the $1$-small case for
clarity.


%\begin{assumption}
%  $\inf_{\nu} \left(\Omega(\theta,\nu) - \Mx{\theta}\right)
%  \ge m\Mx{\theta} \forall \theta$.
 % There exist constants $m_1$ and $m_2$ such that
  %$\inf_\theta \Omega(\theta) = m_1 > 0$ and $\sup_{s,\theta}
  %\frac{A_s(\theta)}{\Omega(\theta)} = m_2 < 1$.
  %\label{asmp:low_bnd}
%\end{assumption}
% \noindent
% \boqian{
% Assumption~\ref{asmp:cond_num} allows us to easily ensure this is
% satisfied by setting $\Omega(\theta) = \Mx{\theta} + \mx{\theta}+m_1$.
% Another option is to set $\Omega(\theta) = \kappa\Mx{\theta} + m_1$ some
% $\kappa > 1$.  If $\inf_\theta \Mx{\theta} > 0$ (as is often the case),
% we can just set $\Omega(\theta) = \Mx{\theta} + \mx{\theta}$ so that
% $m_2 = \frac{1}{1+\mu}$.}


\begin{assumption}
$\exists$ $ \ub > \lb > 0$ s.t.
$\prod P(X | s_o, \theta) \in [\lb, \ub]$ for any state $s_o$ and $\theta$.%\vinayak{Is this for all $s_o$?}
  \label{asmp:obs_bnd}
\end{assumption}
\noindent This assumption follows~\cite{miasojedow2017}, and holds if
%the observation process involves no hard constraints over the latent state, and
$\theta$ does not include parameters of the observation process (or if so,
the likelihood is finite and nonzero for all settings of $\theta$). We can relax this assumption,
though it will introduce technicalities unrelated to our focus, which 
is on complications in parameter inference arising from the continuous-time
dynamics, rather than the observation process. 
%Extensions to the more
%general case should be clear from our proof.

\begin{assumption}
Given the proposal density $q(\nu | \theta)$, $\exists \eta_0 > 0, \theta_2 > 0$ 
such that for $\theta$ satisfying $\| \theta \|  > \theta_2$, 
$ \int_\Theta \Omega(\mu)^2 q(\nu | \theta)d\nu \leq \eta_0 \Omega(\theta)^2.$
\label{asmp:integ_bound}
\end{assumption}
\noindent This mild requirement on the proposal distribution ensures it 
doesn't attempt
to explore large $\theta$'s too aggressively.
\begin{corollary}
Given the proposal density $q(\nu | \theta)$, $\exists \eta_1 > 0, \theta_3 > 0$ such that for $\theta$ 
satisfying $\| \theta \|  > \theta_3$, 
$ \int_\Theta \Omega(\nu) q(\nu | \theta)d\nu \leq \eta_1 \Omega(\theta).$
\label{corol:integ_bound}
\end{corollary}
\begin{proof}
From assumption \ref{asmp:integ_bound},  $\exists \eta_0 > 0$, we have $ \int_\Theta \Omega(\mu)^2 q(\nu | \theta)d\nu \leq \eta_0 \Omega(\theta)^2$ for $\theta$ satisfying $\| \theta \|  > \theta_2$.
For $\theta$ satisfying $\| \theta \|  > \theta_2$, by Cauchy–Schwarz inequality, we have
\begin{align*}
\left[ \int_\Theta \Omega(\nu) q(\nu | \theta) d\nu \right]^2 &\le \int_\Theta \Omega(\nu)^2 q(\nu | \theta) d\nu \cdot \int_\Theta q(\nu | \theta) d\nu \le \eta_0 \Omega(\theta)^2.
\end{align*}
So for $\theta$ satisfying $\| \theta \|  > \theta_2$, we have $\int_\Theta \Omega(\nu) q(\nu | \theta) d\nu \le \sqrt{\eta_0} \Omega(\theta).$
\end{proof}
%\begin{assumption}
%For any positive $\epsilon > 0$ and $K_0 > 1$, there exists $\theta_2$, such that for $\theta > \theta_2$, $P(\mid \frac{q(\theta | \theta')p(\theta')}{q(\theta | \theta')p(\theta)}\mid \leq K_0 | \theta) \geq 1 - \epsilon$, where $\theta' \sim q(\cdot | \theta)$.
%\end{assumption}

%\begin{assumption}
%For any positive $\epsilon$, there exists $h > 0$ and $\theta_2 > 0$, such that $P(\theta' \in B_{\theta, h} | \theta) > 1 - \epsilon$ for $\theta \in B_{0, \theta_2}^c$ .
%\end{assumption}

%\begin{assumption}
%For any positive $\epsilon > 0$ and $K_1 > 1$, there exists $\theta_3$, such that for $\theta > \theta_3$, $P(\frac{1}{K_1}%\frac{\Omega(\theta')}{\Omega(\theta)} \leq K_1 | \theta) \geq 1 - \epsilon$, where $\theta' \sim q(\cdot | \theta)$.
%\end{assumption}
%\begin{assumption}
%There exists $M_1 > 0$ and $\theta_3 > 0$ such that $\frac{-p'(\theta)}{p(\theta)} \le M_1$ for $\theta \in B_{0, \theta_3}^c$.
%\end{assumption}

We need two further assumptions on the proposal distribution $q(\theta'|\theta)$.
\begin{assumption}
For any positive $\epsilon$,  there exist $M_\epsilon$ and $\theta_{4,\epsilon} > 0$,
such that $P(\frac{p(\theta')q(\theta|\theta')}{p(\theta)q(\theta'|\theta)} \le M_\epsilon | \theta) > 1 - \epsilon$,
for $\theta$ satisfying $\| \theta \|  > \theta_{4,\epsilon}$.
  \label{asmp:prior}
\end{assumption}
This holds in our experiments, where $p(\theta)$ is a gamma distribution,
and $q(\theta'|\theta)$, Gaussian.
%\begin{proof}
%From the assumptoin, there exists $\theta_5 > 0$ such that $p(\theta)$ is decreasing with respect to $\Vert \theta \Vert$ for $\theta \in B_{0, \theta_5}^c$. For $\theta > \max(\theta_2, \theta_5)$, $\theta' \in B_{\theta, h}$ with probability greater than $1 - \epsilon$.
%\begin{align*}
%\log \frac{p(\theta')}{p(\theta)} &\le \log\frac{p(\theta - h)}{p(\theta)}
%= \log p(\theta - h) - \log p(\theta)\\
%& \le -h \frac{p'(\chi)}{p(\chi)} \le M_1h
%\end{align*}
%,where $\chi \in B_{\theta, h}$.
%So $p(\theta')/p(\theta) \le M_1 h$ with probability greater than $1 - \epsilon$.
%\end{proof}

%\begin{assumption}
%There exists $M_2 > 0$ and $\theta_4 > 0$ such that $\frac{\Omega'(\theta)}{\Omega(\theta)} \le M_2$ for $\theta \in B_{0, \theta_4}^c$.
%\end{assumption}

%\begin{assumption}
%There exists $\theta_5 > 0$ such that $p(\theta)$ is decreasing with respect to $\Vert \theta \Vert$ for $\theta \in B_{0, \theta_5}^c$.
%\end{assumption}

\begin{assumption}
For any positive $\epsilon$ and  $K > 1$, there exists and $\theta_{5,\epsilon}^K> 0$, such that $P(\frac{\Omega(\theta')}{\Omega(\theta)} \in [1/K, K] | \theta) > 1 - \epsilon$, for $\theta$ satisfying $\| \theta \|  > \theta_{5,\epsilon}^K$.
  \label{asmp:omega}
\end{assumption}
This holds when $q(\theta'|\theta)$ is a centered on $\theta$ and has 
finite variance.

%\begin{proof}
%From the assumptoin, there exists $\theta_5 > 0$ such that $p(\theta)$ is decreasing with respect to $\Vert \theta \Vert$ for $\theta \in B_{0, \theta_5}^c$. For $\theta > \max(\theta_2, \theta_5)$, $\theta' \in B_{\theta, h}$ with probability greater than $1 - \epsilon$.
%\begin{align*}
%\log(\frac{\Omega(\theta')}{\Omega(\theta)}) & \le \log(\frac{\Omega(\theta + h)}{\Omega(\theta)})\\
%& = h \frac{\Omega(\chi')}{\Omega(\chi')} \le h M_2
%\end{align*}
%Similarily, we have
%\begin{align*}
%\log(\frac{\Omega(\theta')}{\Omega(\theta)}) & \ge \log(\frac{\Omega(\theta - h)}{\Omega(\theta)})\\
%& = -h \frac{\Omega(\chi'')}{\Omega(\chi'')} \ge -h M_2
%\end{align*}
%So $\Omega(\theta') / \Omega(\theta) \in [\exp(-hM_2), \exp(hM_2)]$ with probability greater than $1 - \epsilon$.
%\end{proof}

%\begin{assumption}
%For any positive $\epsilon > 0$ and $0 < K_2 < 1$, there exists $\theta_4$, such that for $\theta > \theta_4$, $P(\theta' > \theta K_2 | \theta) %\geq 1 - \epsilon$, where $\theta' \sim q(\cdot | \theta)$.
%\end{assumption}

%\begin{assumption}
%For the above $C \subseteq \Theta$, $\exists$ $\bar{\Omega} > 0$ s.t.
%$\Omega(\theta)  \leq \bar{\Omega}$ for $\forall \theta \in C$.
%\end{assumption}

\begin{definition}
Given $K > 1$, $\epsilon > 0$, define $\theta_\epsilon^K$ as $\max\{ \theta_0, \theta_1, \theta_2, \theta_3, \theta_4^\epsilon, \theta_{5, K}^\epsilon\}$.
\label{def:constant}
\end{definition}


\begin{theorem}
Under the above assumptions, our auxiliary variable sampler is
geometrically ergodic.  \label{thm:geom_erg}
\end{theorem}
\begin{proof}
\noindent This theorem follows from two lemmas we will prove,
lemma~\ref{lem:small_set} showing the existence of a small set (within
which our sampler forgets its
current state with some positive probability), and lemma~\ref{lem:drift}
showing that our sampler drifts towards this set whenever
outside. Together, these two results immediately imply geometric
ergodicity~\citep[Theorems 15.0.1 and Lemma 15.2.8]{meyn2009}.
\end{proof}
We regard our sampler as operating on the space $(\theta,\vartheta,W)$
(recall $\theta$ is the current MJP parameter, $\vartheta$ is the
auxiliary variable, and $W$ is the Poisson grid). An MJP transition step
updates this to $(\theta',\vartheta',W')$ by discarding $\vartheta$, 
sampling $\vartheta'$ from $q$, sampling a new $W'$ given 
$(\theta,\vartheta')$ and then proposing to swap $(\theta,\vartheta')$.
% \boqian{?
% We establish the drift condition via the Lyapunov-Foster function
% $\cV(W,\theta,\vartheta) = \left. \lambda_1|W| + \lambda_2 \Omega(\theta) +
% \Omega(\vartheta) +L \right. := \cV_W(W) + \cV_{\theta}(\theta) +
% \cV_{\theta^*}(\theta^*)$, for settings of $\lambda_1,
%   \lambda_2$ and $L$ we will define later.}
% We point out that if $\Mx{\theta}$ is bounded, then we can replace
% the two $\Omega$ terms by their supremum and absorb them into the
% constant $L$, so that the Lyapunov function only involves $W$.
% The small set will be a level set of this function, i.e.\
% $B_\alpha := \{(\theta,\theta^*,W): \cV(\theta,\vartheta,W) \le \alpha\}$.
% We show that this is a 2-small set (i.e.\ the distribution over states
% after 2-steps of our sampler satisfies $P^2(\cdot|\theta,\vartheta,W)
% \ge \epsilon \nu(\cdot)$ for all $(\theta,\vartheta,W) \in \SM_\alpha$.
% Unlike the ideal sampler, we need a 2-step transition because of the
% exchange move we make,
% where on acceptance $\vartheta'=\theta$, while on rejection, $\theta'=\theta$.

We start with a proposition bounding the self-transition probabilities
of the embedded Markov chain away from $0$.
%\begin{proposition}
%The a priori probability the embedded Markov chain makes a self-transition,
%$P(V_{i + 1}=s|V_i=s,\theta,\theta^*)$ is uniformly bounded away
%from $0$ for all $s,\theta,\theta^*$.
%\end{proposition}
%\begin{proof}
%    & \ge 1-\frac{A_{s_0}(\theta)}{\Mx{\theta}} \ge 1 - \mu
% \intertext{For $\Omega(\theta) \ge k\Omega(\theta^*)$}
% P(V_{i + 1}=s|V_i=s,\theta,\theta^*) &=
%   1 - \frac{A_{s_0}(\theta)}{\Omega(\theta) + \Omega(\theta^*)} \ge
%   1 - \frac{\Omega(\theta)}{m + \Omega(\theta^*)}
% \intertext{For $\Omega(\theta) < k\Omega(\theta^*)$}
% P(V_{i + 1}=s|V_i=s,\theta,\theta^*) &\ge
%   1 - \frac{A_{s_0}(\theta)}{\Omega(\theta) + \Omega(\theta)/k}
%   \ge 1 - \frac{1}{1 + 1/k}
%\end{align*}
%\end{proof}
\begin{proposition}
The a posteriori probability the embedded Markov chain makes a
self-transition,
$P(V_i = V_{i + 1} | W, X, \theta, \vartheta) \ge \delta_1 > 0$,
for %$i = 0, 1, 2, ..., |W|$ and
any $\theta,\vartheta, W$.
\label{prop:self_tr}
\end{proposition}
\begin{proof}
  We use $k_0$ from assumption~\ref{asmp:unif_rate}
  to bound {\em a priori} self-transition probabilities:
  \begin{align*}
    P(V_{i + 1}=s|V_i=s,\theta,\vartheta) &= B_{ss}(\theta,\theta') =
    1 - \frac{A_{s}(\theta)}{\Omega(\theta, \vartheta)}
    \ge 1 - \frac{A_{s}(\theta)}{\Omega(\theta)} \ge 1-\frac{1}{k_0}.
    \intertext{  We then have}
%\mathbb{E}[\mathbb{I}_{\{V_i = V_{i + 1}\}} | W, X, \theta, \theta^*]
  P(V_i = V_{i + 1} | W, X, \theta, \vartheta) &= \sum_v P(V_i = V_{i + 1}
  = v | W, X, \theta, \vartheta)
 =\sum_v \frac{P(V_i = V_{i + 1} = v, X | W, \theta, \vartheta)}{P(X | W,
 \theta, \vartheta)} \\
&=\sum_v \frac{P(X | V_i = V_{i + 1} = v, W, \theta, \vartheta)P( V_i =
V_{i + 1} = v|W, \theta, \vartheta)}{P(X | W, \theta, \vartheta)}\\
& \geq \frac{\lb}{\ub}\sum_v P(V_i = V_{i + 1} = v | W, \theta, \vartheta)\\
&=  \frac{\lb}{\ub} \sum_v P(V_{i + 1} = v | V_i = v, W, \theta,\vartheta)P(V_i = v | \theta, \vartheta) \\
& \geq \frac{\lb}{\ub} (1-\frac{1}{k_0}) \doteq \delta_1 > 0
\end{align*}
\end{proof}


\begin{lemma}
  For $\forall M,h > 0$, the set $B_{h,M} =
\left\lbrace (W, \theta, \vartheta) : |W| \leq h, \theta \in \SM_M, \forall \vartheta
\right\rbrace$ is a small set under our proposed sampler.
\label{lem:small_set}
\end{lemma}
\begin{proof} The $1$-step transition probability of our MCMC algorithm
  consists of two terms, corresponding to the proposed parameter being
  accepted and rejected. Discarding the latter, we have %the bound
\begin{align*}
  P(W',\theta',\vartheta'|W,\theta,\vartheta,X)&\geq
  \delta_\theta(\vartheta') q(\theta'|\theta)
\alpha(\theta', \theta, W';X) \sum_{S,T} P(S,T | W, \theta, \vartheta, X)
P(W'| S, T, \theta', \theta)
\end{align*}
Here we use the fact that the proposal distribution $q(\nu|\theta)$
and $P(W'|S,T,\theta',\theta,X)$ are independent of  $X$.
We further bound the summation over $(S,T)$ by considering only the term
with $S$ a constant. This corresponds to $|W|$ self-transitions, so that
\begin{align*}
P(S=[s_0], T = \emptyset | W, \theta, \vartheta, X) & =
P(S_0=s_0|X,W, \theta, \vartheta)\prod_{i = 0}^{|W| - 1} P(V_{i + 1} = s_0 | V_i = s_0,X,W,\theta,\vartheta) \\ %\prod P(X_{[w_i, w_{i + 1})} | v_i = s_0, \theta)\\
& \geq P(S_0=s_0|X,W, \theta, \vartheta)\delta_1^{|W|} \numberthis %\eta_1
\end{align*}
Given $T = \emptyset$ and $S = s_0$, $W'$ is a Poisson process with rate
$r(\theta', \theta, s_0) = \Omega(\theta',\theta) - A_{s_0}(\theta)$.
%$> \epsilon_1 > 0$.
From the fact that $\Omega(\theta',\theta) = \Omega(\theta') + \Omega(\theta)$,
we apply the superposition theorem to get
%$\PP(r(\theta, \nu, s_0)) = \PP(\Omega(\nu)) \cup \PP(r(\theta, \nu, s_0) - \Omega(\nu))$.
\begin{align*}
P(W' | S, T = \emptyset, \theta', \theta) & \geq P(W' \from
\PP(\Omega(\theta')))
P(\emptyset \from \PP(\Omega(\theta)-A_{s_0}(\theta) ))\\
  & \geq P(W' \from \PP(\Omega(\theta'))) P(\emptyset \from \PP(\Omega(\theta) ))\\
& \geq P(W' \from \PP(\Omega(\theta'))) \exp(-\Omega(\theta)(t_{end} -
t_{start}))\\
& \geq P(W' \from \PP(\Omega(\theta'))) \exp(-M(t_{end}-t_{start}))
\numberthis %\quad (\text{since }\theta\in B_M)
\end{align*}
Thus we have
\begin{align*}
  \sum_{S,T} P(S,T | W, \theta, \vartheta, X) &P(W'| S, T, \theta',
  \theta) \geq \sum_S P(S, T = \emptyset | W, \theta, \vartheta, X)
  P(W' | S, T=\emptyset,\theta', \theta)\\
               &\geq \delta_1^{|W|} \exp(-M(t_{end}-t_{start}))
P(W' \from \PP(\Omega(\theta'))) \numberthis
\end{align*}
Finally consider the acceptance rate:
\begin{align*}
\alpha(\theta', \theta, W'; X) &= 1 \wedge \frac{P(X | W', \theta', \theta)
q(\theta|\theta')p(\theta')}{P(X | W', \theta, \theta')q(\theta'|\theta)p(\theta)}\\
&= 1 \wedge \frac{P(X|W', \theta', \theta) / P(X|\theta')}{P(X|W', \theta,
\theta') / P(X|\theta)} \cdot \frac{P(X | \theta')
q(\theta|\theta')p(\theta')}{P(X | \theta)q(\theta'|\theta)p(\theta)}\\
& \geq 1 \wedge \frac{\lb^2}{\ub^2} \cdot 	\frac{P(X | \theta')
q(\theta|\theta')p(\theta')}{P(X | \theta)q(\theta'|\theta)p(\theta)}\\
& \geq \alpha_I(\theta', \theta;X)\frac{\lb^2}{\ub^2}
\numberthis
\label{eq:acc}
\end{align*}
By assumption \ref{asmp:ideal_geom}, we have the following inequality.
\begin{align*}
P(W', \theta',\vartheta' | W, \theta, \vartheta)  &\ge \frac{\lb^2 }{\ub^2}\delta_1^{h}
\exp(-M(t_{end}-t_{start}))\delta_\theta(\vartheta')P(W' \from \PP(\Omega(\theta')) \\
& \alpha_I(\theta', \theta;X)q(\theta'|\theta)\\
  &\boqian{? \text{assumption \ref{asmp:ideal_geom} should be stronger.}}\\
  & \geq \frac{\lb^2 }{\ub^2}\delta_1^{h}
\exp(-M(t_{end}-t_{start}))\kappa_1\delta_\theta(\vartheta')P(W' \from \PP(\Omega(\theta'))\phi(\theta') \\
  & \doteq \rho_1 \delta_\theta(\vartheta')P(W' \from \PP(\Omega(\theta'))\phi(\theta')
  \intertext{The two-step transition probability is given by}
  P(W'', \theta'',\vartheta'' &| W, \theta, \vartheta)  =
  \int P(W'', \theta'',\vartheta'' | W', \theta', \vartheta')
       P(W', \theta',\vartheta' | W, \theta, \vartheta)
       dW' d\theta' d\vartheta' \\
       &\ge \int_{\SM_{h,M}} P(W'', \theta'',\vartheta'' | W', \theta', \vartheta')
       P(W', \theta',\vartheta' | W, \theta, \vartheta)
       dW' d\theta' d\vartheta' \\
       &\ge \int_{\SM_{h,M}}  \rho_1 \delta_{\theta'}(\vartheta'')P(W'' \from \PP(\Omega(\theta''))\phi(\theta'') \\
         &\qquad \qquad \rho_1 \delta_\theta(\vartheta')P(W' \from \PP(\Omega(\theta'))\phi(\theta')
       dW' d\theta' d\vartheta' \\
       &\ge \rho_1^2 \phi(\theta'')P(W'' \from \PP(\Omega(\theta''))
       \int_{\SM_{h,M}} \!\!\!\! \delta_{\theta'}(\vartheta'')
       F_{Poiss(\Omega(\theta'))}(h)\phi(\theta')
       d\theta'  \\
       & \ge \rho_1^2 P(W'' \from
       \PP(\Omega(\theta''))\phi(\theta'')\phi(\vartheta'')F_{Poiss(\Omega(\vartheta''))}(h)\delta_{\SM_{h,M}}(\vartheta'') \\
       & \ge \rho_1^2 P(W'' \from
       \PP(\Omega(\theta''))\phi(\theta'')
       \phi(\vartheta'')\delta_{\SM_{h,M}}(\vartheta'')\exp(-\Omega(\vartheta''))  \numberthis
       \label{eq:density_lowbound}
\end{align*}
We have
\begin{align*}
\int \phi(\vartheta'')\delta_{\SM_{h,M}}(\vartheta'')\exp(-\Omega(\vartheta'')) d\vartheta'' \le \int_{\SM_{h,M}}\phi(\vartheta'')d\vartheta'' < 1
\end{align*}
So, $\phi(\vartheta'')\delta_{\SM_{h,M}}(\vartheta'')\exp(-\Omega(\vartheta''))$ is proportional to a probability density.
So the inequality \eqref{eq:density_lowbound} is our desired result.
\end{proof}

\noindent Observe that if $\Omega(\theta)$ is bounded uniformly, then
the drift condition need not include $\theta$. To prove the drift
condition for the general case when $\Omega(\theta)$ is unbounded,
we first establish a necessary proposition next.
%is needed when $\Mx{\theta}$ is unbounded as $\theta$ increases.
This result states that the acceptance probabilities of our proposed
sampler and the ideal sampler can be brought arbitrarily close
outside an appropriate small set, so long as $\Omega(\theta)$ is larger
than $\Omega(\theta')$ by a sufficient factor.

\begin{lemma}
%Consider two successive observations, separated by time $t_{diff}$, and
Suppose $\theta$ and $\theta'$ satisfy $\frac{1}{K_0} \le \frac{\Omega(\theta)}{\Omega(\theta')} \leq K_0
$ for some $K_0 > 0$ which satisfies $(1 + \frac{1}{K_0})k_1 > 2$. Write $\mW$ for the
minimum number of grid points between two successive observations.
%$\| \theta'  - \theta \| \leq h$
For any $\epsilon > 0$, there exist  $w_{K_0}^\epsilon,  \theta_{6, K_0}^\epsilon > 0$ such that
$|P(X| W, \theta, \theta') - P(X | \theta)| < \epsilon$
for any $W$ with $\mW > w_{K_0}^\epsilon$ and $\theta $ satisfying $\| \theta \| > \theta_{6, K_0}^\epsilon > 0$.
\label{lem:eigenvalue_lemma}
\end{lemma}
\begin{proof}
Since $B(\theta, \theta') = I+\frac{A(\theta)}{\Omega(\theta, \theta')}$,
%there is a one to one mapping between the eigen values of B and the
it is easy to see that both $A(\theta)$ and $B(\theta,\theta')$ have
the same stationary distribution, call this $P_{st}(s|\theta)$. Further,
the eigenvalues of $A(\theta)$ and $B(\theta,\theta')$ satisfy
$\lambda_B(\theta, \theta') = 1 - \frac{\lambda_A(\theta)}{\Omega(\theta,
\theta')}$, in particular, the second eigenvalue
$\lambda^2_B(\theta,\theta')$ of $B$ (which determines its mixing
properties) equals $1 - \frac{\lambda^2_A(\theta)}{\Omega(\theta, \theta')}$.

\boqian{From lemma \ref{lem:eig_lemma}, for all $\theta, \theta'$ satisfying $\frac{1}{K_0} \le \frac{\Omega(\theta)}{\Omega(\theta')} \leq K_0$ for some $K_0 > 0$ which satisfies $(1 + \frac{1}{K_0})k_1 > 2$, the Markov chain with transition matrix $B(\theta, \theta')$ converges geometrically to its stationary distribution $\pi_\theta$ at rates uniformly bounded away from 0.
}

Our proof strategy is to show that for $\theta$ and $W$ large enough, the
distribution over latent states for the continuous-time MJP and
its discrete-time counterpart embedded in $W$ can be brought arbitrarily
close to $P_{st}$ (and thus to each other). This combined with the
likelihood $p(X|s, \theta)$ being bounded  gives the result.
\boqian{do we need to define $P_{st}$?}
We start with the discrete-time system.
%Write $P^{w}(\cdot | W, \theta, \theta' )$ for the $w$-step transition
%probability (equal to $B^w(\theta, \theta')$).
Assumption~\ref{asmp:cond_num} requires
%that $\lambda^A_2(\theta)$ satisfies
$\lambda^A_2(\theta) \geq \mu \Omega(\theta)$.
Together with the assumption $\Omega(\theta') \leq K_0 \Omega(\theta)$,
this gives
$\frac{\lambda^A_2(\theta)}{\Omega(\theta) + \Omega(\theta')} \geq
\mu / (1 + K_0) $, implying that the second eigenvalue of
$B(\theta, \theta')$ is bounded away from $1$.
Thus for all $(\theta,\theta')$ satisfying the assumptions of the lemma,
and for any initial state, the distribution over states after $w_{0}$ steps
can be brought arbitrarily close to the distribution under the
stationary distribution by setting $w_{0}$ large enough.
%any $\epsilon' > 0$, there exists a $w_0$ such that for all $W$
%with $|W| > w_0$, we have $\| P^{|W|}(\cdot | W, \theta, \theta')
%- P_{st}(\cdot | \theta)\|{\text{TV}} \leq
%{\epsilon'}$.


%From the assumption \ref{asmp:obs_bnd}, we have %$\exists \ \xi_1 > \eta_1 > 0$, such that,
%$0 < \eta_1 \leq P(X | V=v, \theta) \leq \xi_1$, so that
%the following. Assume the state space has $N$ elements.
Write $W_X$ for the indices of the grid $W$ containing observations, and write $V_X$
for the states of the Markov chain at these times.
Let $P_B(V_X | W, \theta, \theta')$ be the probability distribution over
$V_X$ under the Markov chain with transition matrix $B$ given the grid $W$.
Let $P_{st}(V_X|\theta)$ be the probability of the same vector, sampled
independently under the stationary distribution. So
long as $\mW > w_{0}$ for large enough $w_0$,
$P_B(V_X | W, \theta, \theta')$ and  $P_{st}(V_X | W, \theta)$ can be
brought aribtrarily close.
% Then,
% \begin{align*}
%   P(X|W , \theta, \theta') &= \sum_{V_X} P(X | V_X, W, \theta, \theta') P_B(V_X | W, \theta, \theta')\\
% &= \sum_{V_X} P(X | V_{X}, \theta) P_{B}(V_{X} | W, \theta, \theta')\\
% \end{align*}
%  We also define the stationary likelihood as
%  $$P_{st}(X | \theta) = \sum_{V_X} P(X | V_X, \theta) P_{st}(V_X | \theta).$$
Then for any $W$ with $\mW > w_0$, we have
\begin{align*}
|P(X|W , \theta, \theta') - P_{st}(X | \theta)| &= | \sum_{V_X} P(X|V_X, \theta)(P_B(V_X | W, \theta, \theta') -  P_{st}(V_X | \theta))|\\
& \leq \sum_{V_X} P(X | V_X, \theta)|P_B(V_X | W, \theta, \theta') -  P_{st}(V_X | \theta)|\\
& \leq \sum_{V_X} P(X | V_X, \theta){\epsilon'}\\
& \leq N \| X \| \xi_1\epsilon' := \epsilon''
\end{align*}
%So for $W$ with $|W| > W_0$, we have  $|P(X| W, \theta, \theta') - P_{st}(X | \theta)| < \epsilon / 4$. \\
Using the same $w_{0}$ as before, we get a similar result for the
continuous case. First,
\begin{align*}
P(X | \theta) &= \int_W dW P(X , W | \theta, \theta')\\
&= \int_{\mW > w_0} dW P(X | W, \theta, \theta') P(W | \theta, \theta') + \int_{\mW \leq w_0}dW P(X | W, \theta, \theta') P(W | \theta, \theta').
%\sum_{v} P(X | V_{|W|} = v, \theta) P^{(|W|)}(V_{|W|} = v | W, \theta, \theta') P(W|\theta, \theta')
%&= \sum_W \sum_{V_{|W|}} P(X | V_{|W|}, \theta) P^{(|W|)}(V_{|W|} | W, \theta, \theta') P(W|\theta, \theta')\\
%&= \sum_{W s.t. |W| \leq W_0'} \sum_{V_{|W|}} P(X | V_{|W|}, \theta) P^{(|W|)}(V_{|W|} | W, \theta, \theta') P(W|\theta, \theta') + \\
%&\sum_{W s.t. |W| > W_0'} \sum_{V_{|W|}} P(X | V_{|W|}, \theta) P^{(|W|)}(V_{|W|} | W, \theta, \theta') P(W|\theta, \theta')
\end{align*}
% Consider the difference between the first term $\sum_{W s.t. |W| > W_0} P(X | W, \theta, \theta') P(W | \theta, \theta')$ and $P_{st}(X | \theta)P(|W| > W_0 | \theta, \theta')$. Also from the previous derivations, $|P(X| W, \theta, \theta') - P_{st}(X | \theta)| < \epsilon / 4$ for $W$ with $|W| > W_0$.\\
% \begin{align*}
% &|\sum_{W s.t. |W| > W_0} P(X | W, \theta, \theta') P(W | \theta, \theta')  - P_{st}(X | \theta)P(|W| > W_0 | \theta, \theta')|  \leq  \\
% &\sum_{W s.t. |W| > W_0} |P(X | W, \theta, \theta') - P_{st}(X | \theta)|  P(W | \theta, \theta') \leq P(|W| > W_0 | \theta, \theta') \epsilon / 4
% \end{align*}
% For the $W_0$, there exists $\theta_0 > 0$ such that for any $\theta > \theta_0$,and any $\theta'$, $P(|W| \leq W_0 | \theta, \theta') < \frac{\epsilon}{4\xi_1}.$
% For the second term, we have
% \begin{align*}
% &|\sum_{W s.t. |W| \leq W_0} \sum_{v} P(X | V_{|W|} = v, \theta) P^{(|W|)}(V_{|W|} = v | W, \theta, \theta') P(W|\theta, \theta')|  \\
% &\leq \xi_1 P(|W| \leq W_0 | \theta, \theta') \leq \epsilon / 4
% \end{align*}
% Consider the difference between $P(X|\theta)$ and $P_{st}(X|\theta)$ for $\theta > \theta_0$, and any $\theta'$ with $\Omega(\theta') \le K_0\Omega(\theta)$.
% \begin{align*}
% |P(X|\theta) - P_{st}(X|\theta)| &\leq |\sum_{W s.t. |W| > W_0} P(X | W, \theta, \theta') P(W | \theta, \theta')  - P_{st}(X | \theta)P(|W| > W_0 |\theta, \theta')| \\
% &+ |\sum_{W s.t. |W| \leq W_0} \sum_{v} P(X | V_{|W|} = v, \theta) P^{(|W|)}(V_{|W|} = v | W, \theta, \theta') P(W|\theta, \theta')|\\
% &+ |P_{st}(X | \theta)P(|W| \leq W_0 | \theta, \theta')|\\
% &\leq \epsilon / 4 + \epsilon / 4 + \epsilon / 4 = 3\epsilon/4
% \end{align*}
On the set $\{\mW > w_0\}$, $|P(X |W, \theta, \theta')-P_{st}(X|\theta)|
\le \epsilon''$. Since, under uniformization, $W$ comes from a rate-$(\Omega(\theta) + \Omega(\theta'))$
Poisson process, and by choosing $\theta$ large enough, this occurs
with arbitrarily high probability for any $\theta'$. Since the likelihood is bounded, the
second integral can be made arbitrarily small. Finally, from the triangle
inequality,
%So for $\epsilon > 0$, and $K_0 > 0$, if $|W| > W_0$ and $\theta > \theta_0$ and $\Omega(\theta') \leq K_0 \Omega(\theta)$, we have
\begin{align*}
|P(X | \theta) - P(X | W, \theta, \theta')| &\leq |P(X | \theta) -P_{st}(X | \theta) | + | P_{st}(X | \theta) -  P(X | W, \theta, \theta')|\\
       & \leq (\epsilon'' + \epsilon'') + \epsilon'' \doteq \epsilon
\end{align*}

\end{proof}

\begin{proposition}
%Write $P(W', \theta' | W, \theta, \vartheta)$ for the transition kernel of our MCMC sampler given the observations. For any positive $\epsilon$, there exist $\theta_\epsilon > 0$ and corresponding $W_\epsilon > 0$, such that for any $\theta \ge \theta_\epsilon$, and any $\vartheta$, we have
  Let $(W, \theta, \vartheta)$ be the current state of the sampler.
 %Given this, let $E$ be the event that $| \alpha_I(\theta,\vartheta,X) -
 %\alpha(\theta,\vartheta,W',X)| \le \epsilon $.
%Write $q(W', \theta' | W, \theta, \vartheta)$ for the proposal
%distribution of our MCMC sampler.
Then, for any $\epsilon$, there exists $\theta_\epsilon > 0$ as well as a set $\E_\epsilon \subseteq \{(W', \theta'): |\alpha_I(\theta,\theta',X) - \alpha(\theta,\theta',W',X)| \le \epsilon\}$, such that for $\theta$ satisfying $\| \theta \| > \theta_\epsilon$ and any $\theta'$, we have
%and $W_\epsilon > 0$, such that for
%$\theta \ge \theta_\epsilon$, and any $\vartheta$, we have
%  $P(|W'| > W_\epsilon, \theta' > \theta_\epsilon  | W, \theta, \vartheta) \ge 1-\epsilon$.
%  $| \alpha_I(\theta,\vartheta,X) - \alpha(\theta,\vartheta,W',X)|
%  \le \epsilon $ for any $(W', \theta')$ with $|W'| > W_\epsilon$ and $\theta' > \theta_\epsilon$.
  %$(\theta,\theta^*)$ satisfying
$P(E_\epsilon|W,\theta,\theta') > 1-\epsilon$.
\label{prop:mix0}
\end{proposition}
\begin{proof}
\begin{align*}
|\alpha(\theta, \theta', W, X) - \alpha_I(\theta, \theta', X)| &= \ \mid 1 \wedge \frac{P(X | W, \theta' , \theta)q(\theta | \theta')p(\theta')}{P(X | W, \theta , \theta')q(\theta' | \theta)p(\theta)} - 1 \wedge \frac{P(X | \theta')q(\theta | \theta')p(\theta')}{P(X | \theta)q(\theta' | \theta)p(\theta)} \mid \\
& \leq \ \mid \frac{P(X | W, \theta' , \theta)}{P(X | W, \theta , \theta')} - \frac{P(X | \theta')}{P(X | \theta)}\mid  \cdot  \frac{q(\theta | \theta')p(\theta')}{q(\theta' | \theta)p(\theta)}\\
\end{align*}
Given $\epsilon > 0$, from assumption \ref{asmp:prior}, there exist $M_\epsilon > 0$ and $\theta_{1,\epsilon} > 0$, such that $P(\frac{q(\theta | \theta')p(\theta')}{q(\theta' | \theta)p(\theta)}\leq M_\epsilon) > 1 - \epsilon / 2$ {for all } $\theta$ satisfying $ \| \theta \| > \theta_{1,\epsilon}$.
Define $E_1^\epsilon = \{\theta' s.t.\ \frac{q(\theta | \theta')p(\theta')}{q(\theta' | \theta)p(\theta)}\leq M_\epsilon\}$.%, for any $M > 1$.
%From assumption \ref{asmp:prior}, for any $\epsilon$, we can find a
%$\theta_{1,M}^\epsilon$  such that {for all } $\theta$ satisfying $ \| \theta \| > \theta_{1,M}^\epsilon$,
%\begin{align*}
%  P(E_1^\epsilon| \theta) > 1 - \epsilon / 2.
%\end{align*}
%the second term can be bounded with high probability. Also, given large $\theta$ and $\theta'$ which is close to $\theta$, the probability of $|W|$ being large is large.
%For $\epsilon / 4 > 0$,  from assumption \ref{asmp:prior}, there exists $\theta_2 > 0$ and $K_2 > 0$, such that for $\theta > \theta_2$, we have

%For $\epsilon/4 > 0$,
Next, let $E^K_2 = \{\theta' s.t.\ \frac{\Omega(\theta')}{\Omega(\theta)}\in [1/K, K] \}$. For $\epsilon$ and $K$ satisfying $(1 + \frac{1}{K})k_1 > 2$, from assumption \ref{asmp:omega}, we can find $\theta_{2, \epsilon}^K > 0$, such that for all $\theta$ satisfying $\| \theta \| > \theta_{2, \epsilon}^K$,
$P(E^K_2 | \theta) > 1 - \epsilon / 2$.

On this set $E^K_2$, $\Omega(\theta') \le K_\epsilon \Omega(\theta)$ (and also
$\Omega(\theta) \le K_\epsilon \Omega(\theta')$).  Lemma~\ref{lem:eigenvalue_lemma}
ensures that %for $\epsilon$ and $K_3$ and $K_2$, given
%$\frac{1}{K_3}  \leq \frac{\Omega(\theta')}{\Omega(\theta)}  \leq K_3 $,
there exist $\theta_{3,\epsilon}^K > 0, w_\epsilon^K > 0$, such that for $\mW > w_\epsilon^K$,$\theta$ satisfying $ \| \theta \| > \theta_{3,\epsilon}^K$ and $\theta'$ satisfying $ \| \theta' \| > \theta_{3,\epsilon}^K$, we have
$|P(X | W, \theta' , \theta) - P(X | \theta' , \theta)| < \epsilon$, and
$|P(X | W, \theta , \theta') - P(X | \theta , \theta')| < \epsilon$.
Together with the fact that ${P(X | W, \theta , \theta')}$ and ${P(X | \theta)}$
are lower-bounded by $\lb$,
we can find a $\tilde{\theta}_{3,\epsilon}^K > 0, \tilde{w}_\epsilon^K > 0$ so that for $\mW > \tilde{w}_\epsilon^K$,
$\theta$ satisfying $ \| \theta \| > \tilde{\theta}_{3,\epsilon}^K$ and $\theta'$ satisfying $ \| \theta' \| > \tilde{\theta}_{3,\epsilon}^K$, we have
\begin{align*}
|\frac{P(X | W, \theta' , \theta)}{P(X | W, \theta , \theta')} - \frac{P(X | \theta')}{P(X | \theta)}| < \epsilon / M_\epsilon.
\end{align*}
\boqian{
Define $E_{3, \epsilon}^K = \{\theta' s.t. \| \theta'\| > \tilde{\theta}_{3,\epsilon}^K \}$ and $E_{4, \epsilon}^K =\{W s.t. \mW >  \tilde{w}_{\epsilon}^K\}$. Conditioning on $E_1^\epsilon E_2^K E_{3, \epsilon}^K$, we can find $\theta_{4,\epsilon}^K$, such that for $\theta$ satisfying $\| \theta \| > \theta_{4,\epsilon}^K$, the conditional probability 
\begin{align*}
P(E_{4,\epsilon}^K | E_1^\epsilon E_2^K E_{3, \epsilon}^K) > 1 - \epsilon.
\numberthis
\end{align*}
Given current state $(W, \theta)$, set $\theta_\epsilon =
\kappa \max(\theta_{1, \epsilon},\theta_{2, \epsilon}^K,\tilde{\theta}_{3, \epsilon}^K, \theta_{4, \epsilon}^K)$, $\kappa > 1$. \\ For $\theta$ satisfying $ \| \theta \| > \theta_\epsilon$, we have 
\begin{align*}
P(E^K_2 E^\epsilon_1) \ge P(E^K_2) + P(E_1^\epsilon) - 1 \ge 1 - \epsilon.
\end{align*}
When $E^\epsilon_2$ holds, $\theta' \ge \Omega^{-1}(\Omega(\kappa\theta)/K_\epsilon)$.
By choosing $\kappa$ large enough, $\theta'$ satisfying $ \| \theta \| > \theta^K_3$, which means $P(E_3^\epsilon | E_2^\epsilon E_1^\epsilon) = 1$.\\
So \begin{align*}
P(E_1^\epsilon E_2^\epsilon E_3^\epsilon E_4^\epsilon) > (1- \epsilon)^2.
\end{align*} 
Finally, let $E_\epsilon \doteq E_1^\epsilon \cap E_2^\epsilon \cap E_3^\epsilon \cap E_4^\epsilon$ , giving us our result.
}

Call this event $E_{3,K}^\epsilon$. Necessary for $E_3^{K}$ given $E_2^{K}$ is that
$\mW > W^K_3$ for our choice of $K$. As before, we can find a $\theta_{4,\epsilon}$
such that for $\theta$ satisfying $ \| \theta \|  > \theta_{4,\epsilon}$, this probability can brought arbitrarily close to $1$.
%For the $W_4$, there exists $\theta_5 > 0$, such that for $\theta > \theta_5$, and any $\theta'$ with $\frac{1}{K_3}  \leq \frac{\Omega(\theta')}{\Omega(\theta)}  \leq K_3$, we have \begin{align*}
%P(|W'| > W_4 | W, \theta, \theta', \vartheta) > 1- \epsilon / 4.
%\end{align*}

%\boqian{
%There exists $\theta_6 > 0$, such that for $\theta > \theta_6$, we have
%\begin{align*}
%P(\theta' > \theta_4 | \theta) > 1 - \epsilon / 4.
%\end{align*}}

%So given the current state $(W, \theta)$, and $K_0$, $P(\{ \frac{1}{K_0}\leq \frac{\Omega(\theta')}{\Omega(\theta)} \leq K_0 \} \cap \{ \frac{q(\theta | \theta')p(\theta')}{q(\theta' | \theta)p(\theta)} \leq 2\}| \theta ) > 1 - 2\epsilon / 4 $, for any $\theta > \max(\theta_2, \theta_3)$.\\

Given current state $(W, \theta)$, set $\theta_\epsilon =
\kappa \max(\theta_{1, M}^\epsilon,\theta^K_{2, \epsilon},\theta^K_{3, \epsilon}, \theta_4)$, $\kappa > 1$.
%$E_1 = \{ \frac{q(\theta | \theta')p(\theta')}{q(\theta' | \theta)p(\theta)} \leq \max(K_2, 1) \}$ and
%$E_2 = \{ \frac{\Omega(\theta')}{\Omega(\theta)}  \leq K_3\}$.
For $\theta$ satisfying $ \| \theta \| > \theta_\epsilon$, $P(E^K_2 | E_1)>1-\epsilon$, and
\begin{align*}
P(E_1E^K_2) %&= P(E_1) + P(E_2) - P(E_1 \cup E_2) \\
& = P(E_1) P(E^K_2|E_1) \ge (1 - \epsilon)^2.
\end{align*}
When $E^K_2$ holds, $\theta' \ge \Omega^{-1}(\Omega(\kappa\theta)/K)$.
By choosing $\kappa$ large enough, $\theta'$ satisfying $ \| \theta \| > \theta^K_3$. Then, for
$E_3$ to hold, we only need $|W| > W^K_3$. This holds with probability
greater than $1-\epsilon$, so that
\begin{align*}
  P(E_1E_2E_3) > (1- \epsilon)^3.
\end{align*}
Finally, let $E_\epsilon \doteq E_1\cap E_2 \cap E_3$ , giving us our result.
%with probability greater than $1 - \epsilon$.

%Also, the set
%For $K = 1$ and $\epsilon$, there exists $\theta_1$ such that for $\theta > \theta_1$, $\mid \frac{q(\theta | \theta')p(\theta')}{q(\theta' | \theta)p(\theta)}\mid \leq 1$ with probability close to 1. Given $\theta$ and $\theta'$, $P(|W'| > W_0 | \theta, \theta')$ is close to 1.

\end{proof}
%   \begin{proposition}
%     Write $P_{\theta}(W',\theta^*)$ for the conditional distribution
%     over $(W',\theta^*)$ given the current state of the MCMC sampler and the
%     observations, and let
%     $\lim_{\theta\rightarrow\infty} \Mx{\theta} = \infty$. Then for any positive
%     $\epsilon$, there exist $B_\theta$ such that for all
%     $\theta \ge B_\theta$, % with $\Omega(\theta) \ge k\Omega(\vartheta)$,
%     we have
%     $P_{\theta}(\{W'\ s.t.\ |\alpha_I(\theta,\vartheta,X) - \alpha(\theta,\vartheta,W',X)|
%     \le \epsilon)\} \ge 1-\epsilon$.%$(\theta,\theta^*)$ satisfying
%   \label{prop:mix}
%   \end{proposition}
%   \begin{proof}
%   Consider the nearest pair of observations, occuring at times $t$ and
%   $t + \Delta$. %let them be separated by a time interval $\Delta$.
%   By setting $\theta$ high enough, assumption~\ref{asmp:cond_num} ensures
%   that the distribution over waiting times of each state can be concentrated
%   arbitrarily close to $0$. Thus, the distribution over states after an
%   interval $\Delta$ can be brought arbitrarily close to the equilibrium
%   distribution of $A(\theta)$  (call this $p_{\theta}$).

%   Next, recall that the embedded Markov chain has a transition matrix
%   given by $B(\theta^*,\theta) = (I + A(\theta^*)/\Omega(\theta,\theta^*))$.
%   For any setting of $\Omega$, this has the same stationary distribution
%   $p_{\theta}$: this can be verified by multiplying $B(\theta,\theta^*)$ with
%   $p_\theta$. For the embedded Markov chain to be brought close to
%   equilibrium over $[t,t+\Delta]$, we need two conditions, 1) the number of
%   Poisson events must be large enough (to allow sufficient transitions), and 2) the
%   transition matrix $B$ must mix well enough (otherwise even a
%   large number of transition opportunities will not forget the initial state).
%   We show that for $\theta$ large enough, this holds with high probability.
%   Assumption~\ref{asmp:mono_tail} will ensure that this continues to hold
%   for all larger $\theta$ as well.

%   Recall first that the proposal distribution $q(\theta^*|\theta)$ is
%   centered at the current value $\theta$. For any $\epsilon$, we can find
%   an interval $[\theta-h,\theta+h]$ such that for all $\theta$,
%   $q(\theta^* \in [\theta-h,\theta+h]|\theta)
%   \ge 1-\epsilon$. By choosing $\theta$ large
%   enough, we can ensure that %$\Omega(\theta) \approx \Omega(\theta^*)$ (i.e.\
%   $\frac{\Omega(\theta)}{ \Omega(\theta^*)} \in [1-\epsilon,1+\epsilon]$)
%   with probability greater than $1-\epsilon$.
%   %The condition $\Omega(\theta) \ge k\Omega(\vartheta)$,
%   This, together with assumption~\ref{asmp:cond_num}, ensures that with
%   probability greater than $1-\epsilon$, the self-transition probability of $B(\theta,\vartheta)$
%   is bounded away from one, and limits how poorly $B$ can mix.
%   Second, since $W$ comes from a Poisson process with intensity
%   $\Omega(\theta')+ \Omega(\theta) - A_{S(t)}(\theta)$, a large $\theta'$
%   ensures a large $|W'|$ with high probability: by setting $B_\theta$ large
%   enough we can ensure $|W'|$ is large enough with arbitrary probability.

%   Thus, for large enough $B_\theta$ we can ensure that for all $\theta >
%   B_\theta$, the distributions $p(x_{t+\Delta}|s_t)$ and
%   $p(x_{t+\Delta}|W,s_t)$ can be brought arbitrarily close with arbitrarily
%   high probability.
%   By a simple chaining argument, this holds for $p(X|\theta)$ and
%   $p(X|\theta,W)$ as well and so too
%     \vinayak{expand}
%   $\alpha_I(\theta',\theta,X)$ and $\alpha(\theta',\theta,W,X)$
%   \end{proof}
%\begin{lemma}
%Write $P(W' | W, \theta, \vartheta, \theta')$ as the transition density with respect to the grids $W'$, there exists $\psi > 1$, such that $P(W' | W, \theta, \vartheta, \theta') \leq \psi ^{|W'|} \tilde{\Omega}(\theta,\theta')^{|W'|}\exp(-\tilde{\Omega}(\theta,\theta')\tau)$, where $\tilde{\Omega}(\theta, \theta') = (\Omega(\theta) + \Omega(\theta')) / \psi$ and $\tau$ is the length of the time interval.
%\end{lemma}
%\begin{proof}
%As we defined,  there exists $k > 1$, such that $\Omega(\theta) = k \max\{ A_s(\theta)\}$. Recall that given the current state $(W, \theta, \vartheta)$, we first sample $(S, T)$, which is a discrete time Markov chain with rate matrix $B(\theta, \vartheta) = I + \frac{A(\theta)}{\Omega(\theta) + \Omega(\vartheta)}$. Then we sample $W'$ which is a Poisson process with rate $\Omega(\theta) + \Omega(\theta') - A_{S_t}(\theta)$. Integrating out $S, T$, we can get the transition density with respect to the grids $W'$. We have
%\begin{align*}
%P(W' | W, \theta, \vartheta, \theta') &= \sum_{S,T} P(S, T | W, \theta, \vartheta) P(W' | S, T, \theta, \theta') .
%\end{align*}
%Given any $S, T$, the Poisson rate has a lower bound $\frac{k - 1}{k} (\Omega(\theta) +\Omega(\theta'))$. So
%\begin{align*}
%P(W' | S, T, \theta, \theta') &= \prod_{i = 0}^{|T|} (\Omega(\theta) + \Omega(\theta') - A_{S_i}(\theta))^{|W_i'|} \exp(-(\Omega(\theta) + \Omega(\theta') - A_{S_i}(\theta))(T_{i + 1} - T_i))\\
%& \leq \prod_{i = 0}^{|T|} (\frac{k - 1}{k}(\Omega(\theta) + \Omega(\theta')))^{|W_i'|} \exp(- \frac{k - 1}{k} (\Omega(\theta) + \Omega(\theta'))(T_{i + 1} - T_i))(\frac{k}{k - 1})^{|W'|} \\
%&\leq (\frac{k - 1}{k}(\Omega(\theta) + \Omega(\theta')))^{|W'|} \exp(- \frac{k - 1}{k} (\Omega(\theta) + \Omega(\theta'))(T_{|T| + 1} - T_0))(\frac{k}{k - 1})^{|W'|}
%\end{align*}
%So we can set $\psi = \frac{k}{k - 1}$ and $\tilde{\Omega}(\theta, \theta') = (\Omega(\theta) + \Omega(\theta')) / \psi$, and then sum up $S, T$.
%\begin{align*}
%P(W' | W, \theta, \vartheta, \theta') &= \sum_{S,T} P(S, T | W, \theta, \vartheta) P(W' | S, T, \theta, \theta') \\
%& \leq  \psi ^{|W'|} \tilde{\Omega}(\theta,\theta')^{|W'|}\exp(-\tilde{\Omega}(\theta,\theta')\tau)
%\end{align*}
%\end{proof}

\begin{lemma}(drift condition) $\exists \delta_2 \in (0, 1), L > 0$
  s.t.
  $\mathbb{E}\left[\lambda_1|W'| + \Omega(\theta')  | W, \theta, \vartheta, X\right]
  \leq (1 - \delta_2)\left(\lambda_1|W| + \Omega(\theta)   \right) + L$ %where $\lambda = \lceil \frac{(t_{end} - t_{start})k_2(\eta_0 + 1)}{(\eta_1^2 \kappa_1 \mathbb{P}_\phi(C)/\xi_1^2 - \eta_0)} \rceil.$
\label{lem:drift}
\end{lemma}
\begin{proof}
Since $W'=T\cup U'$, we consider $\mathbb{E}[|T| |W,\theta,\vartheta,X]$
and $\mathbb{E}[|U'| | W, \theta, \vartheta, X]$ separately.
An upper bound of $\mathbb{E}[|T| | W,\theta,\vartheta]$ can be derived
directly from proposition~\ref{prop:self_tr}:
\begin{align*}
\mathbb{E}[|T| |W,\theta,\vartheta,X] &= \mathbb{E}[\sum_{i = 0}^{|W|-1}
  \mathbb{I}_{\{ V_{i + 1} \neq V_i \}}| W, \theta, \vartheta, X]\\
&\leq \sum_{i = 0}^{|W| - 1} (1 - \delta_1) = |W|(1 - \delta_1).
\end{align*}
By corollary \ref{corol:integ_bound}, there exists $\eta_1 , \theta_3 > 0$ such that for $\theta$  satisfying $ \| \theta \| > \theta_3$, we have
\begin{align*}
\int \Omega(\nu) q(\nu | \theta)d\nu \leq \eta_1 \Omega(\theta).
\end{align*}
So for $\theta$ satisfying $ \| \theta \| > \theta_3$, we have
\begin{align*}
\mathbb{E}[|U'| |W, \theta, \vartheta, X] &= \mathbb{E}_{S,T, \nu}\mathbb{E}[|U'| | S, T, W, \theta, \nu, X] = \mathbb{E}_{S,T, \nu}\mathbb{E}[|U'| | S, T, W, \theta, \nu] \\
& \leq \mathbb{E}_{S,T, \nu} \left[(t_{end} - t_{start})\Omega(\theta, \nu)\right] = (t_{end} - t_{start})\int \Omega(\theta, \nu) q(\nu | \theta) d\nu\\
& \leq (t_{end} - t_{start})\left[ \left(  \Omega(\theta) +
\int_\Theta \Omega(\nu) q(\nu | \theta)d\nu \right) \right] \\
& \leq (t_{end} - t_{start}) (\eta_1 + 1) \Omega(\theta) %\right]
%\doteq a \Omega(\theta) + b.
\end{align*}

%Next, we note that $\vartheta'$ takes on value $\theta$ with
%probability of acceptance, else it takes the value $\nu$ proposed  from
%$q(\nu|\theta)$. We bound the acceptance
%probability by $1$, so that
%\begin{align}
%\mathbb{E}[\Omega(\vartheta')|\theta,\vartheta,W,X)] &\le \Omega(\theta)
%+ \int d\nu (1-\alpha(\nu,\theta)) q(\nu|\theta) \Omega(\nu)\nonumber \\
 % & \le (1+\eta_0) \Omega(\theta)
%\end{align}

Consider the transition probability over $(W',\theta')$:
\begin{align*}
  P(dW', d\theta'&| W, \theta, \vartheta)
=d\theta' dW' \left[q(\theta' | \theta)
  \sum_{S,T} P(S, T | W, \theta, \vartheta, X)P(W' | S, T, \theta, \theta')
\alpha(\theta, \theta' | W', X)\right. \\
&\left.+ \int q(\nu | \theta) \sum_{S,T} P(S, T|W,\theta,\vartheta,
    X)P(W' | S, T, \theta, \nu) ( 1 - {\alpha(\theta, \nu | W', X)})d\nu
    \delta_\theta(\theta')\right].
\end{align*}
Integrate out $W'$, then we get the following.
\begin{align*}
  P(d\theta'| W, \theta, \vartheta) =d\theta' \int_{W'}dW'&
  \left[q(\theta' | \theta)
    \sum_{S,T} P(S, T | W, \theta, \vartheta, X)P(W' | S, T, \theta,
  \theta')\alpha(\theta, \theta' | W', X)\right.\\
  &\hspace{-1.3in}\left.  + \int q(\nu | \theta) \sum_{S,T} P(S, T |  W, \theta, \vartheta,
X)P(W' | S, T, \theta, \nu) ( 1 - \alpha(\theta, \nu | W', X))d\nu
\delta_\theta(\theta')\right] \\
&:= d\theta' I_1(W, \theta', \theta, \vartheta) + \delta_\theta(\theta')I_2(W, \theta, \vartheta)
\end{align*}
We denote the transition density of $W$, $\sum_{S,T} P(S, T | W, \theta, \vartheta, X)P(W' | S, T, \theta, \theta')$, as $P(W' | W, \theta, \vartheta, \theta', X)$. So we have
\begin{align*}
&I_1(W, \theta', \theta, \vartheta) = q(\theta' | \theta)\int dW'P(W' | W, \theta, \vartheta, \theta', X)\alpha(\theta, \theta' | W', X); \\
&I_2(W, \theta, \vartheta) =\int d\nu  dW'q(\nu | \theta)P(W' | W, \theta, \vartheta, \nu, X)(1 - \alpha(\theta, \nu | W', X)).
\end{align*}

{Consider the second term $I_2$, involving an integral over $(\nu,W')$.
  From Proposition~\ref{prop:mix0}, for any positive $\epsilon$, there
  exists $\theta_\epsilon > 0$ such that the set $E_{\epsilon}$
  (where $|\alpha(\theta, \nu | X,W') - \alpha_I(\theta, \nu | X)| \le
  \epsilon$) has probability greater than $1-\epsilon$.
  Write $I_{2,\E_{\epsilon}}$ for the integral over this set, and
  $I_{2,\E_{\epsilon}^c}$ for that over the complement. Then,
% such that for $\theta > \theta_0$, we have $P(\theta' > \theta_0, |W'| >
% W_0 |\theta, \vartheta, W) > 1 - \epsilon$. Defining $\E_{\epsilon} =
% \{ (\theta', W') | \theta' > \theta_0, |W'| > W_0\}$, we can divide the
% area of integration into two complementary parts: $\E_{\epsilon}$, the
% part of $(\nu,W')$ where
%  $|\alpha(\theta, \nu | X,W') - \alpha_I(\theta, \nu | X)| \le \epsilon$,
%and its complement $\E^c_{\epsilon}$. Call these $I_{2,\E_{\epsilon}}$ and
%$I_{2,\E_{\epsilon}^c}$ respectively.
%For $\theta > \theta_0$, the second term has probability less than $\epsilon$, and using the bound
%$1-\alpha \le 1$, we have
}
%use $\alpha(\theta, \theta' | W', X)\le 1$ and $q(\theta'|\theta) < \epsilon$
%to get}
%%and equation~\eqref{eq:acc} to get}
%  P(d\theta'| &W, \theta, \vartheta)
%\leq d\theta' \epsilon
% &\left.\left(1 -  \int q(\nu | \theta) \sum_S P(S, T|W, \theta,
% \vartheta, X)P(W'|S, T, \theta, \nu)\alpha_I(\theta, \nu)
% \frac{\eta_1^2}{\xi_1^2}d\nu dW' dT\right) \delta_\theta(\theta')\right]\\
%          & \leq d\nu \left[q(\nu | \theta) + \left(1 -
%          \frac{\eta_1^2}{\xi_1^2} \int_\Theta q(\nu |
%      \theta)\alpha_I(\theta, \nu) d\nu\right)\delta_\theta(\theta')\right]
%      \numberthis \label{eq:nu1}
\begin{align*}
I_{2,\E_{\epsilon}}(W, \theta, \vartheta) &= \int_{\E_{\epsilon}} d\nu  dW'q(\nu | \theta)P(W' | W, \theta, \vartheta, \nu, X)(1 - \alpha(\theta, \nu | W', X)) \\
&\le \int_{\E_\epsilon}d\nu dW' q(\nu | \theta)
  P(W' | W, \theta, \vartheta, \nu, X)  [ 1 - (\alpha_I(\theta, \nu | X)-\epsilon)] \\
&\le d\nu \int dW'  q(\nu | \theta)
  P(W' | W, \theta, \vartheta, \nu, X)
  [ 1 - (\alpha_I(\theta, \nu | X)-\epsilon)] \\
  &\le (1+\epsilon)  - \int  q(\nu | \theta) \alpha_I(\theta, \nu | X) d\nu, \\
 & \text{while bounding the acceptance rate by 1,} \\
  I_{2,\E_{\epsilon}^c}(W, \theta, \vartheta)  &= \int_{\E^c_{\epsilon}} d\nu  dW'q(\nu | \theta)P(W' | W, \theta, \vartheta, \nu, X)(1 - \alpha(\theta, \nu | W', X)) \\
  &\le \int_{\E^c_{\epsilon}} d\nu dW'
  q(\nu | \theta) P(W' | W, \theta, \vartheta, \nu, X) \\
 &= \int_{\E^c_{\epsilon}} d\nu dW'
 q(\nu | \theta) P(W' |  W, \theta, \nu, \vartheta, X)
\le \epsilon
\end{align*}
%\int_{W_1}dW' \int_S q(\nu | \theta) &\int \sum_S P(S, T | W, \theta, \vartheta,
%X)P(W' | S, T, \theta, \nu)dT ( 1 - [\alpha_I(\theta, \nu |
%X,W')])d\nu \\
%I_{2,\E_{\epsilon}} \le \int_{\E_\epsilon}dW' d\nu q(\nu | \theta) &
%  \sum_{S,T} P(S, T | W, \theta, \vartheta, X)P(W' | S, T, \theta, \nu)
%  [ 1 - (\alpha_I(\theta, \nu | X)-\epsilon)] \\
%\le \int dW' d\nu q(\nu | \theta) &
%  P(W' | W, \theta, \vartheta, \nu, X)
 % [ 1 - (\alpha_I(\theta, \nu | X)-\epsilon)] \\
 %\le (1+\epsilon) & - \int  q(\nu | \theta) \alpha_I(\theta, \nu | X) d\nu, \text{while} \\
%   \intertext{Then}
%      P(d\theta'| &W, \theta, \vartheta)   \le d\theta' \int_{W'}dW'
%      \left[q(\theta' | \theta) \int \sum_S P(S, T
%   | W, \theta, \vartheta, X)P(W' | S, T, \theta, \theta')dT
%   \left[\alpha_I(\theta', \theta | X)+\epsilon_{}\right]\right.\\
%   &\left. + \int q(\nu | \theta) \int \sum_S P(S, T | W, \theta, \vartheta,
%   X)P(W' | S, T, \theta, \nu)dT ( 1 - [\alpha_I(\theta, \nu |
%   X)-\epsilon_\nu])d\nu \delta_\theta(\theta')\right]  \\
%   & \leq d\theta'\left[q(\theta' | \theta)\left[\alpha_I(\theta', \theta |
%   X)+\epsilon_{\theta'}\right] +
%   \left(1 -  \int q(\nu | \theta) [\alpha_I(\nu, \theta|X)-\epsilon_\nu]
%   d\nu \right) \delta_\theta(\theta')\right]\\
%   & \leq d\theta'\left[q(\theta' | \theta) \alpha_I(\theta', \theta | X) +
%   \left(1 -  \int q(\nu | \theta) \alpha_I(\theta, \nu|X) d\nu \right)
%   \delta_\theta(\theta')\right]+\epsilon_{\theta'}q(\theta'|\theta)+
%   \delta_\theta(\theta')\int \epsilon_{\nu} q(\nu|\theta) d\nu \\
%   & = p_I(\theta'|\theta,X) + \epsilon_{\theta'}q(\theta'|\theta)+
%   \delta_\theta(\theta')\int \epsilon_{\nu} q(\nu|\theta) d\nu
%   \numberthis \label{eq:nu2}
%, and bounding
%$\alpha$ by one on the $W$-set with probability $\epsilon$, we have the
%bound}

Finally, we have
\begin{align*}
  \int \Omega(\theta') P(d\theta'| W, \theta, \vartheta)
  &= \int d\theta' \Omega(\theta') I_1(W, \theta', \theta, \vartheta) + I_2(W, \theta, \vartheta) \Omega(\theta)
\end{align*}
Again, the first term involves an integral over $(\theta',W')$ which as
before, we divide into two regions $\E_\epsilon$ and its complement
$\E^c_\epsilon$.

%From lemma 8, we have
%$P(W' | W, \theta, \vartheta, \theta') \le \psi ^{|W'|} \tilde{\Omega}(\theta,\theta')^{|W'|}\exp(-\tilde{\Omega}(\theta,\theta')\tau)$, where $\tilde{\Omega} = (\Omega(\theta) + \Omega(\theta')) / \psi$ and $0 < \psi < 1$ and $\tau$ is the length of the time interval.\\
%\begin{align*}
%  \tilde{I}_{1,\E^c_\epsilon} &\le  \int_{\E^c_\epsilon}  \Omega(\theta') q(\theta' | \theta)P(W' | W, \theta, \vartheta, \theta')d\theta'dW'\\
%  &=  \int_{\theta' \leq \theta_0}  \Omega(\theta') q(\theta' | \theta)P(W' | W, \theta, \vartheta, \theta')d\theta'dW' +  \int_{\theta' > \theta_0, |W'| < W_0}  \Omega(\theta') q(\theta' | \theta)P(W' | W, \theta, \vartheta, \theta')d\theta'dW'\\
%  & \leq  \Omega(\theta_0) \int_{\theta' > \theta_0}q(\theta' | \theta) d\theta' + \int_{\theta' > \theta_0} \Omega(\theta') q(\theta' | \theta) [ \sum_{w \le W_0}\frac{\tilde{\Omega}(\theta,\theta')^{w}}{w!}\exp(-\tilde{\Omega}(\theta,\theta')\tau)]d\theta'\\
%  & \leq  \Omega(\theta) \epsilon + \int_{\Theta} \Omega(\theta') q(\theta' | \theta)\epsilon d\theta'\\
%  &= \Omega(\theta) \epsilon (1 + \eta_0)
%  \boqian{?}
%  \end{align*}
%  For $\epsilon > 0$, there exists $\theta_1 > 0$, such that for $\theta > \theta_1$ $\int_{\theta' > \theta_0}q(\theta' | \theta) d\theta' \le \epsilon$. There also exists $\theta_2 > 0$, such that for $\theta > \theta_2$ and any $\theta'$,  $\sum_{w \le W_0}\frac{\tilde{\Omega}(\theta,\theta')^{w}}{w!}\exp(-\tilde{\Omega}(\theta,\theta')\tau) < \epsilon$. For $\theta > \max(\theta_0, \theta_1, \theta_2)$, we have
%\begin{align*}
%\tilde{I}_{1,\E^c_\epsilon} &\le \Omega(\theta_0) \int_{\theta' > \theta_0}q(\theta' | \theta) d\theta' + \int_{\theta' > \theta_0} \Omega(\theta') q(\theta' | \theta) [ \sum_{w \le W_0}\frac{\tilde{\Omega}(\theta,\theta')^{w}}{w!}\exp(-\tilde{\Omega}(\theta,\theta')\tau)]d\theta'\\
%& \le  \Omega(\theta_0) \epsilon + \int_{\Theta} \Omega(\theta') q(\theta' | \theta)\epsilon d\theta' \\
%& \le \Omega(\theta_0) \epsilon + \eta_0 \Omega(\theta)\epsilon \\
%& \le (1 + \eta_0) \Omega(\theta)\epsilon
%\end{align*}
$$\int d\theta' \Omega(\theta') I_1(W, \theta', \theta, \vartheta) =
\int_{\E_\epsilon} d\theta' \Omega(\theta') I_1(W, \theta', \theta, \vartheta) +
\int_{\E_\epsilon^c} d\theta' \Omega(\theta') I_1(W, \theta', \theta, \vartheta).
  $$
For the first term, by corollary \ref{corol:integ_bound}, and bounding
the acceptance probability by $\alpha_I + \epsilon$ we get, for $\theta$ satisfying $ \| \theta \| > \theta_3$,
\begin{align*}
\int_{\E_\epsilon} d\theta' \Omega(\theta') I_1(W, \theta', \theta, \vartheta)  &
 \leq \int_{\E_\epsilon} \Omega(\theta')q(\theta' | \theta) (\alpha_I(\theta, \theta', X) + \epsilon) d\theta' \\
& \leq \int q(\theta' | \theta) \Omega(\theta')\alpha_I(\theta, \theta', X) d\theta' + \eta_1 \epsilon \Omega(\theta)
\end{align*}
For $\int_{\E_\epsilon^c} I_1(W, \theta', \theta, \vartheta) \Omega(\theta') d\theta'$, from assumption \ref{asmp:integ_bound}, there exists $\eta_0 , \theta_2 > 0$ such that for $\theta$ satisfying $ \| \theta \| > \theta_2$, we have
\begin{align*}
\int_\Theta \Omega(\nu)^2 q(\nu | \theta)d\nu \leq \eta_0 \Omega(\theta)^2.
\end{align*}
So, by Cauchy–Schwarz inequality and bounding the acceptance probability by one, we have
\begin{align*}
\left[  \int_{\E_\epsilon^c} I_1(W, \theta', \theta, \vartheta) \Omega(\theta') d\theta' \right]^2
& \le \left[  \int_{\E_\epsilon^c}  \Omega(\theta') q(\theta' | \theta) P(W' | \theta, \vartheta, \theta')d\theta' \right]^2\\
& \le \int_{\E_\epsilon^c}  \Omega(\theta')^2 q(\theta' | \theta) P(W' | \theta, \vartheta, \theta')d\theta' \cdot \int_{\E_\epsilon^c} q(\theta' | \theta) P(W' | \theta, \vartheta, \theta')d\theta' \\
& \le  \int  \Omega(\theta')^2 q(\theta' | \theta)d\theta' \cdot \epsilon \\
& \le  \eta_0 \Omega(\theta)^2 \epsilon.
\end{align*}
So we have
\begin{align*}
\int_{\E_\epsilon^c} I_1(W, \theta', \theta, \vartheta) \Omega(\theta') d\theta' \le \sqrt{\eta_0}\Omega(\theta) \sqrt{\epsilon}.
\end{align*}
Then for $\theta$ satisfying $ \| \theta \| >\max(\theta_2, \theta_3, \theta_\epsilon)$, we have the following.
\begin{align*}
  \int \Omega(\theta') P(d\theta'| W, \theta, \vartheta)
  &= \int_{\E_\epsilon} I_1(W, \theta', \theta, \vartheta) \Omega(\theta') d\theta' + \int_{\E_\epsilon^c} I_1(W, \theta', \theta, \vartheta) \Omega(\theta') d\theta' \\ &+ I_{2,\E_{\epsilon}}(W, \theta, \vartheta) + I_{2,\E_{\epsilon}^c}(W, \theta, \vartheta) \\
  & \leq \int q(\theta' | \theta) \Omega(\theta')\alpha_I(\theta, \theta'| X) d\theta'  + \Omega(\theta)\int  q(\nu | \theta) (1 - \alpha_I(\theta, \nu | X)) d\nu+ \\
  &\sqrt{\eta_0}\Omega(\theta) \sqrt{\epsilon} +  \eta_1 \epsilon \Omega(\theta) + 2\epsilon \Omega(\theta)\\
  & \leq (1 - \rho) \Omega(\theta) + (\sqrt{\eta_0} \sqrt{\epsilon} +  \eta_1 \epsilon + 2\epsilon) \Omega(\theta)
\end{align*}
For $\theta$ satisfying $ \| \theta \| >\max(\theta_2, \theta_3, \theta_\epsilon)$, we have %\boqian{need to change afterwards}
\begin{align*}
\mathbb{E}[\lambda_1 | W'| &+ \Omega(\theta')| W, \theta, \vartheta, X] \le \lambda_1(1 - \delta_1)|W| + \lambda_1 (t_{end} - t_{start}) (1 + \eta_1)\Omega(\theta) \\
&+  (1 - \rho) \Omega(\theta) + (\sqrt{\eta_0} \sqrt{\epsilon} +  \eta_1 \epsilon + 2\epsilon) \Omega(\theta)\\
& = (1 - \delta_1)\lambda_1 |W| + [1 - (\rho - \lambda_1 (t_{end} - t_{start}) (1 + \eta_1) - (2 + \eta_1)\epsilon - \sqrt{\eta_0} \sqrt{\epsilon})]\Omega(\theta)
\end{align*}
There exists  $\tilde{\epsilon} > 0$ and $\tilde{\lambda}_1 >0 $ , such that $\rho >  \tilde{\lambda}_1 (t_{end} - t_{start}) (1 + \eta_1) + (2 + \eta_1)\tilde{\epsilon} + \sqrt{\eta_0} \sqrt{\tilde{\epsilon}}$, and denote $\rho_2  \doteq \rho - \tilde{\lambda}_1 (t_{end} - t_{start}) (1 + \eta_1) + (2 + \eta_1)\tilde{\epsilon} + \sqrt{\eta_0} \sqrt{\tilde{\epsilon}}
 > 0$. Let $\tilde{\rho} = \min(\rho_2, \delta_1)$. So there exists a small set $C$ such that \begin{align*}
\mathbb{E}[\tilde{\lambda}_1 | W'| + \Omega(\theta')| W, \theta, \vartheta, X] &\le (1 - \tilde{\rho}) [\tilde{\lambda}_1 | W| + \Omega(\theta)] +
\mathbb{I}_{C} \sup_{(W, \theta)} \{ \mathbb{E}[\tilde{\lambda}_1 | W'| + \Omega(\theta')| W, \theta, \vartheta, X] \}.
\end{align*}
So the drift condition holds.
% From equation~\eqref{eq:nu2} and assumption 2, we have for
% $\Omega(\theta) > k\Omega(\vartheta)$
% \begin{align*}
% \int \Omega(\theta')P(d\theta'| W, \theta, \vartheta)  &\leq
% \int \Omega(\theta') p_I(\theta'|\theta,X) d\theta' +\int d\theta'
% \Omega(\theta') \left[ \epsilon_{\theta'}q(\theta'|\theta)+
% \delta_\theta(\theta')\int \epsilon_{\nu} q(\nu|\theta) d\nu \right] \\
% %\epsilon_\nu[q(\nu|\theta)+\delta_\theta(\nu)] \\
% &\leq (1-\rho_I) \Omega(\theta)  + C_I  + \int d\theta'
% \Omega(\theta') \epsilon_{\theta'}q(\theta'|\theta)+
% \Omega(\theta)\int \epsilon_{\nu} q(\nu|\theta) d\nu
% %&\leq (1-\rho_I) \max_sA_s(\theta)  + C_I  + \epsilon_\theta \max_sA_s(\theta)
% % + \epsilon_\theta \max_s A_s(\theta)  + C_q
% \end{align*}
% We can bound the middle integral as follows
% \begin{align*}
%   \int d\theta'\Omega(\theta')  \epsilon_{\theta'} q(\theta'|\theta) &=
%   \int_{B_\alpha} d\theta'\Omega(\theta')  \epsilon_{\theta'} q(\theta'|\theta) +
%   \int_{B_\alpha^c} d\theta'\Omega(\theta')  \epsilon_{\theta'}
%   q(\theta'|\theta)  \\
%   &\le \alpha +
%   \int_{B_\alpha^c} d\theta'\Omega(\theta')  \epsilon_{\theta'} q(\theta'|\theta) \\
%   &\le \alpha + \epsilon_\theta \eta_0 \Omega(\theta)+1
% \end{align*}

% From these equations, we have for $\Omega(\theta) > k \Omega(\vartheta)$,
% \begin{align*}
%   \mathbb{E}[\lambda |W'| &+ \lambda_2 \Omega(\vartheta') + \Omega(\theta')|
%   W, \theta, \vartheta, X] \leq
%   \lambda \left[|W|(1 - \delta_1) +  \Omega(\theta) + b\right] + \\
%   & \lambda_2 (1+\eta_0)\Omega(\theta) +
%   \left[(1-\rho_I)  +\epsilon_\theta \eta_0 \right] \Omega(\theta) + C_I
% \end{align*}
% For $\Omega(\theta) < k \Omega(\theta^*)$,
% \begin{align*}
%   \mathbb{E}[\lambda |W'| &+ \lambda_2 \Omega(\vartheta') +
%   \Omega(\theta)| W, \theta, \vartheta, X] \leq
%   \lambda \left[|W|(1 - \delta_1) +  \Omega(\theta)| + b\right] + \\
% & \lambda_2(1+\eta_0) \Omega(\theta) +
%     \frac{(\eta_0 + 1)}{k} \Omega(\vartheta) + C_I \\
%     &= (1-\delta_1)\lambda|W| + \frac{(\eta_0 + 1)}{k\lambda_2} \lambda_2\Omega(\vartheta)+
%     (\lambda + \lambda_2(1+\eta_0))a\Omega(\theta)
% \end{align*}
\end{proof}
