\section{Geometrical ergodicity}
\begin{assumption}
  $\exists$ $m > 0$, s.t.\ $\inf \Omega(\theta,\nu) \ge m$.
\end{assumption}
\noindent This is a minimal requirement for our sampler to mix. This holds
if
$\inf_\theta \Mx(\theta) > 0$, else we must set 
$\Omega(\theta, \nu) = \left(  \max_s|A_{ss}(\theta)| +  
\max_s|A_{ss}(\nu)| \right) + m$, for $m > 0$. 

\begin{assumption}
$\exists$ $ \xi_1 > \eta_1 > 0$ s.t. $\prod P(x_o | s_o, \theta) \in (\eta_1, \xi_1)$.
\end{assumption}
\noindent This assumption holds if the observation process involves no hard
constraints over the latent state, and $\theta$ does not include
parameters of the observation process. While we can relax this assumption,
our focus is on complications arising from the continuous-time dynamics
(rather than a general observation process), and for simplicity, we 
restrict ourselves to this.

\begin{assumption}
For the ideal sampler with the proposal density $q(\nu| \theta)$ and the 
acceptance rate $\alpha_I(\theta, \nu) = 1 \wedge 
\frac{P(X | \nu)q(\theta| \nu)p(\nu)}{P(X | \theta)q(\nu|
\theta)p(\theta)}$, \\
i) $\exists$ a probability measure $\phi$, a constant $\kappa_1 > 0$ and a
set $C \subseteq \Theta$, s.t. $q(\nu | \theta) \alpha_I(\theta, \nu) \geq
\kappa_1 \phi(\nu)$ for $\theta \in C$, \\
ii) for $\theta \not\in C$, $\exists \rho < 1$ such that
$\int \max_sA_s(\nu) p_I(\nu|\theta,X) d\nu 
\leq (1-\rho_I) \max_sA_s(\theta)$
\end{assumption}

\begin{assumption}
Given the proposal density $q(\nu | \theta)$, $\exists \eta_0 > 0$ s.t. $$ \int_\Theta \max_s|A_{ss}(\nu)| q(\nu | \theta)d\nu \leq \eta_0 \max_s|A_{ss}(\theta)|.$$
\end{assumption}

%\begin{assumption}
%For the above constants, $\frac{\eta_1^2}{\xi_1^2} \kappa_1 \mathbb{P}_\phi(C)  - \eta_0 \in (0, 1)$, where $\mathbb{P}_\phi(C) = \int_C\phi(\mu)d\mu$.
%\end{assumption}
%\vinayak{What is $p_{\phi}$. Is this a strong assumption? Since the first term on the LHS is less than 1,
%  can we write this as $\frac{\eta_1^2}{\xi_1^2} \kappa_1 \mathbb{P}_\phi(C) >\eta_0$? Also this requires
%$\eta_0 <1$. }
%\begin{assumption}
%$$\exists \rho_1 \geq \left(1 - \frac{\eta_1^2}{\xi_1^2}\right)\sup_\theta \int q(\mu| \theta)\alpha_I(\theta, \mu)d\mu , \text{ and } \rho_1 \leq \sup_\theta \int q(\mu| \theta)\alpha_I(\theta, \mu)d\mu, s.t.$$
%$$\int \max_s |A_{ss}(\nu)|q(\nu | \theta)d\nu + \left(1 - \int q(\mu | \theta)\alpha_I(\theta, \mu)d\mu\right)\max_s |A_{ss}(\theta)| \leq (1 - \rho_1) \max_s |A_{ss}(\theta)|, \forall \theta \in \Theta.$$
%This implies that the drift condition of the ideal sampler.
%\end{assumption}

\begin{assumption}
For the above $C \subseteq \Theta$, $\exists$ $\bar{\Omega} > 0$ s.t. 
$\Omega(\theta)  \leq \bar{\Omega}$ for $\forall \theta \in C$.
\end{assumption}


\begin{theorem}
Under the above assumptions, for $\forall h > 0$, the set $\left\lbrace (W, \theta, \theta^*) | |W| \leq h, \theta \in C \right\rbrace$ is a small set.
\end{theorem}
\begin{proof}
\begin{align*}
P(W', \nu, \theta | W, \theta, \theta^*,X) &\geq q(\nu | \theta)
         \alpha(\theta, \nu, W',X) \int_T \sum_S P(S,T | W, \theta, \theta^*, X) P(W'| S, T, \theta, \nu)dT  
\end{align*}
Here we use the fact that both the proposal distribution $q(\nu|\theta)$
and $P(W'|S,T,\theta,\nu,X))$ are independent of  $X$.
By assumption 1,  $\Omega(\theta, \theta^*) \geq k_1 \max\{ 
\max_s|A_{ss}(\theta)|, \max_s|A_{ss}(\theta^*)|\} + \epsilon_1 $. 
$\forall i, s_0$ the self transition probability $P(V_{i + 1} = s_0 | 
V_i = s_0) = 1 - \frac{|A_{s_0s_0}(\theta)|}{\Omega(\theta, \theta^*)} 
\geq 1 - \frac{\Mx(\theta)|}{m + \Mx(\theta)|} \ge \frac{m}{m+M} := \mu$.
\begin{align*}
P(S=s_0, T = \emptyset | W, \theta, \theta^*, X) & = p_0(s_0)\prod P(V_{i + 1} = s_0 | V_i = s_0) \prod P(X_{[w_i, w_{i + 1})} | v_i = s_0, \theta)\\
& \geq p_0(s_0)\mu^{|W|}\eta_1
\end{align*}
Given $T = \emptyset$ and $S = [s_0]$, $W'$ is a Poisson process with rate 
$r(\theta, \nu, s_0) = \Mx(\theta)+\Mx(\nu) - |A|_{s_0}(\theta)$.
%$> \epsilon_1 > 0$. 
Applying the superposition theorem, $\PP(r(\theta, \nu, s_0)) =
\PP(\Omega(\nu)) \cup \PP(r(\theta, \nu, s_0) - \Omega(\nu))$.
\begin{align*}
P(W' | S, T = \emptyset, \theta, \nu) & \geq P(W' \from \PP(\Omega(\nu)))
P(\emptyset \from \PP(\Omega(\theta) ))\\
& \geq P(W' \from \PP(\Omega(\nu))) \exp(-\Omega(\theta)(t_{end} -
t_{start}))%\\
%& \geq P(W' \from \PP(\Omega(\nu))) \exp(-\bar{\Omega}(t_{end} - t_{start}))
\end{align*}
\begin{align*}
  \int_T \sum_S P(S,T | W, \theta, \theta^*, X) &P(W'| S, T, \theta, \nu)dT \geq \sum_S P(S, T = \emptyset | W, \theta, \theta^*, X) P(W' | S, T=\emptyset,\theta, \nu)\\
         &\geq \sum_S \mu^{|W|}\eta_1 \exp(-{\Omega(\theta)}(t_{end} - t_{start})) 
P(W' \from \PP(\Omega(\nu)))
\end{align*}
Consider the acceptance rate.
\begin{align*}
\alpha(\theta, \nu, W') &= 1 \wedge \frac{P(X | W', \nu, \theta) q(\theta|\nu)p(\nu)}{P(X | W', \theta, \nu)q(\nu|\theta)p(\theta)}\\
&= 1 \wedge \frac{P(X|W', \nu, \theta) / P(X|\nu)}{P(X|W', \theta, \nu) / P(X|\theta)} \frac{P(X | \nu) q(\theta|\nu)p(\nu)}{P(X | \theta)q(\nu|\theta)p(\theta)}\\
& \geq 1 \wedge \frac{\eta_1^2}{\xi_1^2} 	\frac{P(X | \nu) q(\theta|\nu)p(\nu)}{P(X | \theta)q(\nu|\theta)p(\theta)}\\
& \geq \alpha_I(\theta, \nu)\frac{\eta_1^2}{\xi_1^2}
\end{align*}
By assumption 4, we have the following inequality.
\begin{align*}
P(W', \nu, \theta | W, \theta, \theta^*) \geq \frac{(1 - 1/k_1)^{h}
\eta_1^3 \exp(-\bar{\Omega}(t_{end} - t_{start}))\kappa_1}{\xi_1^2} 
 P(W'
   \from \PP(\Omega(\nu))\phi(\nu)
\end{align*}
\end{proof}

\begin{lemma}
$\exists \delta_1 \in (0, 1)$ s.t. $\mathbb{E}[\mathbb{I}_{\{V_i = V_{i + 1}\}} | W, X, \theta, \theta^*] \geq \delta_1$ for $i = 0, 1, 2, ..., |W|$. 
\end{lemma}
\begin{proof}
\begin{align*}
\mathbb{E}[\mathbb{I}_{\{V_i = V_{i + 1}\}} | W, X, \theta, \theta^*] &= P(V_i = V_{i + 1} | W, X, \theta, \theta^*) = \sum_v P(V_i = V_{i + 1} = v | W, X, \theta, \theta^*)\\
& =\sum_v \frac{P(V_i = V_{i + 1} = v, X | W, \theta, \theta^*)}{P(X | W, \theta, \theta^*)} \\
&=\sum_v \frac{P(X | V_i = V_{i + 1} = v, W, \theta, \theta^*)P( V_i = V_{i + 1} = v|W, \theta, \theta^*)}{P(X | W, \theta, \theta^*)}\\
& \geq \eta_1\sum_v P(V_i = V_{i + 1} = v | W) /\xi_1 =  \eta_1 \sum_v P(V_{i + 1} = v | V_i = v, W)P(V_i = v) /\xi_1 \\
& \geq \frac{\eta_1 (1 - 1/k_1)}{\xi_1} \doteq \delta_1 
\end{align*}
\end{proof}

\begin{lemma}
For any $\epsilon$, there exists a $\theta_0$ and $W_0$ such that for all 
$\theta \ge \theta_0$ and $W$ such that $|W|\ge|W_0|$, we have 
$|\alpha_I(\theta,\nu,X) - \alpha(\theta,\nu,W,X)| \le \epsilon$
\end{lemma}
\begin{proof}
  Consider the nearest pair of observations, let them be separated by a
  time interval $\Delta$. By setting $\theta$ high enough, the
  distribution over states after an interval $\Delta$ can be brought 
  arbitrarily close to the equilibrium distribution of $A(\theta)$, call
  this $p_{\theta}$.

  Observe that the embedded Markov chain has a transition matrix given
  by $B(\theta) = (I + A(\theta)/\Omega(\theta))$. For any valid
  $\Omega$, this has the same stationary distribution $p_{\theta}$:
  this is a straightforward consequence of uniformization (but see~\cite{}
  for a formal proof). By choosing a grid $W$ over $\Delta$ with $|W|$
  large enough, this Markov chain can be brought arbitrarily close to
  $p_{\theta}$. Thus for any such pair $(\theta,W)$, terms 
  $p(x_{t+\Delta}|s_t)$ and $p(x_{t+\Delta}|W,s_t)$ can be brought
  arbitrarily close for all values of $s_t$. By a simple chaining,
  $p(X|\theta)$ and $p(X|\theta,W)$ can be brought arbitrarily close, and
  thus $\alpha_I(\nu,\tau,X)$ and $\alpha(\nu,\theta,W,X)$.
\end{proof}

\begin{theorem}(drift condition) $\exists \delta_2 \in (0, 1), L > 0$ 
  s.t. 
  $$\mathbb{E}\left[|W'| + \lambda \max_s |A_{ss}(\nu)| | W, \theta, \theta^*, X\right] \leq (1 - \delta_2)\left(|W| + \lambda \max_s |A_{ss}(\theta)| \right) + L.$$
where $\lambda = \lceil \frac{(t_{end} - t_{start})k_2(\eta_0 + 1)}{(\eta_1^2 \kappa_1 \mathbb{P}_\phi(C)/\xi_1^2 - \eta_0)} \rceil.$
\end{theorem}
\begin{proof}
Since $W' = T \cup U'$, we consider $\mathbb{E}[|T| | W, \theta, \theta^*, X]$ and $\mathbb{E}[|U'| | W, \theta, \theta^*, X]$ respectively.
An upper bound of $\mathbb{E}[|T| | W, \theta, \theta^*]$ can be derived directly from lemma 1.
\begin{align*}
\mathbb{E}[|T| | W, \theta, \theta^*, X] &= \mathbb{E}[\sum_{i = 0}^{|W| - 1} \mathbb{I}_{\{ V_{i + 1} \neq V_i \}}| W, \theta, \theta^*, X]\\
&\leq \sum_{i = 0}^{|W| - 1} (1 - \delta_1) = |W|(1 - \delta_1).
\end{align*}
\begin{align*}
\mathbb{E}[|U'| |W, \theta, \theta^*, X] &= \mathbb{E}_{S,T, \nu}\mathbb{E}[|U'| | S, T, W, \theta, \nu, X] = \mathbb{E}_{S,T, \nu}\mathbb{E}[|U'| | S, T, W, \theta, \nu] \\
& \leq \mathbb{E}_{S,T, \nu} \left[(t_{end} - t_{start})\Omega(\theta, \nu)\right] = (t_{end} - t_{start})\int \Omega(\theta, \nu) q(\nu | \theta) d\nu\\
& \leq (t_{end} - t_{start})\left[ k_2 \left(  \max_s|A_{ss}(\theta)| +  \int_\Theta \max_s|A_{ss}(\nu)|q(\nu | \theta)d\nu \right) + \epsilon_2 \right] \\
& \leq (t_{end} - t_{start}) \left[ k_2 (\eta_0 + 1) \max_s|A_{ss}(\theta)| + \epsilon_2 \right] \doteq a \max_s|A_{ss}(\theta)| + b.
\end{align*}
Consider the transition kernel of the sampler.

\begin{align*}
P(dW', d\nu, \theta | W, \theta, \theta^*) &=d\nu dW' \left[q(\nu | \theta) \int \sum_S P(S, T | W, \theta, \theta^*, X)P(W' | S, T, \theta, \nu)dT\alpha(\theta, \nu | W', X)\right. \\
                                           &\left.+ \int q(\mu | \theta) \int \sum_S P(S, T | W, \theta, \theta^*, X)P(W' | S, T, \theta, \mu)dT ( 1 - {\alpha(\theta, \mu | W', X)})d\mu \delta_\theta(\nu)\right].
\end{align*}
Integrate out $W'$, then we get the following.
\begin{align*}
  P(d\nu| &W, \theta, \theta^*) =d\nu \int_{W'}dW' \left[q(\nu | \theta) \int \sum_S P(S, T | W, \theta, \theta^*, X)P(W' | S, T, \theta, \nu)dT\alpha(\theta, \nu | W', X\right.)\\
&\left. + \int q(\mu | \theta) \int \sum_S P(S, T | W, \theta, \theta^*, X)P(W' | S, T, \theta, \mu)dT ( 1 - \alpha(\theta, \mu | W', X))d\mu \delta_\theta(\nu)\right]\\
          &\le d\nu \int_{W'}dW' \left[q(\nu | \theta) \int \sum_S P(S, T
| W, \theta, \theta^*, X)P(W' | S, T, \theta, \nu)dT\left[\alpha_I(\theta,
\nu | X)+\epsilon_\nu\right]\right.\\
&\left. + \int q(\mu | \theta) \int \sum_S P(S, T | W, \theta, \theta^*,
X)P(W' | S, T, \theta, \mu)dT ( 1 - [\alpha_I(\theta, \mu |
X)-\epsilon_\nu])d\mu \delta_\theta(\nu)\right]\\
& \leq d\nu\left[q(\nu | \theta)\left[\alpha_I(\theta, \nu |
X)+\epsilon_\nu\right] +
\left(1 -  \int q(\mu | \theta) [\alpha_I(\theta, \mu|X)-\epsilon_\nu] d\mu \right) \delta_\theta(\nu)\right]\\
& \leq d\nu\left[q(\nu | \theta) \alpha_I(\theta, \nu | X) +
\left(1 -  \int q(\mu | \theta) \alpha_I(\theta, \mu|X) d\mu \right) \delta_\theta(\nu)\right]+
\epsilon_\nu[q(\nu|\theta)+\delta_\theta(\nu)] \\
& = p_I(\nu|\theta,X) + \epsilon_\nu[q(\nu|\theta)+\delta_\theta(\nu)]
\end{align*}
Because of assumption 2, we have the following.
\begin{align*}
\int \max_sA_s(\nu)P(d\nu| W, \theta, \theta^*)  &\leq 
\int \max_sA_s(\nu) p_I(\nu|\theta,X) d\nu +\int d\nu\max_s A_s(\nu)  \epsilon_\nu[q(\nu|\theta)+\delta_\theta(\nu)] \\ 
&\leq (1-\rho_I) \max_sA_s(\theta)  + C_I  + \epsilon_\theta \max_sA_s(\theta)
 +\int d\nu\max_s A_s(\nu)  \epsilon_\nu q(\nu|\theta)   \\
&\leq (1-\rho_I) \max_sA_s(\theta)  + C_I  + \epsilon_\theta \max_sA_s(\theta)
 + \epsilon_\theta \max_s A_s(\theta)  + C_q  
\end{align*}

From these equations, we have
\begin{align*}
  \mathbb{E}\left[\lambda |W'| + \max_s |A_{ss}(\nu)|\right. & \left.| W, \theta, \theta^*, X\right] \leq 
  \lambda \left[|W|(1 - \delta_1) +  \max_s|A_{ss}(\theta)| + b\right] + \\
  &\left[(1-\rho_I) \max_sA_s(\theta)  +2 \epsilon \max_sA_s(\theta)
\right] \max_sA_s(\theta) + C_I 
\end{align*}
If we set  $\lambda$ small enough, then the drift condition holds.
\end{proof}
